
\PropositionG{1}
{Die Welt ist alles, was der Fall ist.\footnote{Die Decimalzahlen als Nummern der einzelnen S&auml;tze deuten das logische
Gewicht der S&auml;tze an, den Nachdruck, der auf ihnen in meiner Darstellung liegt\DPtypo{,}{.}
Die S&auml;tze \textit{n}.1, \textit{n}.2, \textit{n}.3, etc., sind Bemerkungen zum \DPtypo{S&auml;tze}{Satze} No.\textit{n}; die S&auml;tze \textit{n}.\textit{m}1,
\textit{n}.\textit{m}2, etc.\ Bemerkungen zum Satze No.\textit{n}.\textit{m}; und so weiter.}}


\PropositionG{1.1}
{Die Welt ist die Gesamtheit der Tatsachen,
nicht der Dinge.}


\PropositionG{1.11}
{Die Welt ist durch die Tatsachen bestimmt und
dadurch, dass es \Emph{alle} Tatsachen sind.}


\PropositionG{1.12}
{Denn, die Gesamtheit der Tatsachen bestimmt,
was der Fall ist und auch, was alles nicht der Fall ist.}


\PropositionG{1.13}
{Die Tatsachen im logischen Raum sind die Welt.}


\PropositionG{1.2}
{Die Welt zerf&auml;llt in Tatsachen.}


\PropositionG{1.21}
{Eines kann der Fall sein oder nicht der Fall sein
und alles &uuml;brige gleich bleiben.}


\PropositionG{2}
{Was der Fall ist, die Tatsache, ist das Bestehen
von Sachverhalten.}


\PropositionG{2.01}
{Der Sachverhalt ist eine Verbindung von
Gegenst&auml;nden. (Sachen, Dingen.)}


\PropositionG{2.011}
{Es ist dem Ding wesentlich, der Bestandteil
eines Sachverhaltes sein zu k&ouml;nnen.}


\PropositionG{2.012}
{In der Logik ist nichts zuf&auml;llig: Wenn das Ding
im Sachverhalt vorkommen \Emph{kann}, so muss die
M&ouml;glichkeit des Sachverhaltes im Ding bereits
pr&auml;judiziert sein.}


\PropositionG{2.0121}
{Es erschiene gleichsam als Zufall, wenn dem
Ding, das allein f&uuml;r sich bestehen k&ouml;nnte, nachtr&auml;glich
eine Sachlage passen w&uuml;rde.

Wenn die Dinge in Sachverhalten vorkommen
k&ouml;nnen, so muss dies schon in ihnen liegen.

(Etwas Logisches kann nicht nur-m&ouml;glich sein.
Die Logik handelt von jeder M&ouml;glichkeit und alle
M&ouml;glichkeiten sind ihre Tatsachen.)
% -----File: 032.png---

Wie wir uns r&auml;umliche Gegenst&auml;nde &uuml;berhaupt
nicht ausserhalb des Raumes, zeitliche nicht ausserhalb
der Zeit denken k&ouml;nnen, so k&ouml;nnen wir uns
\Emph{keinen} Gegenstand ausserhalb der M&ouml;glichkeit
seiner Verbindung mit anderen denken.

Wenn ich mir den Gegenstand im Verbande
des Sachverhalts denken kann, so kann ich ihn
nicht ausserhalb der \Emph{M&ouml;glichkeit} dieses Verbandes
denken.}


\PropositionG{2.0122}
{Das Ding ist selbst&auml;ndig, insofern es in allen
\Emph{m&ouml;glichen} Sachlagen vorkommen kann, aber
diese Form der Selbst&auml;ndigkeit ist eine Form des
Zusammenhangs mit dem Sachverhalt, eine Form
der Unselbst&auml;ndigkeit. (Es ist unm&ouml;glich, dass
Worte in zwei verschiedenen Weisen auftreten,
allein und im Satz.)}


\PropositionG{2.0123}
{Wenn ich den Gegenstand kenne, so kenne ich
auch s&auml;mtliche M&ouml;glichkeiten seines Vorkommens
in Sachverhalten.

(Jede solche M&ouml;glichkeit muss in der Natur des
Gegenstandes liegen.)

Es kann nicht nachtr&auml;glich eine neue M&ouml;glichkeit
gefunden werden.}


\PropositionG{2.01231}
{Um einen Gegenstand zu kennen, muss ich zwar
nicht seine externen---aber ich muss alle seine
internen Eigenschaften kennen.}


\PropositionG{2.0124}
{Sind alle Gegenst&auml;nde gegeben, so sind damit
auch alle \Emph{m&ouml;glichen} Sachverhalte gegeben.}


\PropositionG{2.013}
{Jedes Ding ist, gleichsam, in einem Raume
m&ouml;glicher Sachverhalte. Diesen Raum kann ich
mir leer denken, nicht aber das Ding ohne den
Raum.}


\PropositionG{2.0131}
{Der r&auml;umliche Gegenstand muss im unendlichen
Raume liegen. (Der Raumpunkt ist eine Argumentstelle.)

Der Fleck im Gesichtsfeld muss zwar nicht rot
sein, aber eine Farbe muss er haben: er hat sozusagen
den Farbenraum um sich. Der Ton muss
% -----File: 034.png---
\Emph{eine} H&ouml;he haben, der Gegenstand des Tastsinnes
\Emph{eine} H&auml;rte usw.}


\PropositionG{2.014}
{Die Gegenst&auml;nde enthalten die M&ouml;glichkeit aller
Sachlagen.}


\PropositionG{2.0141}
{Die M&ouml;glichkeit seines Vorkommens in Sachverhalten,
ist die Form des Gegenstandes.}


\PropositionG{2.02}
{Der Gegenstand ist einfach.}


\PropositionG{2.0201}
{Jede Aussage &uuml;ber Komplexe l&auml;sst sich in eine
Aussage &uuml;ber deren Bestandteile und in diejenigen
S&auml;tze zerlegen, welche die Komplexe vollst&auml;ndig
beschreiben.}


\PropositionG{2.021}
{Die Gegenst&auml;nde bilden die Substanz der Welt.
Darum k&ouml;nnen sie nicht zusammengesetzt sein.}


\PropositionG{2.0211}
{H&auml;tte die Welt keine Substanz, so w&uuml;rde, ob ein
Satz Sinn hat, davon abh&auml;ngen, ob ein anderer Satz
wahr ist.}


\PropositionG{2.0212}
{Es w&auml;re dann unm&ouml;glich, ein Bild der Welt
(wahr oder falsch) zu entwerfen.}


\PropositionG{2.022}
{Es ist offenbar, dass auch eine von der wirklichen
noch so verschieden gedachte Welt Etwas---eine
Form---mit der wirklichen gemein haben muss.}


\PropositionG{2.023}
{Diese feste Form besteht eben aus den Gegenst&auml;nden.}


\PropositionG{2.0231}
{Die Substanz der Welt \Emph{kann} nur eine Form und
keine materiellen Eigenschaften bestimmen. Denn
diese werden erst durch die S&auml;tze dar\-ge\-stellt---erst
durch die Konfiguration der Gegenst&auml;nde gebildet.}


\PropositionG{2.0232}
{Beil&auml;ufig gesprochen: Die Gegenst&auml;nde sind
farblos.}


\PropositionG{2.0233}
{Zwei Gegenst&auml;nde von der gleichen logischen
Form sind---ab\-ge\-se\-hen von ihren externen Eigenschaften---von
einander nur dadurch unterschieden,
dass sie verschieden sind.}


\PropositionG{2.02331}
{Entweder ein Ding hat Eigenschaften, die kein
anderes hat, dann kann man es ohneweiteres durch
eine Beschreibung aus den anderen herausheben,
und darauf hinweisen; oder aber, es gibt mehrere
Dinge, die ihre s&auml;mtlichen Eigenschaften gemeinsam
% -----File: 036.png---
haben, dann ist es &uuml;berhaupt unm&ouml;glich auf
eines von ihnen zu zeigen.

Denn, ist das Ding durch nichts hervorgehoben,
so kann ich es nicht hervorheben, denn sonst ist
es eben hervorgehoben.}


\PropositionG{2.024}
{Die Substanz ist das, was unabh&auml;ngig von dem
was der Fall ist, besteht.}


\PropositionG{2.025}
{Sie ist Form und Inhalt.}


\PropositionG{2.0251}
{Raum, Zeit und Farbe (F&auml;rbigkeit) sind Formen
der Gegenst&auml;nde.}


\PropositionG{2.026}
{Nur wenn es Gegenst&auml;nde gibt, kann es eine
feste Form der Welt geben.}


\PropositionG{2.027}
{Das Feste, das Bestehende und der Gegenstand
sind Eins.}


\PropositionG{2.0271}
{Der Gegenstand ist das Feste, Bestehende; die
Konfiguration ist das Wechselnde, Unbest&auml;ndige.}


\PropositionG{2.0272}
{Die Konfiguration der Gegenst&auml;nde bildet den
Sachverhalt.}


\PropositionG{2.03}
{Im Sachverhalt h&auml;ngen die Gegenst&auml;nde ineinander,
wie die Glieder einer Kette.}


\PropositionG{2.031}
{Im Sachverhalt verhalten sich die Gegenst&auml;nde
in bestimmter Art und Weise zueinander.}


\PropositionG{2.032}
{Die Art und Weise, wie die Gegenst&auml;nde im
Sachverhalt zusammenh&auml;ngen, ist die Struktur
des Sachverhaltes.}


\PropositionG{2.033}
{Die Form ist die M&ouml;glichkeit der Struktur.}


\PropositionG{2.034}
{Die Struktur der Tatsache besteht aus den
Strukturen der Sachverhalte.}


\PropositionG{2.04}
{Die Gesamtheit der bestehenden Sachverhalte
ist die Welt.}


\PropositionG{2.05}
{Die Gesamtheit der bestehenden Sachverhalte
bestimmt auch, welche Sachverhalte nicht bestehen.}


\PropositionG{2.06}
{Das Bestehen und Nichtbestehen von Sachverhalten
ist die Wirklichkeit.

(Das Bestehen von Sachverhalten nennen wir
auch eine positive, das Nichtbestehen eine negative
Tatsache.)}


\PropositionG{2.061}
{Die Sachverhalte sind von einander unabh&auml;ngig.}
% -----File: 038.png---


\PropositionG{2.062}
{Aus dem Bestehen oder Nichtbestehen eines
Sachverhaltes kann nicht auf das Bestehen oder
Nichtbestehen eines anderen geschlossen werden.}


\PropositionG{2.063}
{Die gesamte Wirklichkeit ist die Welt.}


\PropositionG{2.1}
{Wir machen uns Bilder der Tatsachen.}


\PropositionG{2.11}
{Das Bild stellt die Sachlage im logischen Raume,
das Bestehen und Nichtbestehen von Sachverhalten
vor.}


\PropositionG{2.12}
{Das Bild ist ein Modell der Wirklichkeit.}


\PropositionG{2.13}
{Den Gegenst&auml;nden entsprechen im Bilde die
Elemente des Bildes.}


\PropositionG{2.131}
{Die Elemente des Bildes vertreten im Bild die
Gegenst&auml;nde.}


\PropositionG{2.14}
{Das Bild besteht darin, dass sich seine Elemente
in bestimmter Art und Weise zu einander verhalten.}


\PropositionG{2.141}
{Das Bild ist eine Tatsache.}


\PropositionG{2.15}
{Dass sich die Elemente des Bildes in bestimmter
Art und Weise zu einander verhalten stellt vor,
dass sich die Sachen so zu einander verhalten.

Dieser Zusammenhang der Elemente des Bildes
heisse seine Struktur und ihre M&ouml;glichkeit seine
Form der Abbildung.}


\PropositionG{2.151}
{Die Form der Abbildung ist die M&ouml;glichkeit,
dass sich die Dinge so zu einander verhalten, wie
die Elemente des Bildes.}


\PropositionG{2.1511}
{Das Bild ist \Emph{so} mit der Wirklichkeit verkn&uuml;pft;
es reicht bis zu ihr.}


\PropositionG{2.1512}
{Es ist wie ein \DPtypo{Masstab}{Massstab} an die Wirklichkeit
angelegt.}


\PropositionG{2.15121}
{Nur die &auml;ussersten Punkte der Teilstriche
\Emph{ber&uuml;hren} den zu messenden Gegenstand.}


\PropositionG{2.1513}
{Nach dieser Auffassung geh&ouml;rt also zum Bilde
auch noch die abbildende Beziehung, die es zum
Bild macht.}


\PropositionG{2.1514}
{Die abbildende Beziehung besteht aus den
Zuordnungen der Elemente des Bildes und der
Sachen.}


\PropositionG{2.1515}
{Diese Zuordnungen sind gleichsam die F&uuml;hler
% -----File: 040.png---
der \DPtypo{Bildelmente}{Bildelemente}, mit denen das Bild die Wirklichkeit
ber&uuml;hrt.}


\PropositionG{2.16}
{Die Tatsache muss um Bild zu sein, etwas mit
dem Abgebildeten gemeinsam haben.}


\PropositionG{2.161}
{In Bild und Abgebildetem muss etwas identisch
sein, damit das eine &uuml;berhaupt ein Bild des anderen
sein kann.}


\PropositionG{2.17}
{Was das Bild mit der Wirklichkeit gemein
haben muss, um sie auf seine Art und Weise---richtig
oder falsch---abbilden zu k&ouml;nnen, ist seine
Form der Abbildung.}


\PropositionG{2.171}
{Das Bild kann jede Wirklichkeit abbilden,
deren Form es hat.

Das r&auml;umliche Bild alles R&auml;umliche, das farbige
alles Farbige, etc.}


\PropositionG{2.172}
{Seine Form der Abbildung aber, kann das Bild
nicht abbilden; es weist sie auf.}


\PropositionG{2.173}
{Das Bild stellt sein Objekt von ausserhalb dar
(sein Standpunkt ist seine Form der Darstellung),
darum stellt das Bild sein Objekt richtig oder
falsch dar.}


\PropositionG{2.174}
{Das Bild kann sich aber nicht ausserhalb seiner
Form der Darstellung stellen.}


\PropositionG{2.18}
{Was jedes Bild, welcher Form immer, mit der
Wirklichkeit gemein haben muss, um sie &uuml;berhaupt---richtig
oder falsch---ab\-bil\-den zu k&ouml;nnen,
ist die logische Form, das ist, die Form der
Wirklichkeit.}


\PropositionG{2.181}
{Ist die Form der Abbildung die logische Form,
so heisst das Bild das logische Bild.}


\PropositionG{2.182}
{Jedes Bild ist \Emph{auch} ein logisches. (Dagegen
ist \zumBeispiel\ nicht jedes Bild ein r&auml;umliches.)}


\PropositionG{2.19}
{Das logische Bild kann die Welt abbilden.}


\PropositionG{2.2}
{Das Bild hat mit dem Abgebildeten die logische
Form der Abbildung gemein.}


\PropositionG{2.201}
{Das Bild bildet die Wirklichkeit ab, indem es
eine M&ouml;glichkeit des Bestehens und Nichtbestehens
von Sachverhalten darstellt.}
% -----File: 042.png---


\PropositionG{2.202}
{Das Bild stellt eine m&ouml;gliche Sachlage im
logischen Raume dar.}


\PropositionG{2.203}
{Das Bild enth&auml;lt die M&ouml;glichkeit der Sachlage,
die es darstellt.}


\PropositionG{2.21}
{Das Bild stimmt mit der Wirklichkeit &uuml;berein
oder nicht; es ist richtig oder unrichtig, wahr
oder falsch.}


\PropositionG{2.22}
{{\stretchyspace
Das Bild stellt dar, was es darstellt, unabh&auml;ngig
von seiner Wahr- oder Falschheit, durch die Form
der Abbildung.}}


\PropositionG{2.221}
{Was das Bild darstellt, ist sein Sinn.}


\PropositionG{2.222}
{In der &uuml;bereinstimmung oder Nicht&uuml;bereinstimmung
seines Sinnes mit der Wirklichkeit,
besteht seine Wahrheit oder Falschheit.}


\PropositionG{2.223}
{Um zu erkennen, ob das Bild wahr oder falsch
ist, m&uuml;ssen wir es mit der Wirklichkeit vergleichen.}


\PropositionG{2.224}
{Aus dem Bild allein ist nicht zu erkennen, ob
es wahr oder falsch ist.}


\PropositionG{2.225}
{Ein a priori wahres Bild gibt es nicht.}


\PropositionG{3}
{Das logische Bild der Tatsachen ist der
Gedanke.}


\PropositionG{3.001}
{\glqq{}Ein Sachverhalt ist denkbar\grqq{} heisst: Wir
k&ouml;nnen uns ein Bild von ihm machen.}


\PropositionG{3.01}
{Die Gesamtheit der wahren Gedanken sind
ein Bild der Welt.}


\PropositionG{3.02}
{Der Gedanke enth&auml;lt die M&ouml;glichkeit der
Sachlage die er denkt. Was denkbar ist, ist
auch m&ouml;glich.}


\PropositionG{3.03}
{Wir k&ouml;nnen nichts Unlogisches denken, weil
wir sonst unlogisch denken m&uuml;ssten.}


\PropositionG{3.031}
{Man sagte einmal, dass Gott alles schaffen
k&ouml;nne, nur nichts, was den logischen Gesetzen
zuwider w&auml;re.---Wir k&ouml;nnten n&auml;mlich von einer
\glqq{}unlogischen\grqq{} Welt nicht \Emph{sagen}, wie sie auss&auml;he.}


\PropositionG{3.032}
{Etwas \glqq{}der Logik widersprechendes\grqq{} in der
Sprache darstellen, kann man ebensowenig, wie
in der Geometrie eine den Gesetzen des Raumes
widersprechende Figur durch ihre Koordinaten
% -----File: 044.png---
darstellen; oder die Koordinaten eines Punktes
angeben, welcher nicht existiert.}


\PropositionG{3.0321}
{Wohl k&ouml;nnen wir einen Sachverhalt r&auml;umlich
darstellen, welcher den Gesetzen der Physik,
aber keinen, der den Gesetzen der Geometrie
zuwiderliefe.}


\PropositionG{3.04}
{Ein a priori richtiger Gedanke w&auml;re ein solcher,
dessen M&ouml;glichkeit seine Wahrheit bedingte.}


\PropositionG{3.05}
{Nur so k&ouml;nnten wir a priori wissen, dass ein
Gedanke wahr ist, wenn aus dem Gedanken
selbst (ohne Vergleichsobjekt) seine Wahrheit
zu erkennen w&auml;re.}


\PropositionG{3.1}
{Im Satz dr&uuml;ckt sich der Gedanke sinnlich
wahrnehmbar aus.}


\PropositionG{3.11}
{{\stretchyspace
Wir ben&uuml;tzen das sinnlich wahrnehmbare
Zeichen (Laut- oder Schriftzeichen etc.) des Satzes
als Projektion der m&ouml;glichen Sachlage.}

Die Projektionsmethode ist das Denken des
Satz-Sinnes.}


\PropositionG{3.12}
{Das Zeichen, durch welches wir den Gedanken
ausdr&uuml;cken, nenne ich das Satzzeichen. Und der
Satz ist das Satzzeichen in seiner projektiven
Beziehung zur Welt.}


\PropositionG{3.13}
{Zum Satz geh&ouml;rt alles, was zur Projektion
geh&ouml;rt; aber nicht das Projizierte.

Also die M&ouml;glichkeit des Projizierten, aber nicht
dieses selbst.

Im Satz ist also sein Sinn noch nicht enthalten,
wohl aber die M&ouml;glichkeit ihn auszudr&uuml;cken.

(\glqq{}Der Inhalt des Satzes\grqq{} heisst der Inhalt des
sinnvollen Satzes.)

Im Satz ist die Form seines Sinnes enthalten,
aber nicht dessen Inhalt.}


\PropositionG{3.14}
{Das Satzzeichen besteht darin, dass sich seine
Elemente, die W&ouml;rter, in ihm auf bestimmte Art
und Weise zu einander verhalten.

Das Satzzeichen ist eine Tatsache.}


\PropositionG{3.141}
{Der Satz ist kein W&ouml;rtergemisch.---(Wie
% -----File: 046.png---
das musikalische Thema kein Gemisch von
T&ouml;nen.)

Der Satz ist artikuliert.}


\PropositionG{3.142}
{Nur Tatsachen k&ouml;nnen einen Sinn ausdr&uuml;cken,
eine Klasse von Namen kann es nicht.}


\PropositionG{3.143}
{Dass das Satzzeichen eine Tatsache ist, wird
durch die gew&ouml;hnliche Ausdrucksform der Schrift
oder des Druckes verschleiert.

Denn im gedruckten Satz \zumBeispiel\ sieht das Satzzeichen
nicht wesentlich verschieden aus vom
Wort.

(So war es m&ouml;glich, dass Frege den Satz einen
zusammengesetzten Namen nannte.)}


\PropositionG{3.1431}
{Sehr klar wird das Wesen des Satzzeichens,
wenn wir es uns, statt aus Schriftzeichen, aus
r&auml;umlichen Gegenst&auml;nden (etwa Tischen, St&uuml;hlen,
B&uuml;chern) zusammengesetzt denken.

Die gegenseitige r&auml;umliche Lage dieser Dinge
dr&uuml;ckt dann den Sinn des Satzes aus.}


\PropositionG{3.1432}
{Nicht: \glqq{}Das komplexe Zeichen \glq{}$aRb$\grq{} sagt,
dass $a$ in der Beziehung $R$ zu $b$ steht\grqq{}, sondern:
\Emph{Dass} \glqq{}$a$\grqq{} in einer gewissen Beziehung zu \glqq{}$b$\grqq{}
steht, sagt, \Emph{dass} $aRb$.}


\PropositionG{3.144}
{Sachlagen kann man beschreiben, nicht \Emph{benennen}.

(Namen gleichen Punkten, S&auml;tze Pfeilen, sie
haben Sinn.)}


\PropositionG{3.2}
{Im Satze kann der Gedanke so ausgedr&uuml;ckt sein,
dass den Gegenst&auml;nden des Gedankens Elemente
des Satzzeichens entsprechen.}


\PropositionG{3.201}
{Diese Elemente nenne ich \glqq{}einfache Zeichen\grqq{}
und den Satz \glqq{}vollst&auml;ndig analysiert\grqq{}.}


\PropositionG{3.202}
{Die im Satze angewandten einfachen Zeichen
heissen Namen.}


\PropositionG{3.203}
{Der Name bedeutet den Gegenstand. Der
Gegenstand ist seine Bedeutung. (\glqq{}$A$\grqq{} ist dasselbe
Zeichen wie \glqq{}$A$\grqq{}.)}


\PropositionG{3.21}
{Der Konfiguration der einfachen Zeichen im
% -----File: 048.png---
Satzzeichen entspricht die Konfiguration der Gegenst&auml;nde
in der Sachlage.}


\PropositionG{3.22}
{Der Name vertritt im Satz den Gegenstand.}


\PropositionG{3.221}
{Die Gegenst&auml;nde kann ich nur \Emph{nennen}. Zeichen
vertreten sie. Ich kann nur \Emph{von} ihnen sprechen,
\Emph{sie aussprechen} kann ich nicht. Ein Satz
kann nur sagen, \Emph{wie} ein Ding ist, nicht \Emph{was} es ist.}


\PropositionG{3.23}
{Die Forderung der M&ouml;glichkeit der einfachen
Zeichen ist die Forderung der Bestimmtheit des
Sinnes.}


\PropositionG{3.24}
{Der Satz, welcher vom Komplex handelt, steht
in interner Beziehung zum Satze, der von dessen
Bestandteil handelt.

Der Komplex kann nur durch seine Beschreibung
gegeben sein, und diese wird stimmen oder
nicht stimmen. Der Satz, in welchem von einem
Komplex die Rede ist, wird, wenn dieser nicht
existiert, nicht unsinnig, sondern einfach falsch sein.

Dass ein Satzelement einen Komplex bezeichnet,
kann man aus einer Unbestimmtheit in den S&auml;tzen
sehen, worin es vorkommt. Wir \Emph{wissen}, durch
diesen Satz ist noch nicht alles bestimmt. (Die
Allgemeinheitsbezeichnung \Emph{enth&auml;lt} ja ein Urbild.)

Die Zusammenfassung des Symbols eines Komplexes
in ein einfaches Symbol kann durch eine
\enlargethispage{4pt} % enlarge to make the last word fit
Definition ausgedr&uuml;ckt werden.}


\PropositionG{3.25}
{Es gibt eine und nur eine vollst&auml;ndige Analyse
des Satzes.}


\PropositionG{3.251}
{Der Satz dr&uuml;ckt auf bestimmte, klar angebbare
Weise aus, was er ausdr&uuml;ckt: Der Satz ist artikuliert.}


\PropositionG{3.26}
{Der \DPtypo{name}{Name} ist durch keine Definition weiter zu
zergliedern: er ist ein Urzeichen.}


\PropositionG{3.261}
{Jedes definierte Zeichen bezeichnet \Emph{&uuml;ber} jene
Zeichen, durch welche es definiert wurde; und die
Definitionen weisen den Weg.

Zwei Zeichen, ein Urzeichen, und ein durch
Urzeichen definiertes, k&ouml;nnen nicht auf dieselbe
% -----File: 050.png---
Art und Weise bezeichnen. Namen \Emph{kann} man
nicht durch Definitionen auseinanderlegen. (Kein
Zeichen, welches allein, selbst&auml;ndig eine Bedeutung
hat.)}


\PropositionG{3.262}
{Was in den Zeichen nicht zum Ausdruck kommt,
das zeigt ihre Anwendung. Was die Zeichen
verschlucken, das spricht ihre Anwendung aus.}


\PropositionG{3.263}
{Die Bedeutungen von Urzeichen k&ouml;nnen durch
Erl&auml;uterungen erkl&auml;rt werden. Erl&auml;uterungen
sind S&auml;tze, welche die Urzeichen enthalten. Sie
k&ouml;nnen also nur verstanden werden, wenn die
Bedeutungen dieser Zeichen bereits bekannt sind.}


\PropositionG{3.3}
{Nur der Satz hat Sinn; nur im Zusammenhange
des Satzes hat ein Name Bedeutung.}


\PropositionG{3.31}
{Jeden Teil des Satzes, der seinen Sinn charakterisiert,
nenne ich einen Ausdruck (ein Symbol).

(Der Satz selbst ist ein Ausdruck.)

Ausdruck ist alles, f&uuml;r den Sinn des Satzes
wesentliche, was S&auml;tze miteinander gemein haben
k&ouml;nnen.

Der Ausdruck kennzeichnet eine Form und
einen Inhalt.}


\PropositionG{3.311}
{Der Ausdruck setzt die Formen aller S&auml;tze
voraus, in welchen er vorkommen kann. Er ist
das gemeinsame charakteristische Merkmal einer
Klasse von S&auml;tzen.}


\PropositionG{3.312}
{Er wird also dargestellt durch die allgemeine
Form der S&auml;tze, die er charakterisiert.

Und zwar wird in dieser Form der Ausdruck
\Emph{konstant} und alles &uuml;brige \Emph{variabel} sein.}


\PropositionG{3.313}
{Der Ausdruck wird also durch eine Variable
\enlargethispage{6pt} % enlarge to make the last line fit
dargestellt, deren Werte die S&auml;tze sind, die den
Ausdruck enthalten.

(Im Grenzfall wird die Variable zur Konstanten,
der Ausdruck zum Satz.)

Ich nenne eine solche Variable \glqq{}Satzvariable\grqq{}.}


\PropositionG{3.314}
{Der Ausdruck hat nur im Satz Bedeutung.
Jede Variable l&auml;sst sich als Satzvariable auffassen.
% -----File: 052.png---

(Auch der variable Name.)}


\PropositionG{3.315}
{Verwandeln wir einen Bestandteil eines Satzes
in eine Variable, so gibt es eine Klasse von S&auml;tzen,
welche s&auml;mtlich Werte des so entstandenen variablen
Satzes sind. Diese Klasse h&auml;ngt im allgemeinen
noch davon ab, was wir, nach willk&uuml;rlicher
&uuml;bereinkunft, mit Teilen jenes Satzes meinen.
Verwandeln wir aber alle jene Zeichen, deren
Bedeutung willk&uuml;rlich bestimmt wurde, in Variable,
so gibt es nun noch immer eine solche Klasse.
Diese aber ist nun von keiner &uuml;bereinkunft
abh&auml;ngig, sondern nur noch von der Natur des
Satzes. Sie entspricht einer logischen Form---einem
logischen Urbild.}


\PropositionG{3.316}
{Welche Werte die Satzvariable annehmen darf,
wird festgesetzt.

Die Festsetzung der Werte \Emph{ist} die Variable.}


\PropositionG{3.317}
{Die Festsetzung der Werte der Satzvariablen
ist die \Emph{Angabe der S&auml;tze}, deren gemeinsames
Merkmal die Variable ist.

Die Festsetzung ist eine Beschreibung dieser
S&auml;tze.

Die Festsetzung wird also nur von Symbolen,
nicht von deren Bedeutung handeln.

Und \Emph{nur} dies ist der Festsetzung wesentlich,
\Emph{dass sie nur eine Beschreibung von
Symbolen ist und nichts &uuml;ber das Bezeichnete
aussagt}.

Wie die Beschreibung der S&auml;tze geschieht, ist
unwesentlich.}


\PropositionG{3.318}
{Den Satz fasse ich---wie Frege und Russell---als
Funktion der in ihm enthaltenen Ausdr&uuml;cke auf.}


\PropositionG{3.32}
{Das Zeichen ist das sinnlich Wahrnehmbare am
Symbol.}


\PropositionG{3.321}
{Zwei verschiedene Symbole k&ouml;nnen also das
Zeichen (Schriftzeichen oder Lautzeichen etc.)
miteinander gemein haben---sie bezeichnen dann
auf verschiedene Art und Weise.}
% -----File: 054.png---


\PropositionG{3.322}
{Es kann nie das gemeinsame Merkmal zweier
Gegenst&auml;nde anzeigen, dass wir sie mit demselben
Zeichen, aber durch zwei verschiedene \Emph{Bezeichnungsweisen}
bezeichnen. Denn das Zeichen
ist ja willk&uuml;rlich. Man k&ouml;nnte also auch zwei verschiedene
Zeichen w&auml;hlen, und wo bliebe dann das
Gemeinsame in der Bezeichnung.}


\PropositionG{3.323}
{In der Umgangssprache kommt es ungemein
h&auml;ufig vor, dass dasselbe Wort auf verschiedene
Art und Weise bezeichnet---also verschiedenen
Symbolen an\-ge\-h&ouml;rt---, oder, dass zwei W&ouml;rter,
die auf verschiedene Art und Weise bezeichnen,
&auml;usserlich in der gleichen Weise im Satze angewandt
werden.

So erscheint das Wort \glqq{}ist\grqq{} als Kopula, als
Gleichheitszeichen und als Ausdruck der Existenz;
\glqq{}existieren\grqq{} als intransitives Zeitwort wie \glqq{}gehen\grqq{};
\glqq{}identisch\grqq{} als Eigenschaftswort; wir reden von
\Emph{Etwas}, aber auch davon, dass \Emph{etwas} geschieht.

(Im Satze \glqq{}Gr&uuml;n ist gr&uuml;n\grqq{}---wo das erste Wort
ein Personenname, das letzte ein Eigenschaftswort
ist---haben diese Worte nicht einfach verschiedene
Bedeutung, sondern es sind \Emph{verschiedene
Symbole}.)}


\PropositionG{3.324}
{So entstehen leicht die fundamentalsten Verwechslungen
(deren die ganze Philosophie voll
ist).}


\PropositionG{3.325}
{Um diesen Irrt&uuml;mern zu entgehen, m&uuml;ssen
wir eine Zeichensprache verwenden, welche sie
ausschliesst, indem sie nicht das gleiche Zeichen
in verschiedenen Symbolen, und Zeichen, welche
auf verschiedene Art bezeichnen, nicht &auml;usserlich
auf die gleiche Art verwendet. Eine Zeichensprache
also, die der \Emph{logischen} Grammatik---der logischen
Syntax---gehorcht.

(Die Begriffsschrift Frege's und Russell's ist
eine solche Sprache, die allerdings noch nicht alle
Fehler ausschliesst.)}
% -----File: 056.png---


\PropositionG{3.326}
{Um das Symbol am Zeichen zu erkennen, muss
man auf den sinnvollen Gebrauch achten.}


\PropositionG{3.327}
{Das Zeichen bestimmt erst mit seiner logisch-syntaktischen
Verwendung zusammen eine logische
Form.}


\PropositionG{3.328}
{Wird ein Zeichen \Emph{nicht gebraucht}, so ist
es bedeutungslos. Das ist der Sinn der Devise
Occams.

(Wenn sich alles so verh&auml;lt als h&auml;tte ein Zeichen
Bedeutung, dann hat es auch Bedeutung.)}


\PropositionG{3.33}
{In der logischen Syntax darf nie die Bedeutung
eines Zeichens eine Rolle spielen; sie muss sich
aufstellen lassen, ohne dass dabei von der \Emph{Bedeutung}
eines Zeichens die Rede w&auml;re, sie darf \Emph{nur}
die Beschreibung der Ausdr&uuml;cke voraussetzen.}


\PropositionG{3.331}
{Von dieser Bemerkung sehen wir in Russell's
\glqq{}Theory of types\grqq{} hin&uuml;ber: Der Irrtum Russell's
zeigt sich darin, dass er bei der Aufstellung der
Zeichenregeln von der Bedeutung der Zeichen
reden musste.}


\PropositionG{3.332}
{Kein Satz kann etwas &uuml;ber sich selbst aussagen,
weil das Satzzeichen nicht in sich selbst enthalten
sein kann, (das ist die ganze \glqq{}Theory of types\grqq{}).}


\PropositionG{3.333}
{Eine Funktion kann darum nicht ihr eigenes
Argument sein, weil das Funktionszeichen bereits
das Urbild seines Arguments enth&auml;lt und es sich
nicht selbst enthalten kann.

Nehmen wir n&auml;mlich an, die Funktion $F (fx)$
k&ouml;nnte ihr eigenes Argument sein; dann g&auml;be es
also einen Satz: \glqq{}$F(F(fx))$\grqq{} und in diesem m&uuml;ssen
die &auml;ussere Funktion $F$ und die innere Funktion $F$
verschiedene Bedeutungen haben, denn die innere
hat die Form $\phi(fx)$, die &auml;ussere, die Form $\psi(\phi(fx))$.
Gemeinsam ist den beiden Funktionen nur der
Buchstabe \glqq{}$F$\grqq{}, der aber allein nichts bezeichnet.

Dies wird sofort klar, wenn wir statt \glqq{}$F(F(u))$\grqq{}
schreiben \glqq{}$(\exists\phi) : F(\phi u) \DotOp \phi u = Fu$\grqq{}.

Hiermit erledigt sich Russell's Paradox.}
% -----File: 058.png---


\PropositionG{3.334}
{Die Regeln der logischen Syntax m&uuml;ssen sich
von selbst verstehen, wenn man nur weiss, wie
ein jedes Zeichen bezeichnet.}


\PropositionG{3.34}
{Der Satz besitzt wesentliche und zuf&auml;llige Z&uuml;ge.

Zuf&auml;llig sind die Z&uuml;ge, die von der besonderen
Art der Hervorbringung des Satzzeichens herr&uuml;hren.
Wesentlich diejenigen, welche allein den Satz bef&auml;higen,
seinen Sinn auszudr&uuml;cken.}


\PropositionG{3.341}
{Das Wesentliche am Satz ist also das, was allen
S&auml;tzen, welche den gleichen Sinn ausdr&uuml;cken
k&ouml;nnen, gemeinsam ist.

Und ebenso ist allgemein das Wesentliche am
Symbol das, was alle Symbole, die denselben
Zweck erf&uuml;llen k&ouml;nnen, gemeinsam haben.}


\PropositionG{3.3411}
{Man k&ouml;nnte also sagen: Der eigentliche Name
ist das, was alle Symbole, die den Gegenstand
bezeichnen, gemeinsam haben. Es w&uuml;rde sich so
successive ergeben, dass keinerlei Zusammensetzung
f&uuml;r den Namen wesentlich ist.}


\PropositionG{3.342}
{An unseren Notationen ist zwar etwas willk&uuml;rlich,
aber \Emph{das} ist nicht willk&uuml;rlich: Dass, \Emph{wenn} wir
etwas willk&uuml;rlich bestimmt haben, dann etwas
anderes der Fall sein muss. (Dies h&auml;ngt von dem
\Emph{Wesen} der Notation ab.)}


\PropositionG{3.3421}
{{\stretchyspace
Eine besondere Bezeichnungsweise mag unwichtig
sein, aber wichtig ist es immer, dass diese
eine \Emph{m&ouml;gliche} Bezeichnungsweise ist. Und so
verh&auml;lt es sich in der Philosophie &uuml;berhaupt: Das
Einzelne erweist sich immer wieder als unwichtig,
aber die M&ouml;glichkeit jedes Einzelnen gibt uns
einen Aufschluss &uuml;ber das Wesen der Welt.}}


\PropositionG{3.343}
{Definitionen sind Regeln der &uuml;bersetzung von
einer Sprache in eine andere. Jede richtige Zeichensprache
muss sich in jede andere nach solchen
Regeln &uuml;bersetzen lassen: \Emph{Dies} ist, was sie alle
gemeinsam haben.}


\PropositionG{3.344}
{Das, was am Symbol bezeichnet, ist das Gemeinsame
aller jener Symbole, durch die das erste den
% -----File: 060.png---
Regeln der logischen Syntax zufolge ersetzt werden
kann.}


\PropositionG{3.3441}
{Man kann \zumBeispiel\ das Gemeinsame aller Notationen
f&uuml;r die Wahrheitsfunktionen so ausdr&uuml;cken: Es ist
ihnen gemeinsam, dass sich alle---\zumBeispiel---durch die
Notation von \glqq{}$\Not{p}$\grqq{} (\glqq{}nicht $p$\grqq{}) und \glqq{}$p \lor q$\grqq{} (\glqq{}$p$ oder $q$\grqq{})
\Emph{ersetzen lassen}.

{\stretchyspace
(Hiermit ist die Art und Weise gekennzeichnet,
wie eine spezielle m&ouml;gliche Notation uns allgemeine
Aufschl&uuml;sse geben kann.)}}


\PropositionG{3.3442}
{Das Zeichen des Komplexes l&ouml;st sich auch bei
der Analyse nicht willk&uuml;rlich auf, so dass etwa seine
Aufl&ouml;sung in jedem Satzgef&uuml;ge eine andere w&auml;re.}


\PropositionG{3.4}
{Der Satz bestimmt einen Ort im logischen Raum.
Die Existenz dieses logischen Ortes ist durch die
Existenz der Bestandteile allein verb&uuml;rgt, durch die
Existenz des sinnvollen Satzes.}


\PropositionG{3.41}
{Das Satzzeichen und die logischen Koordinaten:
\enlargethispage{1pt} % enlarge to make the last line fit
Das ist der logische Ort.}


\PropositionG{3.411}
{Der geometrische und der logische Ort stimmen
darin &uuml;berein, dass beide die M&ouml;glichkeit einer
Existenz sind.}


\PropositionG{3.42}
{Obwohl der Satz nur einen Ort des logischen
Raumes bestimmen darf, so muss doch durch
ihn schon der ganze logische Raum gegeben
sein.

(Sonst w&uuml;rden durch die Verneinung, die logische
Summe, das logische Produkt, etc.\ immer neue
Elemente---in Ko\-or\-di\-na\-ti\-on---eingef&uuml;hrt.)

(Das logische Ger&uuml;st um das Bild herum bestimmt
den logischen Raum. Der Satz durchgreift den
ganzen logischen Raum.)}


\PropositionG{3.5}
{Das angewandte, gedachte, Satzzeichen ist der
Gedanke.}


\PropositionG{4}
{Der Gedanke ist der sinnvolle Satz.}


\PropositionG{4.001}
{Die Gesamtheit der S&auml;tze ist die Sprache.}


\PropositionG{4.002}
{Der Mensch besitzt die F&auml;higkeit Sprachen zu
bauen, womit sich jeder Sinn ausdr&uuml;cken l&auml;sst,
% -----File: 062.png---
ohne eine Ahnung davon zu haben, wie und was
jedes Wort bedeutet.---Wie man auch spricht, ohne
zu wissen, wie die einzelnen Laute hervorgebracht
werden.

Die Umgangssprache ist ein Teil des menschlichen
Organismus und nicht weniger kompliziert als
dieser.

Es ist menschenunm&ouml;glich, die Sprachlogik aus
ihr unmittelbar zu entnehmen.

Die Sprache verkleidet den Gedanken. Und
zwar so, dass man nach der &auml;usseren Form des
Kleides, nicht auf die Form des bekleideten Gedankens
schliessen kann; weil die &auml;ussere Form des
Kleides nach ganz anderen Zwecken gebildet ist, als
danach, die Form des K&ouml;rpers erkennen zu lassen.

{\stretchyspace
Die stillschweigenden Abmachungen zum Verst&auml;ndnis
der Umgangssprache sind enorm kompliziert.}}


\PropositionG{4.003}
{Die meisten S&auml;tze und Fragen, welche &uuml;ber
philosophische Dinge geschrieben worden sind, sind
nicht falsch, sondern unsinnig. Wir k&ouml;nnen daher
Fragen dieser Art &uuml;berhaupt nicht beantworten,
sondern nur ihre Unsinnigkeit feststellen. Die
meisten Fragen und S&auml;tze der Philosophen beruhen
darauf, \DPtypo{das}{dass} wir unsere Sprachlogik nicht verstehen.

(Sie sind von der Art der Frage, ob das Gute
\enlargethispage{1pt} % enlarge to make the last line fit
mehr oder weniger identisch sei als das Sch&ouml;ne.)

Und es ist nicht verwunderlich, dass die tiefsten
Probleme eigentlich \Emph{keine} Probleme sind.}


\PropositionG{4.0031}
{Alle Philosophie ist \glqq{}Sprachkritik\grqq{}. (Allerdings
nicht im Sinne Mauthners.) Russell's Verdienst ist
es, gezeigt zu haben, dass die scheinbare logische
Form des Satzes nicht seine wirkliche sein muss.}


\PropositionG{4.01}
{Der Satz ist ein Bild der Wirklichkeit.

Der Satz ist ein Modell der Wirklichkeit, so wie
wir sie uns denken.}


\PropositionG{4.011}
{Auf den ersten Blick scheint der Satz---wie er
etwa auf dem Papier gedruckt steht---kein Bild der
% -----File: 064.png---
Wirklichkeit zu sein, von der er handelt. Aber
auch die Notenschrift scheint auf den ers\-ten Blick
kein Bild der Musik zu sein, und unsere Lautzeichen-\mbox{(Buchstaben-)}\AllowBreak{}Schrift
kein Bild unserer Lautsprache.

Und doch erweisen sich diese Zeichensprachen
auch im gew&ouml;hnlichen Sinne als Bilder dessen, was
sie darstellen.}


\PropositionG{4.012}
{Offenbar ist, dass wir einen Satz von der Form
\glqq{}$aRb$\grqq{} als Bild empfinden. Hier ist das Zeichen
offenbar ein Gleichnis des Bezeichneten.}


\PropositionG{4.013}
{Und wenn wir in das Wesentliche dieser Bildhaftigkeit
eindringen, so sehen wir, dass dieselbe
durch \Emph{scheinbare Unregelm&auml;ssigkeiten}
(wie die Verwendung der $\sharp$ und $\flat$ in der Notenschrift)
\Emph{nicht} gest&ouml;rt wird.

Denn auch diese Unregelm&auml;ssigkeiten bilden
das ab, was sie ausdr&uuml;cken sollen; nur auf eine
andere Art und Weise.}


\PropositionG{4.014}
{Die Grammophonplatte, der musikalische Gedanke,
die Notenschrift, die Schallwellen, stehen
alle in jener abbildenden internen Beziehung zu
einander, die zwischen Sprache und Welt besteht.

Ihnen allen ist der logische Bau gemeinsam.

(Wie im M&auml;rchen die zwei J&uuml;nglinge, ihre zwei
Pferde und ihre Lilien. Sie sind alle in gewissem
Sinne Eins.)}


\PropositionG{4.0141}
{Dass es eine allgemeine Regel gibt, durch die
der Musiker aus der Partitur die Symphonie
entnehmen kann, durch welche man aus der Linie
auf der Grammophonplatte die Symphonie und
nach der ersten Regel wieder die Partitur ableiten
kann, darin besteht eben die innere &auml;hnlichkeit
dieser scheinbar so ganz verschiedenen Gebilde.
Und jene Regel ist das Gesetz der Projektion,
welches die Symphonie in die Notensprache projiziert.
Sie ist die Regel der &uuml;bersetzung der
Notensprache in die Sprache der Grammophonplatte.}


\PropositionG{4.015}
{Die M&ouml;glichkeit aller Gleichnisse, der ganzen
% -----File: 066.png---
Bildhaftigkeit unserer Ausdrucksweise, ruht in der
Logik der Abbildung.}


\PropositionG{4.016}
{Um das Wesen des Satzes zu verstehen, denken
wir an die Hieroglyphenschrift, welche die Tatsachen
die sie beschreibt abbildet.

Und aus ihr wurde die Buchstabenschrift, ohne
das Wesentliche der Abbildung zu verlieren.}


\PropositionG{4.02}
{Dies sehen wir daraus, dass wir den Sinn des
Satzzeichens verstehen, ohne dass er uns erkl&auml;rt
wurde.}


\PropositionG{4.021}
{Der Satz ist ein Bild der Wirklichkeit: Denn
ich kenne die von ihm dargestellte Sachlage, wenn
ich den Satz verstehe. Und den Satz verstehe ich,
ohne dass mir sein Sinn erkl&auml;rt wurde.}


\PropositionG{4.022}
{Der Satz \Emph{zeigt} seinen Sinn.

Der Satz \Emph{zeigt}, wie es sich verh&auml;lt, \Emph{wenn} er
wahr ist. Und er \Emph{sagt}, \Emph{dass} es sich so verh&auml;lt.}


\PropositionG{4.023}
{\DPtypo{Der}{Die} Wirklichkeit muss durch den Satz auf ja
oder nein fixiert sein.

Dazu muss sie durch ihn vollst&auml;ndig beschrieben
werden.

Der Satz ist die Beschreibung eines Sachverhaltes.

Wie die Beschreibung einen Gegenstand nach
seinen externen Eigenschaften, so beschreibt der
Satz die Wirklichkeit nach ihren internen Eigenschaften.

Der Satz konstruiert eine Welt mit Hilfe eines
logischen Ger&uuml;stes und darum kann man am Satz
auch sehen, wie sich alles Logische verh&auml;lt, \Emph{wenn}
er wahr ist. Man kann aus einem falschen Satz
\Emph{Schl&uuml;sse ziehen}.}


\PropositionG{4.024}
{Einen Satz verstehen, heisst, wissen was der
Fall ist, wenn er wahr ist.

(Man kann ihn also verstehen, ohne zu wissen,
ob er wahr ist.)

Man versteht ihn, wenn man seine Bestandteile
versteht.}


\PropositionG{4.025}
{Die &uuml;bersetzung einer Sprache in eine andere
% -----File: 068.png---
geht nicht so vor sich, dass man jeden \Emph{Satz} der
einen in einen \Emph{Satz} der anderen &uuml;bersetzt, sondern
nur die Satzbestandteile werden &uuml;bersetzt.

(Und das W&ouml;rterbuch &uuml;bersetzt nicht nur
Substantiva, sondern auch \mbox{Zeit-,} Eigenschafts- und
Bindew&ouml;rter etc.; und es behandelt sie alle gleich.)}


\PropositionG{4.026}
{Die Bedeutungen der einfachen Zeichen (der
W&ouml;rter) m&uuml;ssen uns erkl&auml;rt werden, dass wir sie
verstehen.

Mit den S&auml;tzen aber verst&auml;ndigen wir uns.}


\PropositionG{4.027}
{Es liegt im Wesen des Satzes, dass er uns einen
\Emph{neuen} Sinn mitteilen kann.}


\PropositionG{4.03}
{Ein Satz muss mit alten Ausdr&uuml;cken einen
neuen Sinn mitteilen.

Der Satz teilt uns eine Sachlage mit, also
muss er \Emph{wesentlich} mit der Sachlage zusammenh&auml;ngen.

Und der Zusammenhang ist eben, dass er ihr
logisches Bild ist.

Der Satz sagt nur insoweit etwas aus, als er ein
Bild ist.}


\PropositionG{4.031}
{Im Satz wird gleichsam eine Sachlage probeweise
zusammengestellt.

Man kann geradezu sagen: statt, dieser Satz
hat diesen und diesen Sinn; dieser Satz stellt diese
und diese Sachlage dar.}


\PropositionG{4.0311}
{Ein Name steht f&uuml;r ein Ding, ein anderer f&uuml;r
ein anderes Ding und untereinander sind sie
verbunden, so stellt das Ganze---wie ein lebendes
Bild---den Sachverhalt vor.}


\PropositionG{4.0312}
{Die M&ouml;glichkeit des Satzes beruht auf dem
Prinzip der Vertretung von Gegenst&auml;nden durch
Zeichen.

{\stretchyspace
Mein Grundgedanke ist, dass die \glqq{}logischen
Konstanten\grqq{} nicht vertreten. Dass sich die \Emph{Logik}
der Tatsachen nicht vertreten l&auml;sst.}}


\PropositionG{4.032}
{Nur insoweit ist der Satz ein Bild einer Sachlage,
als er logisch gegliedert ist.
% -----File: 070.png---

(Auch der Satz \glqq{}ambulo\grqq{} ist zusammengesetzt,
denn sein Stamm ergibt mit einer anderen Endung
und seine Endung mit einem anderen Stamm, einen
anderen Sinn.)}


\PropositionG{4.04}
{Am Satz muss gerade soviel zu unterscheiden
sein, als an der Sachlage die er darstellt.

{\stretchyspace
Die beiden m&uuml;ssen die gleiche logische (mathematische)
Mannigfaltigkeit besitzen. (Vergleiche
Hertz's Mechanik, &uuml;ber Dynamische Modelle.)}}


\PropositionG{4.041}
{Diese mathematische Mannigfaltigkeit kann
man nat&uuml;rlich nicht selbst wieder abbilden. Aus
ihr kann man beim Abbilden nicht heraus.}


\PropositionG{4.0411}
{Wollten wir \zumBeispiel\ das, was wir durch \glqq{}$(x) fx$\grqq{}
ausdr&uuml;cken, durch Vorsetzen eines Indexes vor
\glqq{}$fx$\grqq{} ausdr&uuml;cken---etwa so: \glqq{}Alg. $fx$\grqq{}, es w&uuml;rde
nicht gen&uuml;gen---wir w&uuml;ssten nicht, was verallgemeinert
wurde. Wollten wir es durch einen
Index \glqq{}$a$\grqq{} anzeigen---etwa so: \glqq{}$f(x_{a}$)\grqq{}---es w&uuml;rde
auch nicht gen&uuml;gen---wir w&uuml;ssten nicht den
Bereich der Allgemeinheitsbezeichnung.

Wollten wir es durch Einf&uuml;hrung einer Marke
in die Argumentstellen ver\-su\-chen---etwa so:
\glqq{}$(A, A) \DotOp F (A, A)$\grqq{}---es w&uuml;rde nicht ge\-n&uuml;gen---wir
k&ouml;nnten die Identit&auml;t der Variablen nicht feststellen.
U.s.w.

Alle diese Bezeichnungsweisen gen&uuml;gen nicht,
weil sie nicht die notwendige mathematische
Mannigfaltigkeit haben.}


\PropositionG{4.0412}
{{\stretchyspace
Aus demselben Grunde gen&uuml;gt die idealistische
Erkl&auml;rung des Sehens der r&auml;umlichen Beziehungen
durch die \glqq{}Raumbrille\grqq{} nicht, weil sie nicht die
Mannigfaltigkeit dieser Beziehungen erkl&auml;ren kann.}}


\PropositionG{4.05}
{Die Wirklichkeit wird mit dem Satz verglichen.}


\PropositionG{4.06}
{Nur dadurch kann der Satz wahr oder falsch
sein, indem er ein Bild der Wirklichkeit ist.}


\PropositionG{4.061}
{Beachtet man nicht, dass der Satz einen von
den Tatsachen unabh&auml;ngigen Sinn hat, so kann
man leicht glauben, dass wahr und falsch gleichberechtigte
% -----File: 072.png---
Beziehungen von Zeichen und Bezeichnetem
sind.

Man k&ouml;nnte dann \zumBeispiel\ sagen, dass \glqq{}$p$\grqq{} auf die
wahre Art bezeichnet, was \glqq{}$\Not{p}$\grqq{} auf die falsche
Art, etc.}


\PropositionG{4.062}
{Kann man sich nicht mit falschen S&auml;tzen, wie
bisher mit wahren, verst&auml;ndigen? Solange man
nur weiss, dass sie falsch gemeint sind. Nein!
Denn, wahr ist ein Satz, wenn es sich so verh&auml;lt,
wie wir es durch ihn sagen; und wenn wir mit
\glqq{}$p$\grqq{} $\Not{p}$ meinen, und es sich so verh&auml;lt wie wir es
meinen, so ist \glqq{}$p$\grqq{} in der neuen Auffassung wahr
und nicht falsch.}


\PropositionG{4.0621}
{Dass aber die Zeichen \glqq{}$p$\grqq{} und \glqq{}$\Not{p}$\grqq{} das gleiche
sagen \Emph{k&ouml;nnen}, ist wichtig. Denn es zeigt, dass
dem Zeichen \glqq{}$\Not{}$\grqq{} in der Wirklichkeit nichts
entspricht.

Dass in einem Satz die Verneinung vorkommt,
ist noch kein Merkmal seines Sinnes ($\Not{\Not{p}} = p$).

Die S&auml;tze \glqq{}$p$\grqq{} und \glqq{}$\Not{p}$\grqq{} haben entgegengesetzten
Sinn, aber es entspricht ihnen eine und
dieselbe Wirklichkeit.}


\PropositionG{4.063}
{Ein Bild zur Erkl&auml;rung des Wahrheitsbegriffes:
Schwarzer Fleck auf weissem Papier; die Form
des Fleckes kann man beschreiben, indem man
f&uuml;r jeden Punkt der Fl&auml;che angibt, ob er weiss
oder schwarz ist. Der Tatsache, dass ein Punkt
schwarz ist, entspricht eine positive---der, dass
ein Punkt weiss (nicht schwarz) ist, eine negative
Tatsache. Bezeichne ich einen Punkt der Fl&auml;che
(einen Frege'schen Wahrheitswert), so entspricht
dies der Annahme, die zur Beurteilung aufgestellt
wird, etc.\ etc.

Um aber sagen zu k&ouml;nnen, ein Punkt sei
schwarz oder weiss, muss ich vorerst wissen,
wann man einen Punkt schwarz und wann
man ihn weiss nennt; um sagen zu k&ouml;nnen:
\glqq{}$p$\grqq{} ist wahr (oder falsch), muss ich bestimmt
% -----File: 074.png---
haben, unter welchen Umst&auml;nden ich \glqq{}$p$\grqq{} wahr
nenne, und damit bestimme ich den Sinn des
Satzes.

Der Punkt an dem das Gleichnis hinkt ist
nun der: Wir k&ouml;nnen auf einen Punkt des Papiers
zeigen, auch ohne zu wissen, was weiss und
schwarz ist; einem Satz ohne Sinn aber entspricht
gar nichts, denn er bezeichnet kein Ding (Wahrheitswert)
dessen Eigenschaften etwa \glqq{}falsch\grqq{} oder
\glqq{}wahr\grqq{} hiessen; das Verbum eines Satzes ist nicht
\glqq{}ist wahr\grqq{} oder \glqq{}ist falsch\grqq{}---wie Frege glaubte---,
sondern das, was \glqq{}wahr ist\grqq{} muss das Verbum
schon enthalten.}


\PropositionG{4.064}
{Jeder Satz muss \Emph{schon} einen Sinn haben;
die Bejahung kann ihn ihm nicht geben, denn
sie bejaht ja gerade den Sinn. Und dasselbe gilt
von der Verneinung, etc.}


\PropositionG{4.0641}
{Man k&ouml;nnte sagen: Die Verneinung bezieht
sich schon auf den logischen Ort, den der verneinte
Satz bestimmt.

Der verneinende Satz bestimmt einen \Emph{anderen}
logischen Ort als der verneinte.

Der verneinende Satz bestimmt einen logischen
Ort mit Hilfe des logischen Ortes des verneinten
Satzes, indem er jenen ausserhalb diesem liegend
beschreibt.

Dass man den verneinten Satz wieder verneinen
kann, zeigt schon, dass das, was verneint wird,
schon ein Satz und nicht erst die Vorbereitung
zu einem Satze ist.}


\PropositionG{4.1}
{Der Satz stellt das Bestehen und Nichtbestehen
der Sachverhalte dar.}


\PropositionG{4.11}
{Die Gesamtheit der wahren S&auml;tze ist die
gesamte Naturwissenschaft (oder die Gesamtheit
der Naturwissenschaften).}


\PropositionG{4.111}
{Die Philosophie ist keine der Naturwissenschaften.

(Das Wort \glqq{}Philosophie\grqq{} muss etwas bedeuten,
% -----File: 076.png---
was &uuml;ber oder unter, aber nicht neben den Naturwissenschaften
steht.)}


\PropositionG{4.112}
{Der Zweck der Philosophie ist die logische
Kl&auml;rung der Gedanken.

Die Philosophie ist keine Lehre, sondern eine
T&auml;tigkeit.

Ein philosophisches Werk besteht wesentlich
aus Erl&auml;uterungen.

Das Resultat der Philosophie sind nicht \glqq{}philosophische
S&auml;tze\grqq{}, sondern das Klarwerden von
S&auml;tzen.

Die Philosophie soll die Gedanken, die sonst,
gleichsam, tr&uuml;be und verschwommen sind, klar
machen und scharf abgrenzen.}


\PropositionG{4.1121}
{Die Psychologie ist der Philosophie nicht verwandter
als irgend eine andere Naturwissenschaft.

Erkenntnistheorie ist die Philosophie der
Psychologie.

Entspricht nicht mein Studium der Zeichensprache
dem Studium der Denkprozesse, welches
die Philosophen f&uuml;r die Philosophie der Logik f&uuml;r
so wesentlich hielten? Nur verwickelten sie sich
meistens in unwesentliche psychologische Untersuchungen
und eine analoge Gefahr gibt es auch
bei meiner Methode.}


\PropositionG{4.1122}
{Die Darwinsche Theorie hat mit der Philosophie
nicht mehr zu schaffen, als irgend eine andere
Hypothese der Naturwissenschaft.}


\PropositionG{4.113}
{Die Philosophie begrenzt das bestreitbare
Gebiet der Naturwissenschaft.}


\PropositionG{4.114}
{Sie soll das Denkbare abgrenzen und damit das
Undenkbare.

Sie soll das Undenkbare von innen durch das
Denkbare begrenzen.}


\PropositionG{4.115}
{Sie wird das Unsagbare bedeuten, indem sie
das Sagbare klar darstellt.}


\PropositionG{4.116}
{Alles was &uuml;berhaupt gedacht werden kann,
% -----File: 078.png---
kann klar gedacht werden. Alles was sich aussprechen
l&auml;sst, l&auml;sst sich klar aussprechen.}


\PropositionG{4.12}
{Der Satz kann die gesamte Wirklichkeit darstellen,
aber er kann nicht das darstellen, was er
mit der Wirklichkeit gemein haben muss, um sie
darstellen zu k&ouml;nnen---die logische Form.

Um die logische Form darstellen zu k&ouml;nnen,
m&uuml;ssten wir uns mit dem Satze ausserhalb der
Logik aufstellen k&ouml;nnen, das heisst ausserhalb der
Welt.}


\PropositionG{4.121}
{Der Satz kann die logische Form nicht darstellen,
sie spiegelt sich in ihm.

Was sich in der Sprache spiegelt, kann sie
nicht darstellen.

Was \Emph{sich} in der Sprache ausdr&uuml;ckt, k&ouml;nnen
\Emph{wir} nicht durch sie ausdr&uuml;cken.

Der Satz \Emph{zeigt} die logische Form der Wirklichkeit.

Er weist sie auf.}


\PropositionG{4.1211}
{So zeigt ein Satz \glqq{}$fa$\grqq{}, dass in seinem Sinn der
Gegenstand $a$ vorkommt, zwei S&auml;tze \glqq{}$fa$\grqq{} und \glqq{}$ga$\grqq{},
dass in ihnen beiden von demselben Gegenstand
die Rede ist.

Wenn zwei S&auml;tze einander widersprechen, so
zeigt dies ihre Struktur; ebenso, wenn einer aus
dem anderen folgt. U.s.w.}


\PropositionG{4.1212}
{Was gezeigt werden \Emph{kann}, \Emph{kann} nicht gesagt
werden.}


\PropositionG{4.1213}
{Jetzt verstehen wir auch unser Gef&uuml;hl: dass wir
im Besitze einer richtigen logischen Auffassung
seien, wenn nur einmal alles in unserer Zeichensprache
stimmt.}


\PropositionG{4.122}
{
Wir k&ouml;nnen in gewissem Sinne von formalen
Eigenschaften der Gegenst&auml;nde und Sachverhalte
bezw. von Eigenschaften der Struktur der Tatsachen
reden und in demselben Sinne von formalen
Relationen und Relationen von Strukturen.

(Statt Eigenschaft der Struktur sage ich auch
% -----File: 080.png---
\glqq{}interne Eigenschaft\grqq{}; statt Relation der Strukturen
\glqq{}interne Relation\grqq{}.

Ich f&uuml;hre diese Ausdr&uuml;cke ein, um den Grund
der, bei den Philosophen sehr verbreiteten Verwechslung
zwischen den internen Relationen und
den eigentlichen (externen) Relationen zu zeigen.)

Das Bestehen solcher interner Eigenschaften
und Relationen kann aber nicht durch S&auml;tze
behauptet werden, sondern es zeigt sich in den
S&auml;tzen, welche jene Sachverhalte darstellen und
von jenen Gegenst&auml;nden handeln.}


\PropositionG{4.1221}
{Eine interne Eigenschaft einer Tatsache k&ouml;nnen
wir auch einen Zug dieser Tatsache nennen. (In
dem Sinn, in welchem wir etwa von Gesichtsz&uuml;gen
sprechen.)}


\PropositionG{4.123}
{Eine Eigenschaft ist intern, wenn es undenkbar
ist, dass ihr Gegenstand sie nicht besitzt.

(Diese blaue Farbe und jene stehen in der
internen Relation von heller und dunkler eo ipso.
Es ist undenkbar, dass \Emph{diese} beiden Gegenst&auml;nde
nicht in dieser Relation st&uuml;nden.)

(Hier entspricht dem schwankenden Gebrauch
der Worte \glqq{}Eigenschaft\grqq{} und \glqq{}Relation\grqq{} der
schwankende Gebrauch des Wortes \glqq{}Gegenstand\grqq{}.)}


\PropositionG{4.124}
{Das Bestehen einer internen Eigenschaft einer
m&ouml;glichen Sachlage wird nicht durch einen Satz
ausgedr&uuml;ckt, sondern es dr&uuml;ckt sich in dem sie
darstellenden Satz, durch eine interne Eigenschaft
dieses Satzes aus.

Es w&auml;re ebenso unsinnig, dem Satze eine
formale Eigenschaft zuzusprechen, als sie ihm
abzusprechen.}


\PropositionG{4.1241}
{Formen kann man nicht dadurch von einander
unterscheiden, dass man sagt, die eine habe diese,
die andere aber jene Eigenschaft; denn dies setzt
voraus, dass es einen Sinn habe, beide Eigenschaften
von beiden Formen auszusagen.}


\PropositionG{4.125}
{Das Bestehen einer internen Relation zwischen
% -----File: 082.png---
m&ouml;glichen Sachlagen dr&uuml;ckt sich sprachlich durch
eine interne Relation zwischen den sie darstellenden
S&auml;tzen aus.}


\PropositionG{4.1251}
{Hier erledigt sich nun die Streitfrage \glqq{}ob alle
Relationen intern oder extern\grqq{} seien.}


\PropositionG{4.1252}
{Reihen, welche durch \Emph{interne} Relationen
geordnet sind, nenne ich Formenreihen.

Die Zahlenreihe ist nicht nach einer externen,
sondern nach einer internen Relation geordnet.

{\stretchyspace
Ebenso die Reihe der S&auml;tze \glqq{}$aRb$\grqq{},
\glqq{}$(\exists x): aRx \DotOp xRb$\grqq{},
\glqq{}$(\exists x,y): aRx \DotOp xRy \DotOp yRb$\grqq{}, \undSoFort}

(Steht $b$ in einer dieser Beziehungen zu $a$, so
nenne ich $b$ einen Nachfolger von $a$.)}


\PropositionG{4.126}
{In dem Sinne, in welchem wir von formalen
Eigenschaften sprechen, k&ouml;nnen wir nun auch
von formalen Begriffen reden.

(Ich f&uuml;hre diesen Ausdruck ein, um den Grund
der Verwechslung der formalen Begriffe mit den
eigentlichen Begriffen, welche die ganze alte Logik
durchzieht, klar zu machen.)

Dass etwas unter einen formalen Begriff als
dessen Gegenstand f&auml;llt, kann nicht durch einen
Satz ausgedr&uuml;ckt werden. Sondern es zeigt sich
an dem Zeichen dieses Gegenstandes selbst. (Der
Name zeigt, dass er einen Gegenstand bezeichnet,
das Zahlenzeichen, dass es eine Zahl bezeichnet etc.)

Die formalen Begriffe k&ouml;nnen ja nicht, wie
die eigentlichen Begriffe, durch eine Funktion
dargestellt werden.

Denn ihre Merkmale, die formalen Eigenschaften,
werden nicht durch Funktionen ausgedr&uuml;ckt.

Der Ausdruck der formalen Eigenschaft ist ein
Zug gewisser Symbole.

Das Zeichen der Merkmale eines formalen
Begriffes ist also ein charakteristischer Zug aller
Symbole, deren Bedeutungen unter den Begriff
fallen.
% -----File: 084.png---

Der Ausdruck des formalen Begriffes also, eine
Satzvariable, in welcher nur dieser charakteristische
Zug konstant ist.}


\PropositionG{4.127}
{Die Satzvariable bezeichnet den formalen
Begriff und ihre Werte die Gegenst&auml;nde, welche
unter diesen Begriff fallen.}


\PropositionG{4.1271}
{Jede Variable ist das Zeichen eines formalen
Begriffes.

Denn jede Variable stellt eine konstante Form
dar, welche alle ihre Werte besitzen, und die als
\enlargethispage{-4pt} % force a line to the next page
formale Eigenschaft dieser Werte aufgefasst werden
kann.}


\PropositionG{4.1272}
{So ist der variable Name \glqq{}$x$\grqq{} das eigentliche
Zeichen des Scheinbegriffes \Emph{Gegenstand}.

Wo immer das Wort \glqq{}Gegenstand\grqq{} (\glqq{}Ding\grqq{},
\glqq{}Sache\grqq{}, etc.) richtig gebraucht wird, wird es in
der Begriffsschrift durch den variablen Namen
ausgedr&uuml;ckt.

Zum Beispiel in dem Satz \glqq{}es gibt 2 Gegenst&auml;nde,
welche\ \ldots\grqq{} durch \glqq{}$(\exists x, y)$ $\ldots$\grqq{}.

Wo immer es anders, also als eigentliches
Begriffswort gebraucht wird, entstehen unsinnige
Scheins&auml;tze.

So kann man \zumBeispiel\ nicht sagen \glqq{}Es gibt Gegenst&auml;nde\grqq{},
wie man etwa sagt \glqq{}Es gibt B&uuml;cher\grqq{}.
Und ebenso wenig \glqq{}Es gibt 100 Gegenst&auml;nde\grqq{},
oder \glqq{}Es gibt $\aleph_0$ Gegenst&auml;nde\grqq{}.

Und es ist unsinnig, von der \Emph{Anzahl aller
Gegenst&auml;nde} zu sprechen.

Dasselbe gilt von den Worten \glqq{}Komplex\grqq{},
\glqq{}Tatsache\grqq{}, \glqq{}Funktion\grqq{}, \glqq{}Zahl\grqq{}, etc.

Sie alle bezeichnen formale Begriffe und werden
in der Begriffsschrift durch Variable, nicht durch
Funktionen oder Klassen dargestellt. (Wie Frege
und Russell glaubten.)

Ausdr&uuml;cke wie \glqq{}1 ist eine Zahl\grqq{}, \glqq{}es gibt nur
Eine Null\grqq{} und alle &auml;hnlichen sind unsinnig.

(Es ist ebenso unsinnig zu sagen \glqq{}es gibt nur
% -----File: 086.png---
eine 1\grqq{}, als es unsinnig w&auml;re, zu sagen: $2 + 2$ ist
um 3 Uhr gleich 4.)}


\PropositionG{4.12721}
{Der formale Begriff ist mit einem Gegenstand,
der unter ihn f&auml;llt, bereits gegeben. Man kann
also nicht Gegenst&auml;nde eines formalen Begriffes
\Emph{und} den formalen Begriff selbst als Grundbegriffe
einf&uuml;hren. Man kann also \zumBeispiel\ nicht den Begriff
der Funktion, und auch spezielle Funktionen (wie
Russell) als Grundbegriffe einf&uuml;hren; oder den
Begriff der Zahl und bestimmte Zahlen.}


\PropositionG{4.1273}
{Wollen wir den allgemeinen Satz: \glqq{}$b$ ist ein
Nachfolger von $a$\grqq{} in der Begriffsschrift ausdr&uuml;cken,
so brauchen wir hierzu einen Ausdruck
f&uuml;r das allgemeine Glied der Formenreihe: $aRb$,
$(\exists x) : aRx \DotOp xRb$, $(\exists x,y) : aRx \DotOp xRy \DotOp yRb$, \ldots Das
allgemeine Glied einer Formenreihe kann man nur
durch eine Variable ausdr&uuml;cken, denn der Begriff:
Glied dieser Formenreihe, ist ein \Emph{formaler}
Begriff. (Dies haben Frege und Russell &uuml;bersehen;
die Art und Weise wie sie allgemeine
S&auml;tze, wie den obigen ausdr&uuml;cken wollen ist daher
falsch; sie enth&auml;lt einen circulus vitiosus.)

Wir k&ouml;nnen das allgemeine Glied der Formenreihe
bestimmen, indem wir ihr erstes Glied
angeben und die allgemeine Form der Operation,
welche das folgende Glied aus dem vorhergehenden
Satz erzeugt.}


\PropositionG{4.1274}
{Die Frage nach der Existenz eines formalen
Begriffes ist unsinnig. Denn kein Satz kann eine
solche Frage beantworten.

(Man kann also \zumBeispiel\ nicht fragen: \glqq{}Gibt es
unanalysierbare Sub\-jekt-Pr&auml;di\-kat\-s&auml;t\-ze?\grqq{})}


\PropositionG{4.128}
{Die logischen Formen sind zahl\EmphPart{los}.

Darum gibt es in der Logik keine ausgezeichneten
Zahlen und darum gibt es keinen philosophischen
Monismus oder Dualismus, etc.}


\PropositionG{4.2}
{Der Sinn des Satzes ist seine &uuml;bereinstimmung,
und Nicht&uuml;bereinstimmung mit den M&ouml;glichkeiten
% -----File: 088.png---
des Bestehens und Nichtbestehens der
Sachverhalte.}


\PropositionG{4.21}
{Der einfachste Satz, der Elementarsatz, behauptet
das Bestehen eines Sachverhaltes.}


\PropositionG{4.211}
{Ein Zeichen des Elementarsatzes ist es, dass
kein Elementarsatz mit ihm in Widerspruch stehen
kann.}


\PropositionG{4.22}
{Der Elementarsatz besteht aus Namen. Er ist
ein Zusammenhang, eine Verkettung, von Namen.}


\PropositionG{4.221}
{Es ist offenbar, dass wir bei der Analyse der
S&auml;tze auf Elementars&auml;tze kommen m&uuml;ssen, die aus
Namen in unmittelbarer Verbindung bestehen.

Es fr&auml;gt sich hier, wie kommt der Satzverband
zustande.}


\PropositionG{4.2211}
{Auch wenn die Welt unendlich komplex ist,
so dass jede Tatsache aus unendlich vielen Sachverhalten
besteht und jeder Sachverhalt aus unendlich
vielen Gegenst&auml;nden zusammengesetzt ist,
auch dann m&uuml;sste es Gegenst&auml;nde und Sachverhalte
geben.}


\PropositionG{4.23}
{Der Name kommt im Satz nur im Zusammenhange
des Elementarsatzes vor.}


\PropositionG{4.24}
{Die Namen sind die einfachen Symbole, ich
deute sie durch einzelne Buchstaben (\glqq{}$x$\grqq{}, \glqq{}$y$\grqq{}, \glqq{}$z$\grqq{})
an.

Den Elementarsatz schreibe ich als Funktion
der Namen in der Form: \glqq{}$fx$\grqq{}, \glqq{}$\phi(x,y\DPtypo{,}{})$\grqq{}, etc.

Oder ich deute ihn durch die Buchstaben $p$, $q$,
$r$ an.}


\PropositionG{4.241}
{Gebrauche ich zwei Zeichen in ein und derselben
Bedeutung, so dr&uuml;cke ich dies aus, indem
ich zwischen beide das Zeichen \glqq{}$=$\grqq{} setze.

\glqq{}$a = b$\grqq{} heisst also: das Zeichen \glqq{}$a$\grqq{} ist durch
das Zeichen \glqq{}$b$\grqq{} ersetzbar.

(F&uuml;hre ich durch eine Gleichung ein neues
Zeichen \glqq{}$b$\grqq{} ein, indem ich bestimme, es solle ein
bereits bekanntes Zeichen \glqq{}$a$\grqq{} ersetzen, so schreibe
ich die Gleichung---Definition---(wie Russell) in
% -----File: 090.png---
der Form \glqq{}$a = b$ Def.\grqq{}. Die Definition ist eine
Zeichenregel.)}


\PropositionG{4.242}
{Ausdr&uuml;cke von der Form \glqq{}$a = b$\grqq{} sind also nur
Behelfe der Darstellung; sie sagen nichts &uuml;ber die
Bedeutung der Zeichen \glqq{}$a$\grqq{}, \glqq{}$b$\grqq{} aus.}


\PropositionG{4.243}
{K&ouml;nnen wir zwei Namen verstehen, ohne zu
wissen, ob sie dasselbe Ding oder zwei verschiedene
Dinge bezeichnen?---K&ouml;nnen wir einen Satz,
worin zwei Namen vorkommen, verstehen, ohne
zu wissen, ob sie Dasselbe oder Verschiedenes
bedeuten?

Kenne ich etwa die Bedeutung eines englischen
und eines gleichbedeutenden deutschen Wortes, so
ist es unm&ouml;glich, dass ich nicht weiss, dass die
beiden gleichbedeutend sind; es ist unm&ouml;glich,
dass ich sie nicht ineinander &uuml;bersetzen kann.

Ausdr&uuml;cke wie \glqq{}$a = a$\grqq{}, oder von diesen abgeleitete,
sind weder Elementars&auml;tze, noch sonst sinnvolle
Zeichen. (Dies wird sich sp&auml;ter zeigen.)}


\PropositionG{4.25}
{Ist der Elementarsatz wahr, so besteht der
Sachverhalt; ist der Elementarsatz falsch, so besteht
der Sachverhalt nicht.}


\PropositionG{4.26}
{Die Angabe aller wahren Elementars&auml;tze beschreibt
die Welt vollst&auml;ndig. Die Welt ist
vollst&auml;ndig beschrieben durch die Angaben aller
Elementars&auml;tze plus der Angabe, welche von ihnen
wahr und welche falsch sind.}


\PropositionG{4.27}
{Bez&uuml;glich des Bestehens und Nichtbestehens von
\enlargethispage{9pt} % enlarge to make the last line fit
$n$ Sachverhalten gibt es $K_{n} = \sum\limits_{\nu = 0}^n\binom{n}{\nu}$ M&ouml;glichkeiten.

Es k&ouml;nnen alle Kombinationen der Sachverhalte
bestehen, die andern nicht bestehen.}


\PropositionG{4.28}
{Diesen Kombinationen entsprechen ebenso viele
M&ouml;glichkeiten der Wahr\-heit---und Falschheit---von
$n$ Elementars&auml;tzen.}


\PropositionG{4.3}
{Die Wahrheitsm&ouml;glichkeiten der Elementars&auml;tze
bedeuten die M&ouml;glichkeiten des Bestehens und
Nichtbestehens der Sachverhalte.}
% -----File: 092.png---


\PropositionG{4.31}
{Die Wahrheitsm&ouml;glichkeiten k&ouml;nnen wir durch
Schemata folgender Art darstellen (\glqq{}W\grqq{} bedeutet
\glqq{}wahr\grqq{}, \glqq{}F\grqq{}, \glqq{}falsch\grqq{}. Die Reihen der \glqq{}W\grqq{} und
\glqq{}F\grqq{} unter der Reihe der Elementars&auml;tze bedeuten
in leichtverst&auml;ndlicher Symbolik deren Wahrheitsm&ouml;glichkeiten):

<div class="truthtable__container">
    <table class="truthtable">
        <tr>
            <th>p</th>
            <th>q</th>
            <th>r</th>
        </tr>
        <tr>
            <td class="truthtable__separator"></td>
            <td class="truthtable__separator"></td>
            <td class="truthtable__separator"></td>
        </tr>
        <tr>
            <td>W</td>
            <td>W</td>
            <td>W</td>
        </tr>
        <tr>
            <td>F</td>
            <td>W</td>
            <td>W</td>
        </tr>
        <tr>
            <td>W</td>
            <td>F</td>
            <td>W</td>
        </tr>
        <tr>
            <td>W</td>
            <td>W</td>
            <td>F</td>
        </tr>
        <tr>
            <td>F</td>
            <td>F</td>
            <td>W</td>
        </tr>
        <tr>
            <td>F</td>
            <td>W</td>
            <td>F</td>
        </tr>
        <tr>
            <td>W</td>
            <td>F</td>
            <td>F</td>
        </tr>
        <tr>
            <td>F</td>
            <td>F</td>
            <td>F</td>
        </tr>
    </table>

    <table class="truthtable">
        <tr>
            <th>p</th>
            <th>q</th>
        </tr>
        <tr>
            <td class="truthtable__separator"></td>
            <td class="truthtable__separator"></td>
            <td class="truthtable__separator"></td>
        </tr>
        <tr>
            <td>W</td>
            <td>W</td>
        </tr>
        <tr>
            <td>F</td>
            <td>W</td>
        </tr>
        <tr>
            <td>W</td>
            <td>F</td>
        </tr>
        <tr>
            <td>F</td>
            <td>F</td>
        </tr>
    </table>

    <table class="truthtable">
        <tr>
            <th>p</th>
        </tr>
        <tr>
            <td class="truthtable__separator"></td>
            <td class="truthtable__separator"></td>
            <td class="truthtable__separator"></td>
        </tr>
        <tr>
            <td>W</td>
        </tr>
        <tr>
            <td>F</td>
        </tr>
    </table>
</div>
}


\PropositionG{4.4}
{Der Satz ist der Ausdruck der &uuml;bereinstimmung
und Nicht&uuml;bereinstimmung mit den Wahrheitsm&ouml;glichkeiten
der Elementars&auml;tze.}


\PropositionG{4.41}
{Die Wahrheitsm&ouml;glichkeiten der Elementars&auml;tze
sind die Bedingungen der Wahrheit und Falschheit
der S&auml;tze.}


\PropositionG{4.411}
{{\stretchyspace
Es ist von vornherein wahrscheinlich, dass die
Einf&uuml;hrung der Elementars&auml;tze f&uuml;r das Verst&auml;ndnis
aller anderen Satzarten grundlegend ist. Ja, das
Verst&auml;ndnis der allgemeinen S&auml;tze h&auml;ngt \Emph{f&uuml;hlbar}
von dem der Elementars&auml;tze ab.}}


\PropositionG{4.42}
{Bez&uuml;glich der &uuml;bereinstimmung und Nicht&uuml;bereinstimmung
eines Satzes mit den Wahrheitsm&ouml;glichkeiten
von $n$ Elementars&auml;tzen gibt es
$\sum\limits_{\kappa = 0}^{K_n}\binom{K_n}{\kappa} = L_{n}$ M&ouml;glichkeiten.}


\PropositionG{4.43}
{Die &uuml;bereinstimmung mit den Wahrheitsm&ouml;glichkeiten
% -----File: 094.png---
k&ouml;nnen wir dadurch ausdr&uuml;cken, indem
wir ihnen im Schema etwa das Abzeichen \glqq{}W\grqq{}
(wahr) zuordnen.

Das Fehlen dieses Abzeichens bedeutet die
Nicht&uuml;bereinstimmung.}


\PropositionG{4.431}
{Der Ausdruck der &uuml;bereinstimmung und Nicht&uuml;bereinstimmung
mit den Wahrheitsm&ouml;glichkeiten
der Elementars&auml;tze dr&uuml;ckt die Wahrheitsbedingungen
des Satzes aus.

Der Satz ist der Ausdruck seiner Wahrheitsbedingungen.

(Frege hat sie daher ganz richtig als Erkl&auml;rung
der Zeichen seiner Begriffsschrift vorausgeschickt.
Nur ist die Erkl&auml;rung des Wahrheitsbegriffes bei
Frege falsch: W&auml;ren \glqq{}das Wahre\grqq{} und \glqq{}das Falsche\grqq{}
wirklich Gegenst&auml;nde und die Argumente in $\Not{p}$
etc.\ dann w&auml;re nach Frege's Bestimmung der Sinn
von \glqq{}$\Not{p}$\grqq{} keineswegs bestimmt.)}


\PropositionG{4.44}
{Das Zeichen, welches durch die Zuordnung
jener Abzeichen \glqq{}W\grqq{} und der Wahrheitsm&ouml;glichkeiten
entsteht, ist ein Satzzeichen.}


\PropositionG{4.441}
{Es ist klar, dass dem Komplex der Zeichen
\glqq{}F\grqq{} und \glqq{}W\grqq{} kein Gegenstand (oder Komplex von
Gegenst&auml;nden) entspricht; so wenig, wie den horizontalen
und vertikalen Strichen oder den Klammern.---\glqq{}Logische
Gegenst&auml;nde\grqq{} gibt es nicht.

Analoges gilt nat&uuml;rlich f&uuml;r alle Zeichen, die dasselbe
ausdr&uuml;cken wie die Schemata der \glqq{}W\grqq{} und \glqq{}F\grqq{}.}


\PropositionG{4.442}
{Es ist \zumBeispiel:

<div class="truthtable__container">
    <span>``</span>
    <table class="truthtable">
        <tr>
            <th>p</th>
            <th>q</th>
            <th class="truthtable__separator--left"></th>
            <th></th>
        </tr>
        <tr>
            <td class="truthtable__separator"></td>
            <td class="truthtable__separator"></td>
            <td class="truthtable__separator--both"></td>
            <td class="truthtable__separator"></td>
        </tr>
        <tr>
            <td>W</td>
            <td>W</td>
            <td class="truthtable__separator--left"></td>
            <td>W</td>
        </tr>
        <tr>
            <td>F</td>
            <td>W</td>
            <td class="truthtable__separator--left"></td>
            <td>W</td>
        </tr>
        <tr>
            <td>W</td>
            <td>F</td>
            <td class="truthtable__separator--left"></td>
            <td></td>
        </tr>
        <tr>
            <td>F</td>
            <td>F</td>
            <td class="truthtable__separator--left"></td>
            <td>W</td>
        </tr>
    </table>
    <span>''</span>
</div>

ein Satzzeichen.

(Frege's \glqq{}\DPtypo{Urteilstrich}{Urteilsstrich}\grqq{} \glqq{}$\vdash$\grqq{} ist logisch ganz
% -----File: 096.png---
bedeutungslos; er zeigt bei Frege (und Russell)
nur an, dass diese Autoren die so bezeichneten
S&auml;tze f&uuml;r wahr halten. \glqq{}$\vdash$\grqq{} geh&ouml;rt daher ebenso
wenig zum Satzgef&uuml;ge, wie etwa die Nummer des
Satzes. Ein Satz kann unm&ouml;glich von sich selbst
aussagen, dass er wahr ist.)

Ist die Reihenfolge der Wahrheitsm&ouml;glichkeiten
im Schema durch eine Kombinationsregel ein f&uuml;r
allemal festgesetzt, dann ist die letzte Kolonne
allein schon ein Ausdruck der Wahrheitsbedingungen.
Schreiben wir diese Kolonne als Reihe
hin, so wird das Satzzeichen zu:

\glqq{}(WW-W)($p$, $q$)\grqq{} oder deutlicher \glqq{}(WWFW)($p$, $q$)\grqq{}.

(Die Anzahl der Stellen in der linken Klammer
ist durch die Anzahl der Glieder in der rechten
bestimmt.)}


\PropositionG{4.45}
{F&uuml;r $n$ Elementars&auml;tze gibt es $L_{n}$ m&ouml;gliche Gruppen
von Wahrheitsbedingungen.

{\stretchyspace
Die Gruppen von Wahrheitsbedingungen,
welche zu den Wahrheitsm&ouml;glichkeiten einer
Anzahl von Elementars&auml;tzen geh&ouml;ren, lassen sich
in eine Reihe ordnen.}}


\PropositionG{4.46}
{Unter den m&ouml;glichen Gruppen von Wahrheitsbedingungen
gibt es zwei extreme F&auml;lle.

In dem einen Fall ist der Satz f&uuml;r s&auml;mtliche
Wahrheitsm&ouml;glichkeiten der Elementars&auml;tze wahr.
Wir sagen, die Wahrheitsbedingungen sind
\Emph{tautologisch}.

Im zweiten Fall ist der Satz f&uuml;r s&auml;mtliche
Wahrheitsm&ouml;glichkeiten falsch: Die Wahrheitsbedingungen
sind \Emph{kontradiktorisch}.

Im ersten Fall nennen wir den Satz eine
Tautologie, im zweiten Fall eine Kontradiktion.}


\PropositionG{4.461}
{Der Satz zeigt was er sagt, die Tautologie und
die Kontradiktion, dass sie nichts sagen.

Die Tautologie hat keine Wahrheitsbedingungen,
denn sie ist bedingungslos wahr; und
% -----File: 098.png---
die Kontradiktion ist unter keiner Bedingung
wahr.

Tautologie und Kontradiktion sind sinnlos.

(Wie der Punkt von dem zwei Pfeile in
entgegengesetzter Richtung auseinandergehen.)

(Ich weiss \zumBeispiel\ nichts &uuml;ber das Wetter, wenn
ich weiss, dass es regnet oder nicht regnet.)}


\PropositionG{4.4611}
{Tautologie und Kontradiktion sind aber nicht
unsinnig; sie geh&ouml;ren zum Symbolismus, und
zwar &auml;hnlich wie die \glqq{}0\grqq{} zum Symbolismus der
Arithmetik.}


\PropositionG{4.462}
{Tautologie und Kontradiktion sind nicht Bilder
der Wirklichkeit. Sie stellen keine m&ouml;gliche
Sachlage dar. Denn jene l&auml;sst \Emph{jede} m&ouml;gliche
Sachlage zu, diese \Emph{keine}.

In der Tautologie heben die Bedingungen der
&uuml;bereinstimmung mit der Welt---die darstellenden
Beziehungen---einander auf, so dass sie in keiner
darstellenden Beziehung zur Wirklichkeit steht.}


\PropositionG{4.463}
{Die Wahrheitsbedingungen bestimmen den
Spielraum, der den Tatsachen durch den Satz
gelassen wird.

(Der Satz, das Bild, das Modell, sind im
negativen Sinne wie ein fester K&ouml;rper, der die
Bewegungsfreiheit der anderen beschr&auml;nkt; im
positiven Sinne, wie der von fester Substanz
begrenzte Raum, worin ein K&ouml;rper Platz hat.)

Die Tautologie l&auml;sst der Wirklichkeit den gan\-zen---un\-end\-li\-chen---lo\-gi\-schen
Raum; die Kontradiktion
erf&uuml;llt den ganzen logischen Raum und l&auml;sst
der Wirklichkeit keinen Punkt. Keine von beiden
kann daher die Wirklichkeit irgendwie bestimmen.}


\PropositionG{4.464}
{Die Wahrheit der Tautologie ist gewiss, des
Satzes m&ouml;glich, der Kontradiktion unm&ouml;glich.

(Gewiss, m&ouml;glich, unm&ouml;glich: Hier haben wir
das Anzeichen jener Gradation, die wir in der
Wahrscheinlichkeitslehre brauchen.)}


\PropositionG{4.465}
{Das logische Produkt einer Tautologie und
% -----File: 100.png---
eines Satzes sagt dasselbe, wie der Satz. Also ist
jenes Produkt identisch mit dem Satz. Denn man
kann das Wesentliche des Symbols nicht &auml;ndern,
ohne seinen Sinn zu &auml;ndern.}


\PropositionG{4.466}
{Einer bestimmten logischen Verbindung von
Zeichen entspricht eine bestimmte logische Verbindung
ihrer Bedeutungen; \Emph{jede beliebige}
Verbindung entspricht nur den unverbundenen
Zeichen.

Das heisst, S&auml;tze die f&uuml;r jede Sachlage wahr
sind, k&ouml;nnen &uuml;berhaupt keine Zeichenverbindungen
sein, denn sonst k&ouml;nnten ihnen nur bestimmte
Verbindungen von Gegenst&auml;nden entsprechen.

(Und keiner logischen Verbindung entspricht
\Emph{keine} Verbindung der Gegenst&auml;nde.)

Tautologie und Kontradiktion sind die Grenzf&auml;lle
der Zeichenverbindung, n&auml;mlich ihre Aufl&ouml;sung.}


\PropositionG{4.4661}
{Freilich sind auch in der Tautologie und Kontradiktion
die Zeichen noch mit einander verbunden,
\dasHeiszt\ sie stehen in Beziehungen zu einander,
aber diese Beziehungen sind bedeutungslos, dem
\Emph{Symbol} unwesentlich.}


\PropositionG{4.5}
{Nun scheint es m&ouml;glich zu sein, die allgemeinste
Satzform anzugeben: das heisst, eine Beschreibung
der S&auml;tze \Emph{irgend einer} Zeichensprache zu geben,
so dass jeder m&ouml;gliche Sinn durch ein Symbol,
auf welches die Beschreibung passt, ausgedr&uuml;ckt
werden kann, und dass jedes Symbol, worauf die
Beschreibung passt, einen Sinn ausdr&uuml;cken kann,
wenn die Bedeutungen der Namen entsprechend
gew&auml;hlt werden.

Es ist klar, dass bei der Beschreibung der
allgemeinsten Satzform \Emph{nur} ihr Wesentliches
beschrieben werden darf,---sonst w&auml;re sie n&auml;mlich
nicht die allgemeinste.

Dass es eine allgemeine Satzform gibt, wird
dadurch bewiesen, dass es keinen Satz geben darf,
dessen Form man nicht h&auml;tte voraussehen (\dasHeiszt\ konstruieren)
% -----File: 102.png---
k&ouml;nnen. Die allgemeine Form des
Satzes ist: Es verh&auml;lt sich so und so.}


\PropositionG{4.51}
{Angenommen, mir w&auml;ren \Emph{alle} Elementars&auml;tze
gegeben: Dann l&auml;sst sich einfach fragen: welche
S&auml;tze kann ich aus ihnen bilden. Und das sind
\Emph{alle} S&auml;tze und \Emph{so} sind sie begrenzt.}


\PropositionG{4.52}
{Die S&auml;tze sind Alles, was aus der Gesamtheit
aller Elementars&auml;tze folgt (nat&uuml;rlich auch daraus,
dass es die \Emph{Gesamtheit aller} ist). (So k&ouml;nnte
man in gewissem Sinne sagen, dass \Emph{alle} S&auml;tze
Verallgemeinerungen der Elementars&auml;tze sind.)}


\PropositionG{4.53}
{Die allgemeine Satzform ist eine Variable.}


\PropositionG{5}
{Der Satz ist eine Wahrheitsfunktion der Elementars&auml;tze.

(Der Elementarsatz ist eine Wahrheitsfunktion
seiner selbst.)}


\PropositionG{5.01}
{Die Elementars&auml;tze sind die Wahrheitsargumente
des Satzes.}


\PropositionG{5.02}
{Es liegt nahe, die Argumente von Funktionen
mit den Indices von Namen zu verwechseln. Ich
erkenne n&auml;mlich sowohl am Argument wie am
Index die Bedeutung des sie enthaltenden Zeichens.

In Russell's \glqq{}$+_{c}$\grqq{} ist \zumBeispiel\ \glqq{}$c$\grqq{} ein Index, der darauf
hinweist, dass das ganze Zeichen das Additionszeichen
f&uuml;r Kardinalzahlen ist. Aber diese Bezeichnung
beruht auf willk&uuml;rlicher &uuml;bereinkunft und
man k&ouml;nnte statt \glqq{}$+_{c}$\grqq{} auch ein einfaches Zeichen
w&auml;hlen; in \glqq{}$\Not{p}$\grqq{} aber ist \glqq{}$p$\grqq{} kein Index, sondern
ein Argument: der Sinn von \glqq{}$\Not{p}$\grqq{} \Emph{kann nicht}
verstanden werden, ohne dass vorher der Sinn von
\glqq{}$p$\grqq{} verstanden worden w&auml;re. (Im Namen Julius
C&auml;sar ist \glqq{}Julius\grqq{} ein Index. Der Index ist immer
ein Teil einer Beschreibung des Gegenstandes,
dessen Namen wir ihn anh&auml;ngen. \ZumBeispiel\ \Emph{Der}
C&auml;sar aus dem Geschlechte der Julier.)

Die Verwechslung von Argument und Index
liegt, wenn ich mich nicht irre, der Theorie Frege's
von der Bedeutung der S&auml;tze und Funktionen
% -----File: 104.png---
zugrunde. F&uuml;r Frege waren die S&auml;tze der Logik
Namen, und deren Argumente die Indices dieser
Namen.}


\PropositionG{5.1}
{Die Wahrheitsfunktionen lassen sich in Reihen
ordnen.

Das ist die Grundlage der Wahrscheinlichkeitslehre.}


\PropositionG{5.101}
{Die Wahrheitsfunktionen jeder Anzahl von
Elementars&auml;tzen lassen sich in einem Schema
folgender Art hinschreiben:

<table class="table5101">
    <tr>
        <td>(WWWW)($p, q$)</td> 
        <td>Tautologie</td> 
        <td>(Wenn $p$, so $p$; und wenn $q$, so $q$.) <br/> ($p \Implies p \DotOp q \Implies q$)</td>
    </tr>
    <tr>
        <td>(FWWW)($p, q$)</td>
        <td>in Worten:</td> 
        <td>Nicht beides $p$ und $q$. ($\Not{(p \DotOp q)}$)</td>
    </tr>
    <tr>
        <td>(WFWW)($p, q$)</td> 
        <td>" &nbsp &nbsp "</td>
        <td>Wenn $q$, so $p$. ($q \Implies p$)</td>
    </tr>
    <tr>
        <td>(WWFW)($p, q$)</td> 
        <td>" &nbsp &nbsp "</td>
        <td>Wenn $p$, so $q$. ($p \Implies q$)</td>
    </tr>
    <tr>
        <td>(WWWF)($p, q$)</td>
        <td>" &nbsp &nbsp "</td>
        <td>$p$ oder $q$. ($p \lor q$)</td>
    </tr>
    <tr>
        <td>(FFWW)($p, q$)</td>
        <td>" &nbsp &nbsp "</td>
        <td>Nicht $q$. ($\Not{q}$)</td>
    </tr>
    <tr>
        <td>(FWFW)($p, q$)</td>
        <td>" &nbsp &nbsp "</td>
        <td>Nicht $p$. ($\Not{p}$)</td>
    </tr>
    <tr>
        <td>(FWWF)($p, q$)</td>
        <td>" &nbsp &nbsp "</td>
        <td>$p$, oder $q$, aber nicht beide. <br/> ($p \DotOp \Not{q} : \lor : q \DotOp \Not{p}$)</td>
    </tr>
    <tr>
        <td>(WFFW)($p, q$)</td>
        <td>" &nbsp &nbsp "</td>
        <td>Wenn $p$, so $q$; und wenn $q$, so $p$. </br> ($p \equiv q$)</td>
    </tr>
    <tr>
        <td>(WFWF)($p, q$)</td>
        <td>" &nbsp &nbsp "</td>
        <td>$p$</td>
    </tr>
    <tr>
        <td>(WWFF)($p, q$)</td>
        <td>" &nbsp &nbsp "</td>
        <td>$q$</td>
    </tr>
    <tr>
        <td>(FFFW)($p, q$)</td>
        <td>" &nbsp &nbsp "</td>
        <td>Weder $p$ noch $q$. <br/> ($\Not{p} \DotOp \Not{q}$) oder ($p \BarOp q$)</td>
    </tr>
    <tr>
        <td>(FFWF)($p, q$)</td>
        <td>" &nbsp &nbsp "</td>
        <td>$p$ und nicht $q$. ($p \DotOp \Not{q}$)</td>
    </tr>
    <tr>
        <td>(FWFF)($p, q$)</td>
        <td>" &nbsp &nbsp "</td>
        <td>$q$ und nicht $p$. ($q \DotOp \Not{p}$)</td>
    </tr>
    <tr>
        <td>(WFFF)($p, q$)</td>
        <td>" &nbsp &nbsp "</td>
        <td>$q$ und $p$. ($q \DotOp p$)</td>
    </tr>
    <tr>
        <td>(FFFF)($p, q$)</td>
        <td>Kontradiktion</td>
        <td>($p$ und nicht $p$; und $q$ und nicht $q$.) <br/> ($p \DotOp \Not{p} \DotOp q \DotOp \Not{q}$)</td>
    </tr>
</table>

{\verystretchyspace
Diejenigen Wahrheitsm&ouml;glichkeiten seiner
Wahrheitsargumente, welche den Satz bewahrheiten,
will ich seine \Emph{Wahrheitsgr&uuml;nde}
nennen.}}


\PropositionG{5.11}
{Sind die Wahrheitsgr&uuml;nde, die einer Anzahl
von S&auml;tzen gemeinsam sind, s&auml;mtlich auch Wahrheitsgr&uuml;nde
eines bestimmten Satzes, so sagen
wir, die Wahrheit dieses Satzes folge aus der
Wahrheit jener S&auml;tze.}


\PropositionG{5.12}
{Insbesondere folgt die Wahrheit eines Satzes
\glqq{}$p$\grqq{} aus der Wahrheit eines anderen \glqq{}$q$\grqq{}, wenn
alle Wahrheitsgr&uuml;nde des zweiten Wahrheitsgr&uuml;nde
des ersten sind.}
% -----File: 106.png---


\PropositionG{5.121}
{Die Wahrheitsgr&uuml;nde des einen sind in denen
des anderen enthalten; $p$ folgt aus $q$.}


\PropositionG{5.122}
{Folgt $p$ aus $q$, so ist der Sinn von \glqq{}$p$\grqq{} im
Sinne von \glqq{}$q$\grqq{} enthalten.}


\PropositionG{5.123}
{Wenn ein Gott eine Welt erschafft, worin
gewisse S&auml;tze wahr sind, so schafft er damit auch
schon eine Welt, in welcher alle ihre Folges&auml;tze
stimmen. Und &auml;hnlich k&ouml;nnte er keine Welt
schaffen, worin der Satz \glqq{}$p$\grqq{} wahr ist, ohne seine
s&auml;mtlichen Gegenst&auml;nde zu schaffen.}


\PropositionG{5.124}
{Der Satz bejaht jeden Satz der aus ihm
folgt.}


\PropositionG{5.1241}
{\glqq{}$p \DotOp q$\grqq{} ist einer der S&auml;tze, welche \glqq{}$p$\grqq{} bejahen
und zugleich einer der S&auml;tze, welche \glqq{}$q$\grqq{}
bejahen.

Zwei S&auml;tze sind einander entgegengesetzt, wenn
es keinen sinnvollen Satz gibt, der sie beide
bejaht.

Jeder Satz der einem anderen widerspricht,
verneint ihn.}


\PropositionG{5.13}
{Dass die Wahrheit eines Satzes aus der Wahrheit
anderer S&auml;tze folgt, ersehen wir aus der
Struktur der S&auml;tze.}


\PropositionG{5.131}
{Folgt die Wahrheit eines Satzes aus der Wahrheit
anderer, so dr&uuml;ckt sich dies durch Beziehungen
aus, in welchen die Formen jener S&auml;tze zu
einander stehen; und zwar brauchen wir sie nicht
erst in jene Beziehungen zu setzen, indem wir
sie in einem Satze miteinander verbinden, sondern
diese Beziehungen sind intern und bestehen, sobald,
und dadurch dass, jene S&auml;tze bestehen.}


\PropositionG{5.1311}
{Wenn wir von $p \lor q$ und $\Not{p}$ auf $q$ schliessen,
so ist hier durch die Bezeichnungsweise die Beziehung
der Satzformen von \glqq{}$p \lor q$\grqq{} und \glqq{}$\Not{p}$\grqq{} verh&uuml;llt.
Schreiben wir aber \zumBeispiel\ statt \glqq{}$p \lor q$\grqq{}
\glqq{}$p \BarOp q \DotOp \BarOp \DotOp p \BarOp q$\grqq{} und statt \glqq{}$\Not{p}$\grqq{} \glqq{}$p \BarOp p$\grqq{} ($p \BarOp q$ = weder
$p$, noch $q$), so wird der innere Zusammenhang
offenbar.
% -----File: 108.png---

(Dass man aus $(x) \DotOp fx$ auf $fa$ schliessen kann,
das zeigt, dass die Allgemeinheit auch im Symbol
\glqq{}$(x) \DotOp fx$\grqq{} vorhanden ist.)}


\PropositionG{5.132}
{Folgt $p$ aus $q$, so kann ich von $q$ auf $p$ schliessen;
$p$ aus $q$ folgern.

Die Art des Schlusses ist allein aus den beiden
S&auml;tzen zu entnehmen.

Nur sie selbst k&ouml;nnen den Schluss rechtfertigen.

{\stretchyspace
\glqq{}Schlussgesetze\grqq{}, welche---wie bei Frege und
Russell---die Schl&uuml;sse rechtfertigen sollen, sind
sinnlos, und w&auml;ren &uuml;berfl&uuml;ssig.}}


\PropositionG{5.133}
{Alles Folgern geschieht a priori.}


\PropositionG{5.134}
{Aus einem Elementarsatz l&auml;sst sich kein anderer
folgern.}


\PropositionG{5.135}
{Auf keine Weise kann aus dem Bestehen irgend
einer Sachlage auf das Bestehen einer, von ihr g&auml;nzlich
verschiedenen Sachlage geschlossen werden.}


\PropositionG{5.136}
{Einen Kausalnexus, der einen solchen Schluss
rechtfertigte, gibt es nicht.}


\PropositionG{5.1361}
{Die Ereignisse der Zukunft \Emph{k&ouml;nnen} wir nicht
aus den gegenw&auml;rtigen erschliessen.

Der Glaube an den Kausalnexus ist der \Emph{Aberglaube}.}


\PropositionG{5.1362}
{Die Willensfreiheit besteht darin, dass zuk&uuml;nftige
Handlungen jetzt nicht gewusst werden k&ouml;nnen.
Nur dann k&ouml;nnten wir sie wissen, wenn die Kausalit&auml;t
eine \Emph{innere} Notwendigkeit w&auml;re, wie die
des logischen Schlusses.---Der Zusammenhang
von Wissen und Gewusstem, ist der der logischen
Notwendigkeit.

(\glqq{}A weiss, dass $p$ der Fall ist\grqq{} ist sinnlos, wenn
$p$ eine Tautologie ist.)}


\PropositionG{5.1363}
{Wenn daraus, dass ein Satz uns einleuchtet,
nicht \Emph{folgt}, dass er wahr ist, so ist das Einleuchten
auch keine Rechtfertigung f&uuml;r unseren
Glauben an seine Wahrheit.}


\PropositionG{5.14}
{Folgt ein Satz aus einem anderen, so sagt
dieser mehr als jener, jener weniger als dieser.}
% -----File: 110.png---


\PropositionG{5.141}
{Folgt $p$ aus $q$ und $q$ aus $p$, so sind sie ein und
derselbe Satz.}


\PropositionG{5.142}
{Die Tautologie folgt aus allen S&auml;tzen: sie sagt
Nichts.}


\PropositionG{5.143}
{Die Kontradiktion ist das Gemeinsame der
S&auml;tze, was \Emph{kein} Satz mit einem anderen gemein
hat. Die Tautologie ist das Gemeinsame aller
S&auml;tze, welche nichts miteinander gemein haben.

Die Kontradiktion verschwindet sozusagen
ausserhalb, die Tautologie innerhalb aller S&auml;tze.

Die Kontradiktion ist die &auml;ussere Grenze der
S&auml;tze, die Tautologie ihr substanzloser Mittelpunkt.}


\PropositionG{5.15}
{Ist $W_{r}$ die Anzahl der Wahrheitsgr&uuml;nde des
Satzes \glqq{}$r$\grqq{}, $W_{rs}$ die Anzahl derjenigen Wahrheitsgr&uuml;nde
des Satzes \glqq{}$s$\grqq{}, die zugleich Wahrheitsgr&uuml;nde
von \glqq{}$r$\grqq{} sind, dann nennen wir das Verh&auml;ltnis: $W_{rs} :
W_{r}$ das Mass der \Emph{Wahrscheinlichkeit}, welche
der Satz \glqq{}$r$\grqq{} dem Satz \glqq{}$s$\grqq{} gibt.}


\PropositionG{5.151}
{Sei in einem Schema wie dem obigen in No.~\PropGRef{5.101}
$W_{r}$ die Anzahl der \glqq{}$W$\grqq{} im Satze $r$; $W_{rs}$ die
Anzahl derjenigen \glqq{}$W$\grqq{} im Satze $s$, die in gleichen
Kolonnen mit \glqq{}$W$\grqq{} des Satzes $r$ stehen. Der Satz
$r$ gibt dann dem Satze $s$ die Wahrscheinlichkeit:
$W_{rs} : W_{r}$.}


\PropositionG{5.1511}
{Es gibt keinen besonderen Gegenstand, der den
Wahrscheinlichkeitss&auml;tzen eigen w&auml;re.}


\PropositionG{5.152}
{S&auml;tze, welche keine Wahrheitsargumente mit
einander gemein haben, nennen wir von einander
unabh&auml;ngig.

Von einander unabh&auml;ngige S&auml;tze (\zumBeispiel\ irgend
zwei Elementars&auml;tze) geben einander die Wahrscheinlichkeit $\frac{1}{2}$.

Folgt $p$ aus $q$, so gibt der Satz \glqq{}$q$\grqq{} dem Satz
\glqq{}$p$\grqq{} die Wahrscheinlichkeit 1. Die Gewissheit
des logischen Schlusses ist ein Grenzfall der
Wahrscheinlichkeit.

(Anwendung auf Tautologie und Kontradiktion.)}


\PropositionG{5.153}
{Ein Satz ist an sich weder wahrscheinlich noch
% -----File: 112.png---
unwahrscheinlich. Ein Ereignis trifft ein, oder
es trifft nicht ein, ein Mittelding gibt es nicht.}


\PropositionG{5.154}
{In einer Urne seien gleichviel weisse und
schwarze Kugeln (und keine anderen). Ich ziehe
eine Kugel nach der anderen und lege sie wieder
in die Urne zur&uuml;ck. Dann kann ich durch den
Versuch feststellen, dass sich die Zahlen der
gezogenen schwarzen und weissen Kugeln bei
fortgesetztem Ziehen einander n&auml;hern.

\Emph{Das} ist also kein mathematisches Faktum.

Wenn ich nun sage: Es ist gleich wahrscheinlich,
dass ich eine weisse Kugel wie eine
schwarze ziehen werde, so heisst das: Alle mir
bekannten Umst&auml;nde (die hypothetisch angenommenen
Naturgesetze mitinbegriffen) geben dem
Eintreffen des einen Ereignisses nicht \Emph{mehr}
Wahrscheinlichkeit als dem Eintreffen des anderen.
Das heisst, sie geben---wie aus den obigen Erkl&auml;rungen
leicht zu entnehmen ist---jedem die
Wahrscheinlichkeit $\frac{1}{2}$.

Was ich durch den Versuch best&auml;tige ist, dass
das Eintreffen der beiden Ereignisse von den Umst&auml;nden,
die ich nicht n&auml;her kenne, unabh&auml;ngig ist.}


\PropositionG{5.155}
{Die Einheit des Wahrscheinlichkeitssatzes ist:
Die Umst&auml;nde---die ich sonst nicht weiter kenne---geben
dem Eintreffen eines bestimmten Ereignisses
den und den Grad der Wahrscheinlichkeit.}


\PropositionG{5.156}
{So ist die Wahrscheinlichkeit eine Verallgemeinerung.

Sie involviert eine allgemeine Beschreibung
einer Satzform.

Nur in Ermanglung der Gewissheit gebrauchen
wir die Wahr\-schein\-lich\-keit.---Wenn wir zwar eine
Tatsache nicht vollkommen kennen, wohl aber
\Emph{etwas} &uuml;ber ihre Form wissen.

(Ein Satz kann zwar ein unvollst&auml;ndiges Bild
einer gewissen Sachlage sein, aber er ist immer
\Emph{ein} vollst&auml;ndiges Bild.)
% -----File: 114.png---

Der Wahrscheinlichkeitssatz ist gleichsam ein
Auszug aus anderen S&auml;tzen.}


\PropositionG{5.2}
{Die Strukturen der S&auml;tze stehen in internen
Beziehungen zu einander.}


\PropositionG{5.21}
{Wir k&ouml;nnen diese internen Beziehungen
dadurch in unserer Ausdrucksweise hervorheben,
dass wir einen Satz als Resultat einer Operation
darstellen, die ihn aus anderen S&auml;tzen (den Basen
der Operation) hervorbringt.}


\PropositionG{5.22}
{Die Operation ist der Ausdruck einer Beziehung
zwischen den Strukturen ihres Resultats und ihrer
Basen.}


\PropositionG{5.23}
{Die Operation ist das, was mit dem einen Satz
geschehen muss, um aus ihm den anderen zu machen.}


\PropositionG{5.231}
{Und das wird nat&uuml;rlich von ihren formalen
Eigenschaften, von der internen &auml;hnlichkeit ihrer
Formen abh&auml;ngen.}


\PropositionG{5.232}
{Die interne Relation, die eine Reihe ordnet, ist
&auml;quivalent mit der Operation, durch welche ein
Glied aus dem anderen entsteht.}


\PropositionG{5.233}
{Die Operation kann erst dort auftreten, wo ein
Satz auf logisch bedeutungsvolle Weise aus einem
anderen entsteht. Also dort, wo die logische
Konstruktion des Satzes anf&auml;ngt.}


\PropositionG{5.234}
{Die Wahrheitsfunktionen der Elementars&auml;tze
sind Resultate von Operationen, die die Elementars&auml;tze
als Basen haben. (Ich nenne diese Operationen
Wahrheitsoperationen.)}


\PropositionG{5.2341}
{Der Sinn einer Wahrheitsfunktion von $p$ ist
eine Funktion des Sinnes von $p$.

Verneinung, logische Addition, logische Multiplikation,
etc., etc.\ sind Operationen.

(Die Verneinung verkehrt den Sinn des Satzes.)}


\PropositionG{5.24}
{Die Operation zeigt sich in einer Variablen;
sie zeigt, wie man von einer Form von S&auml;tzen zu
einer anderen gelangen kann.

Sie bringt den Unterschied der Formen zum
Ausdruck.
% -----File: 116.png---

(Und das Gemeinsame zwischen den Basen
und dem Resultat der Operation sind eben die
Basen.)}


\PropositionG{5.241}
{Die Operation kennzeichnet keine Form, sondern
nur den Unterschied der Formen.}


\PropositionG{5.242}
{Dieselbe Operation, die \glqq{}$q$\grqq{} aus \glqq{}$p$\grqq{} macht,
macht aus \glqq{}$q$\grqq{} \glqq{}$r$\grqq{} \undSoFort Dies kann nur darin
ausgedr&uuml;ckt sein, dass \glqq{}$p$\grqq{}, \glqq{}$q$\grqq{}, \glqq{}$r$\grqq{}, etc.\ Variable
sind, die gewisse formale Relationen allgemein
zum Ausdruck bringen.}


\PropositionG{5.25}
{Das Vorkommen der Operation charakterisiert
den Sinn des Satzes nicht.

Die Operation sagt ja nichts aus, nur ihr Resultat,
und dies h&auml;ngt von den Basen der Operation
ab.

(Operation und Funktion d&uuml;rfen nicht miteinander
verwechselt werden.)}


\PropositionG{5.251}
{Eine Funktion kann nicht ihr eigenes Argument
sein, wohl aber kann das Resultat einer Operation
ihre eigene Basis werden.}


\PropositionG{5.252}
{Nur so ist das Fortschreiten von Glied zu Glied
in einer Formenreihe (von Type zu Type in den
Hierarchien Russells und Whiteheads) m&ouml;glich.
(Russell und Whitehead haben die M&ouml;glichkeit
dieses Fortschreitens nicht zugegeben, aber immer
wieder von ihr Gebrauch gemacht.)}


\PropositionG{5.2521}
{Die fortgesetzte Anwendung einer Operation
\enlargethispage{2pt} % enlarge to make one more line fit
auf ihr eigenes Resultat nenne ich ihre successive
Anwendung (\glqq{}$O' O' O' a$\grqq{} ist das Resultat der
dreimaligen successiven Anwendung von \glqq{}$O' \xi$\grqq{}
auf \glqq{}$a$\grqq{}).

In einem &auml;hnlichen Sinne rede ich von der
successiven Anwendung \Emph{mehrerer} Operationen
auf eine Anzahl von S&auml;tzen.}


\PropositionG{5.2522}
{
Das allgemeine Glied einer Formenreihe $a$, $O' a$,
$O' O' a$, $\fourdots$ schreibe ich daher so: \glqq{}$[a, x, O' x]$\grqq{}.
Dieser Klammerausdruck ist eine Variable. Das
erste Glied des Klammerausdruckes ist der Anfang
% -----File: 118.png---
der Formenreihe, das zweite die Form eines
beliebigen Gliedes $x$ der Reihe und das dritte
die Form desjenigen Gliedes der Reihe, welches
auf $x$ unmittelbar folgt.}


\PropositionG{5.2523}
{Der Begriff der successiven Anwendung der
Operation ist &auml;quivalent mit dem Begriff \glqq{}und so
weiter\grqq{}.}


\PropositionG{5.253}
{Eine Operation kann die Wirkung einer anderen
r&uuml;ckg&auml;ngig machen. Operationen k&ouml;nnen einander
aufheben.}


\PropositionG{5.254}
{Die Operation kann verschwinden (\zumBeispiel\ die
Verneinung in \glqq{}$\Not{\Not{p}}$\grqq{} $\Not{\Not{p}}$ $= p$).}


\PropositionG{5.3}
{Alle S&auml;tze sind Resultate von Wahrheitsoperationen
mit den Elementars&auml;tzen.

Die Wahrheitsoperation ist die Art und Weise,
wie aus den Elementars&auml;tzen die Wahrheitsfunktion
entsteht.

{\verystretchyspace
Nach dem Wesen der Wahrheitsoperation wird
auf die gleiche Weise, wie aus den Elementars&auml;tzen
ihre Wahrheitsfunktion, aus Wahrheitsfunktionen
eine Neue. Jede Wahrheitsoperation erzeugt aus
Wahrheitsfunktionen von Elementars&auml;tzen wieder
eine Wahrheitsfunktion von Elementars&auml;tzen, einen
Satz. Das Resultat jeder Wahrheitsoperation mit
den Resultaten von Wahrheitsoperationen mit
Elementars&auml;tzen ist wieder das Resultat \Emph{Einer}
Wahrheitsoperation mit Elementars&auml;tzen.}

Jeder Satz ist das Resultat von Wahrheitsoperationen
mit Elementars&auml;tzen.}


\PropositionG{5.31}
{Die Schemata No.~\PropGRef{4.31} haben auch dann eine
Bedeutung, wenn \glqq{}$p$\grqq{}, \glqq{}$q$\grqq{}, \glqq{}$r$\grqq{}, etc.\ nicht Elementars&auml;tze
sind.

{\verystretchyspace
Und es ist leicht zu sehen, dass das Satzzeichen
in No.~\DPtypo{\PropGRef{4.42}}{\PropGRef{4.442}}, auch wenn \glqq{}$p$\grqq{} und \glqq{}$q$\grqq{} Wahrheitsfunktionen
von Elementars&auml;tzen sind, Eine
Wahrheitsfunktion von Elementars&auml;tzen ausdr&uuml;ckt.}}


\PropositionG{5.32}
{Alle Wahrheitsfunktionen sind Resultate der
% -----File: 120.png---
successiven Anwendung einer endlichen Anzahl
von Wahrheitsoperationen auf die Elementars&auml;tze.}


\PropositionG{5.4}
{Hier zeigt es sich, dass es \glqq{}logische Gegenst&auml;nde\grqq{},
\glqq{}logische Konstante\grqq{} (im Sinne Freges
und Russells) nicht gibt.}


\PropositionG{5.41}
{Denn: Alle Resultate von Wahrheitsoperationen
mit Wahrheitsfunktionen sind identisch,
welche eine und dieselbe Wahrheitsfunktion von
Elementars&auml;tzen sind.}


\PropositionG{5.42}
{Dass $\lor$, $\Implies$, etc.\ nicht Beziehungen im Sinne von
rechts und links etc.\ sind, leuchtet ein.

Die M&ouml;glichkeit des kreuzweisen Definierens der
logischen \glqq{}Urzeichen\grqq{} Freges und Russells zeigt
schon, dass dies keine Urzeichen sind, und schon
erst recht, dass sie keine Relationen bezeichnen.

Und es ist offenbar, dass das \glqq{}$\Implies$\grqq{}, welches wir
durch \glqq{}$\Not{}$\grqq{} und \glqq{}$\lor$\grqq{} definieren, identisch ist mit dem,
durch welches wir \glqq{}$\lor$\grqq{} mit \glqq{}$\Not{}$\grqq{} definieren und dass
dieses \glqq{}$\lor$\grqq{} mit dem ersten identisch ist. \UndSoWeiter}


\PropositionG{5.43}
{Dass aus einer Tatsache $p$ unendlich viele
\Emph{andere} folgen sollten, n&auml;mlich $\Not{\Not{p}}$, $\Not{\Not{\Not{\Not{p}}}}$,
etc., ist doch von vornherein kaum zu glauben.
Und nicht weniger merkw&uuml;rdig ist, dass die unendliche
Anzahl der S&auml;tze der Logik (der Mathematik)
aus einem halben Dutzend \glqq{}Grundgesetzen\grqq{} folgen.

Alle S&auml;tze der Logik sagen aber dasselbe. N&auml;mlich
Nichts.}


\PropositionG{5.44}
{Die Wahrheitsfunktionen sind keine materiellen
Funktionen.

Wenn man \zumBeispiel\ eine Bejahung durch doppelte
Verneinung erzeugen kann, ist dann die Verneinung---in
irgend einem Sinn\AllowBreak---in der Bejahung enthalten?
Verneint \glqq{}$\Not{\Not{p}}$\grqq{} $\Not{p}$, oder bejaht es $p$; oder beides?

Der Satz \glqq{}$\Not{\Not{p}}$\grqq{} handelt nicht von der Verneinung
wie von einem Gegenstand; wohl aber ist
die M&ouml;glichkeit der Verneinung in der Bejahung
bereits pr&auml;judiziert.

Und g&auml;be es einen Gegenstand, der \glqq{}$\Not{}$\grqq{} hiesse,
% -----File: 122.png---
so m&uuml;sste \glqq{}$\Not{\Not{p}}$\grqq{} etwas anderes sagen als \glqq{}$p$\grqq{}.
Denn der eine Satz w&uuml;rde dann eben von $\Not{}$
handeln, der andere nicht.}


\PropositionG{5.441}
{Dieses Verschwinden der scheinbaren logischen
Konstanten tritt auch ein, wenn \glqq{}$\Not{(\exists x) \DotOp \Not{fx}}$\grqq{}
dasselbe sagt wie \glqq{}$(x) \DotOp fx$\grqq{}, oder \glqq{}$(\exists x) \DotOp fx \DotOp x = a$\grqq{}
dasselbe wie \glqq{}$fa$\grqq{}.}


\PropositionG{5.442}
{Wenn uns ein Satz gegeben ist, so sind \Emph{mit
ihm} auch schon die Resultate aller Wahrheitsoperationen,
die ihn zur Basis haben, gegeben.}


\PropositionG{5.45}
{Gibt es logische Urzeichen, so muss eine richtige
Logik ihre Stellung zueinander klar machen und
ihr Dasein rechtfertigen. Der Bau der Logik \Emph{aus}
ihren Urzeichen muss klar werden.}


\PropositionG{5.451}
{Hat die Logik Grundbegriffe, so m&uuml;ssen sie von
einander unabh&auml;ngig sein. Ist ein Grundbegriff
eingef&uuml;hrt, so muss er in allen Verbindungen
eingef&uuml;hrt sein, worin er &uuml;berhaupt vorkommt. Man
kann ihn also nicht zuerst f&uuml;r \Emph{eine} Verbindung,
dann noch einmal f&uuml;r eine andere einf&uuml;hren.
\ZumBeispiel: Ist die Verneinung eingef&uuml;hrt, so m&uuml;ssen
wir sie jetzt in S&auml;tzen von der Form \glqq{}$\Not{p}$\grqq{} ebenso
verstehen, wie in S&auml;tzen wie \glqq{}$\Not{(p \lor q)}$\grqq{}, \glqq{}$(\exists x) \DotOp \Not{fx}$\grqq{}~\undAndere\
Wir d&uuml;rfen sie nicht erst f&uuml;r die eine Klasse
von F&auml;llen, dann f&uuml;r die andere einf&uuml;hren, denn es
bliebe dann zweifelhaft, ob ihre Bedeutung in beiden
F&auml;llen die gleiche w&auml;re und es w&auml;re kein Grund
vorhanden, in beiden F&auml;llen dieselbe Art der
Zeichenverbindung zu ben&uuml;tzen.

(Kurz, f&uuml;r die Einf&uuml;hrung der Urzeichen gilt,
mutatis mutandis, dasselbe, was Frege (\glqq{}Grundgesetze
der Arithmetik\grqq{}) f&uuml;r die Einf&uuml;hrung von
Zeichen durch Definitionen gesagt hat.)}


\PropositionG{5.452}
{Die Einf&uuml;hrung eines neuen Behelfes in den Symbolismus
der Logik muss immer ein folgenschweres
Ereignis sein. Kein neuer Behelf darf in die Logik---sozusagen,
mit ganz unschuldiger Miene---in Klammern
oder unter dem Striche eingef&uuml;hrt werden.
% -----File: 124.png---

(So kommen in den \glqq{}Principia Mathematica\grqq{}
von Russell und Whitehead Definitionen und
Grundgesetze in Worten vor. Warum hier pl&ouml;tzlich
Worte? Dies bed&uuml;rfte einer Rechtfertigung.
Sie fehlt und muss fehlen, da das Vorgehen tats&auml;chlich
unerlaubt ist.)

Hat sich aber die Einf&uuml;hrung eines neuen
Behelfes an einer Stelle als n&ouml;tig erwiesen, so muss
man sich nun sofort fragen: Wo muss dieser
Behelf nun \Emph{immer} angewandt werden? Seine
Stellung in der Logik muss nun erkl&auml;rt werden.}


\PropositionG{5.453}
{Alle Zahlen der Logik m&uuml;ssen sich rechtfertigen
lassen.

Oder vielmehr: Es muss sich herausstellen,
dass es in der Logik keine Zahlen gibt.

Es gibt keine ausgezeichneten Zahlen.}


\PropositionG{5.454}
{In der Logik gibt es kein Nebeneinander, kann
es keine Klassifikation geben.

In der Logik kann es nicht Allgemeineres und
Spezielleres geben.}


\PropositionG{5.4541}
{Die L&ouml;sungen der logischen Probleme m&uuml;ssen
einfach sein, denn sie setzen den Standard der
Einfachheit.

Die Menschen haben immer geahnt, dass es ein
Gebiet von Fragen geben m&uuml;sse, deren Antworten---a
priori---symmetrisch, und zu einem abgeschlossenen,
regelm&auml;ssigen Gebilde vereintliegen.

Ein Gebiet, in dem der Satz gilt: simplex
sigillum veri.}


\PropositionG{5.46}
{Wenn man die logischen Zeichen richtig
einf&uuml;hrte, so h&auml;tte man damit auch schon den Sinn
aller ihrer Kombinationen eingef&uuml;hrt; also nicht
nur \glqq{}$p \lor q$\grqq{} sondern auch schon \glqq{}$\Not{(p \lor \Not{q})}$\grqq{} etc.\ etc.
Man h&auml;tte damit auch schon die Wirkung
aller nur m&ouml;glichen Kombinationen von Klammern
eingef&uuml;hrt. Und damit w&auml;re es klar geworden,
dass die eigentlichen allgemeinen Urzeichen nicht
% -----File: 126.png---
die \glqq{}$p \lor q$\grqq{}, \glqq{}$(\exists x) \DotOp fx$\grqq{}, etc.\ sind, sondern die allgemeinste
Form ihrer Kombinationen.}


\PropositionG{5.461}
{Bedeutungsvoll ist die scheinbar unwichtige
Tatsache, dass die logischen Scheinbeziehungen,
wie $\lor$ und $\Implies$, der Klammern be\-d&uuml;r\-fen---im Gegensatz
zu den wirklichen Beziehungen.

Die Ben&uuml;tzung der Klammern mit jenen scheinbaren
Urzeichen deutet ja schon darauf hin, dass
diese nicht die wirklichen Urzeichen sind. Und
es wird doch wohl niemand glauben, dass die
Klammern eine selbst&auml;ndige Bedeutung haben.}


\PropositionG{5.4611}
{Die logischen Operationszeichen sind Interpunktionen.}


\PropositionG{5.47}
{Es ist klar, dass alles was sich &uuml;berhaupt \Emph{von
vornherein} &uuml;ber die Form aller S&auml;tze sagen
l&auml;sst, sich \Emph{aufeinmal} sagen lassen muss.

Sind ja schon im Elementarsatze alle logischen
\enlargethispage{1pt} % enlarge to make the last line fit
Operationen enthalten. Denn \glqq{}$fa$\grqq{} sagt dasselbe
wie \glqq{}$(\exists x) \DotOp fx \DotOp x = a$\grqq{}.

Wo Zusammengesetztheit ist, da ist Argument
und Funktion, und wo diese sind, sind bereits alle
logischen Konstanten.

Man k&ouml;nnte sagen: Die Eine logische Konstante
ist das, was \Emph{alle} S&auml;tze, ihrer Natur nach, mit
einander gemein haben.

Das aber ist die allgemeine Satzform.}


\PropositionG{5.471}
{Die allgemeine Satzform ist das Wesen des
Satzes.}


\PropositionG{5.4711}
{Das Wesen des Satzes angeben, heisst, das
Wesen aller Beschreibung angeben, also das
Wesen der Welt.}


\PropositionG{5.472}
{Die Beschreibung der allgemeinsten Satzform
ist die Beschreibung des einen und einzigen
allgemeinen Urzeichens der Logik.}


\PropositionG{5.473}
{Die Logik muss f&uuml;r sich selber sorgen.

Ein \Emph{m&ouml;gliches} Zeichen muss auch bezeichnen
k&ouml;nnen. Alles was in der Logik m&ouml;glich ist, ist
auch erlaubt. (\glqq{}Sokrates ist identisch\grqq{} heisst darum
% -----File: 128.png---
nichts, weil es keine Eigenschaft gibt, die
\glqq{}identisch\grqq{} heisst. Der Satz ist unsinnig, weil
wir eine willk&uuml;rliche Bestimmung nicht getroffen
haben, aber nicht darum, weil das Symbol an und
f&uuml;r sich unerlaubt w&auml;re.)

Wir k&ouml;nnen uns, in gewissem Sinne, nicht in
der Logik irren.}


\PropositionG{5.4731}
{Das Einleuchten, von dem Russell so viel
sprach, kann nur dadurch in der Logik entbehrlich
werden, dass die Sprache selbst jeden logischen
Fehler ver\-hin\-dert.---Dass die Logik a priori ist,
besteht darin, dass nicht unlogisch gedacht werden
\Emph{kann}.}


\PropositionG{5.4732}
{Wir k&ouml;nnen einem Zeichen nicht den unrechten
Sinn geben.}


\PropositionG{5.47321}
{Occams Devise ist nat&uuml;rlich keine willk&uuml;rliche,
oder durch ihren praktischen Erfolg gerechtfertigte,
Regel: Sie besagt, dass \Emph{unn&ouml;tige} Zeicheneinheiten
nichts bedeuten.

Zeichen, die \Emph{Einen} Zweck erf&uuml;llen, sind logisch
&auml;quivalent, Zeichen, die \Emph{keinen} Zweck erf&uuml;llen,
logisch bedeutungslos.}


\PropositionG{5.4733}
{Frege sagt: Jeder rechtm&auml;ssig gebildete Satz
muss einen Sinn haben; und ich sage: Jeder
m&ouml;gliche Satz ist rechtm&auml;ssig gebildet, und wenn er
keinen Sinn hat, so kann das nur daran liegen, dass
wir einigen seiner Bestandteile keine \Emph{Bedeutung}
gegeben haben.

(Wenn wir auch glauben, es getan zu haben.)

So sagt \glqq{}Sokrates ist identisch\grqq{} darum nichts,
weil wir dem Wort \glqq{}identisch\grqq{} als \Emph{Eigenschaftswort}
\Emph{keine} Bedeutung gegeben haben. Denn,
wenn es als Gleichheitszeichen auftritt, so symbolisiert
es auf ganz andere Art und Weise---die
bezeichnende Beziehung ist eine an\-de\-re,---also ist
auch das Symbol in beiden F&auml;llen ganz verschieden;
die beiden Symbole haben nur das Zeichen zuf&auml;llig
miteinander gemein.}
% -----File: 130.png---


\PropositionG{5.474}
{Die Anzahl der n&ouml;tigen Grundoperationen h&auml;ngt
\Emph{nur} von unserer Notation ab.}


\PropositionG{5.475}
{Es kommt nur darauf an, ein Zeichensystem von
einer bestimmten Anzahl von Dimensionen---von
einer bestimmten mathematischen Man\-nig\-fal\-tig\-keit---zu
bilden.}


\PropositionG{5.476}
{Es ist klar, dass es sich hier nicht um eine
\Emph{Anzahl von Grundbegriffen} handelt, die
bezeichnet werden m&uuml;ssen, sondern um den
Ausdruck einer Regel.}


\PropositionG{5.5}
{Jede Wahrheitsfunktion ist ein Resultat der
successiven Anwendung der Operation \mbox{(- - - - -W)}\AllowBreak($\xi, \fourdots$)
auf Elementars&auml;tze.

Diese Operation verneint s&auml;mtliche S&auml;tze in der
rechten Klammer und ich nenne sie die Negation
dieser S&auml;tze.}


\PropositionG{5.501}
{Einen Klammerausdruck, dessen Glieder S&auml;tze
sind, deute ich\AllowBreak---wenn die Reihenfolge der Glieder in
der Klammer gleichg&uuml;ltig ist---durch ein Zeichen von
der Form \glqq{}$(\overline{\xi})$\grqq{} an. \glqq{}$\xi$\grqq{} ist eine Variable, deren Werte
die Glieder des Klammerausdruckes sind; und der
Strich &uuml;ber der Variablen deutet an, dass sie ihre
s&auml;mtlichen Werte in der Klammer vertritt.

(Hat also $\xi$ etwa die 3 Werte P, Q, R, so ist
($\overline{\xi}$) = (P, Q, R).)

Die Werte der Variablen werden festgesetzt.

Die Festsetzung ist die Beschreibung der S&auml;tze,
welche die Variable vertritt.

Wie die Beschreibung der Glieder des Klammerausdruckes
geschieht, ist unwesentlich.

Wir \Emph{k&ouml;nnen} drei Arten der Beschreibung
unterscheiden: 1. Die direkte Aufz&auml;hlung. In
diesem Fall k&ouml;nnen wir statt der Variablen einfach
ihre konstanten Werte setzen. 2. Die Angabe
einer Funktion $fx$, deren Werte f&uuml;r alle Werte von
$x$ die zu beschreibenden S&auml;tze sind. 3. Die Angabe
eines formalen Gesetzes, nach welchem jene S&auml;tze
gebildet sind. In diesem Falle sind die Glieder des
% -----File: 132.png---
Klammerausdrucks s&auml;mtliche Glieder einer Formenreihe.}


\PropositionG{5.502}
{Ich schreibe also statt \mbox{\glqq{}(- - - - -W)}\AllowBreak($\xi, \fourdots$)\grqq{}
\glqq{}$N(\overline{\xi})$\grqq{}.

$N(\overline{\xi})$ ist die Negation s&auml;mtlicher Werte der
Satzvariablen $\xi$.}


\PropositionG{5.503}
{Da sich offenbar leicht ausdr&uuml;cken l&auml;sst, wie mit
dieser Operation S&auml;tze gebildet werden k&ouml;nnen und
wie S&auml;tze mit ihr nicht zu bilden sind, so muss
dies auch einen exakten Ausdruck finden k&ouml;nnen.}


\PropositionG{5.51}
{Hat $\xi$ nur einen Wert, so ist $N(\overline{\xi}) = \Not{p}$ (nicht $p$),
hat es zwei Werte, so ist $N(\overline{\xi}) = \Not{p} \DotOp \Not{q}$ (weder
$p$ noch $q$).}


\PropositionG{5.511}
{Wie kann die allumfassende, weltspiegelnde
Logik so spezielle Haken und Manipulationen
gebrauchen? Nur, indem sich alle diese zu einem
unendlich feinen Netzwerk, zu dem grossen Spiegel,
verkn&uuml;pfen.}


\PropositionG{5.512}
{\glqq{}$\Not{p}$\grqq{} ist wahr, wenn \glqq{}$p$\grqq{} falsch ist. Also in
dem wahren Satz \glqq{}$\Not{p}$\grqq{} ist \glqq{}$p$\grqq{} ein falscher Satz.
Wie kann ihn nun der Strich \glqq{}$\Not{}$\grqq{} mit der Wirklichkeit
zum Stimmen bringen?

Das, was in \glqq{}$\Not{p}$\grqq{} verneint, ist aber nicht das
\glqq{}$\Not{}$\grqq{}, sondern dasjenige, was allen Zeichen dieser
Notation, welche $p$ verneinen, gemeinsam ist.

Also die gemeinsame Regel, nach welcher
\glqq{}$\Not{p}$\grqq{}, \glqq{}$\Not{\Not{\Not{p}}}$\grqq{}, \glqq{}$\Not{p} \lor \Not{p}$\grqq{}, \glqq{}$\Not{p} \DotOp \Not{p}$\grqq{}, etc.\ etc.\ (ad
inf.) gebildet werden. Und dies Gemeinsame
spiegelt die Verneinung \DPtypo{wieder}{wider}.}


\PropositionG{5.513}
{Man k&ouml;nnte sagen: Das Gemeinsame aller Symbole,
die sowohl $p$ als $q$ bejahen, ist der Satz
\glqq{}$p \DotOp q$\grqq{}. Das Gemeinsame aller Symbole, die
entweder $p$ oder $q$ bejahen, ist der Satz \glqq{}$p \lor q$\grqq{}.

Und so kann man sagen: Zwei S&auml;tze sind
einander entgegengesetzt, wenn sie nichts miteinander
gemein haben, und: Jeder Satz hat nur ein
Negativ, weil es nur einen Satz gibt, der ganz
ausserhalb seiner liegt.
% -----File: 134.png---

Es zeigt sich so auch in Russells Notation, dass
\glqq{}$q : p \lor \Not{p}$\grqq{} dasselbe sagt wie \glqq{}$q$\grqq{}; dass \glqq{}$p \lor \Not{p}$\grqq{}
\DPtypo{nichtssagt}{nichts sagt}.}


\PropositionG{5.514}
{Ist eine Notation festgelegt, so gibt es in ihr eine
Regel, nach der alle $p$ verneinenden \DPtypo{S&auml;tz}{S&auml;tze} gebildet
werden, eine Regel, nach der alle $p$ bejahenden
S&auml;tze gebildet werden, eine Regel, nach der alle
$p$ oder $q$ bejahenden S&auml;tze gebildet werden, \undSoFort
Diese Regeln sind den Symbolen &auml;quivalent
und in ihnen spiegelt sich ihr Sinn \DPtypo{wieder}{wider}.}


\PropositionG{5.515}
{Es muss sich an unseren Symbolen zeigen, dass
das, was durch \glqq{}$\lor$\grqq{}, \glqq{}$\DotOp$\grqq{}, etc.\ miteinander verbunden
ist, S&auml;tze sein m&uuml;ssen.

Und dies ist auch der Fall, denn das Symbol \glqq{}$p$\grqq{}
und \glqq{}$q$\grqq{} setzt ja selbst das \glqq{}$\lor$\grqq{}, \glqq{}$\Not{}$\grqq{}, etc.\ voraus.
Wenn das Zeichen \glqq{}$p$\grqq{} in \glqq{}$p \lor q$\grqq{} nicht f&uuml;r ein komplexes
Zeichen steht, dann kann es allein nicht
Sinn haben; dann k&ouml;nnen aber auch die mit \glqq{}$p$\grqq{}
gleichsinnigen Zeichen \glqq{}$p \lor p$\grqq{}, \glqq{}$p \DotOp p$\grqq{}, etc.\ keinen
Sinn haben. Wenn aber \glqq{}$p \lor p$\grqq{} keinen Sinn hat,
dann kann auch \glqq{}$p \lor q$\grqq{} keinen Sinn haben.}


\PropositionG{5.5151}
{Muss das Zeichen des negativen Satzes mit dem
Zeichen des positiven gebildet werden? Warum
sollte man den negativen Satz nicht durch eine negative
Tatsache ausdr&uuml;cken k&ouml;nnen. (Etwa: Wenn
\glqq{}$a$\grqq{} nicht in einer bestimmten Beziehung zu \glqq{}$b$\grqq{} steht,
k&ouml;nnte das ausdr&uuml;cken, dass $aRb$ nicht der Fall ist.)

Aber auch hier ist ja der negative Satz indirekt
durch den positiven gebildet.

Der positive \Emph{Satz} muss die Existenz des negativen
\Emph{Satzes} voraussetzen und umgekehrt.}


\PropositionG{5.52}
{Sind die Werte von $\xi$ s&auml;mtliche Werte einer
Funktion $fx$ f&uuml;r alle Werte von $x$, so wird
$N(\overline{\xi}) = \Not{(\exists x) \DotOp fx}$.}


\PropositionG{5.521}
{Ich trenne den Begriff \Emph{Alle} von der Wahrheitsfunktion.

Frege und Russell haben die Allgemeinheit in
Verbindung mit dem logischen Produkt oder der
% -----File: 136.png---
logischen Summe eingef&uuml;hrt. So wurde es schwer,
die S&auml;tze \glqq{}$(\exists x) \DotOp fx$\grqq{} und \glqq{}$(x) \DotOp fx$\grqq{}, in welchen beide
Ideen beschlossen liegen, zu verstehen.}


\PropositionG{5.522}
{Das Eigent&uuml;mliche der Allgemeinheitsbezeichnung
ist erstens, dass sie auf ein logisches Urbild
hinweist, und zweitens, dass sie Konstante
hervorhebt.}


\PropositionG{5.523}
{Die Allgemeinheitsbezeichnung tritt als Argument
auf.}


\PropositionG{5.524}
{Wenn die Gegenst&auml;nde gegeben sind, so sind
uns damit auch schon \Emph{alle} Gegenst&auml;nde gegeben.

Wenn die Elementars&auml;tze gegeben sind, so sind
damit auch \Emph{alle} Elementars&auml;tze gegeben.}


\PropositionG{5.525}
{Es ist unrichtig, den Satz \glqq{}$(\exists x) \DotOp fx$\grqq{}---wie
Russell dies tut---in Worten durch \glqq{}$fx$ ist \Emph{m&ouml;glich}\grqq{}
wiederzugeben.

Gewissheit, M&ouml;glichkeit oder Unm&ouml;glichkeit
einer Sachlage wird nicht durch einen Satz ausgedr&uuml;ckt,
sondern dadurch, dass ein Ausdruck eine
Tautologie, ein sinnvoller Satz, oder eine Kontradiktion
ist.

Jener Pr&auml;zedenzfall, auf den man sich immer
berufen m&ouml;chte, muss schon im Symbol selber
liegen.}


\PropositionG{5.526}
{Man kann die Welt vollst&auml;ndig durch vollkommen
verallgemeinerte S&auml;tze beschreiben, das
heisst also, ohne irgend einen Namen von vornherein
einem bestimmten Gegenstand zuzuordnen.

Um dann auf die gew&ouml;hnliche Ausdrucksweise
zu kommen, muss man einfach nach einem Ausdruck
\glqq{}es gibt ein und nur ein $x$, welches~$\fourdots$\grqq{} sagen:
Und dies $x$ ist $a$.}


\PropositionG{5.5261}
{Ein vollkommen verallgemeinerter Satz ist, wie
jeder andere Satz zusammengesetzt. (Dies zeigt
sich daran, dass wir in \glqq{}$(\exists x, \phi) \DotOp \phi x$\grqq{} \glqq{}$\phi$\grqq{} und \glqq{}$x$\grqq{}
getrennt erw&auml;hnen m&uuml;ssen. Beide stehen unabh&auml;ngig
in bezeichnenden Beziehungen zur Welt,
wie im unverallgemeinerten Satz.)
% -----File: 138.png---

Kennzeichen des zusammengesetzten Symbols:
Es hat etwas mit \Emph{anderen} Symbolen gemeinsam.}


\PropositionG{5.5262}
{Es ver&auml;ndert ja die Wahr- oder Falschheit \Emph{jedes}
Satzes etwas am allgemeinen Bau der Welt. Und
der Spielraum, welcher ihrem Bau durch die
Gesamtheit der Elementars&auml;tze gelassen wird, ist
eben derjenige, welchen die ganz allgemeinen
S&auml;tze begrenzen.

(Wenn ein Elementarsatz wahr ist, so ist damit
doch jedenfalls Ein Elementarsatz \Emph{mehr} wahr.)}


\PropositionG{5.53}
{{\verystretchyspace
Gleichheit des Gegenstandes dr&uuml;cke ich durch
Gleichheit des Zeichens aus, und nicht mit Hilfe
eines Gleichheitszeichens. Verschiedenheit der
\enlargethispage{7pt} % enlarge to make the last word fit
Gegenst&auml;nde durch Verschiedenheit der Zeichen.}}


\PropositionG{5.5301}
{Dass die Identit&auml;t keine Relation zwischen Gegenst&auml;nden
ist, leuchtet ein. Dies wird sehr klar,
wenn man \zumBeispiel\ den Satz \glqq{}$(x) : fx \DotOp \Implies \DotOp x = a$\grqq{}
betrachtet. Was dieser Satz sagt, ist einfach,
dass \Emph{nur} $a$ der Funktion $f$ gen&uuml;gt, und nicht,
dass nur solche Dinge der Funktion $f$ gen&uuml;gen,
welche eine gewisse Beziehung zu $a$ haben.

Man k&ouml;nnte nun freilich sagen, dass eben \Emph{nur}
$a$ diese Beziehung zu $a$ habe, aber um dies auszudr&uuml;cken,
brauchten wir das Gleichheitszeichen
selber.}


\PropositionG{5.5302}
{Russells Definition von \glqq{}$=$\grqq{} gen&uuml;gt nicht; weil
man nach ihr nicht sagen kann, dass zwei Gegenst&auml;nde
alle Eigenschaften gemeinsam haben.
(Selbst wenn dieser Satz nie richtig ist, hat er
doch \Emph{Sinn}.)}


\PropositionG{5.5303}
{Beil&auml;ufig gesprochen: Von \Emph{zwei} Dingen zu
sagen, sie seien identisch, ist ein Unsinn, und von
\Emph{Einem} zu sagen, es sei identisch mit sich selbst,
sagt gar nichts.}


\PropositionG{5.531}
{Ich schreibe also nicht \glqq{}$f(a,b) \DotOp a = b$\grqq{}, sondern
\glqq{}$f(a,a)$\grqq{} (oder \glqq{}$f(b,b)$\grqq{}). Und nicht \glqq{}$f(a,b) \DotOp \Not{a = b}$\grqq{},
sondern \glqq{}$f(a,b)$\grqq{}.}


\PropositionG{5.532}
{Und analog: Nicht \glqq{}$(\exists x,y) \DotOp f(x,y) \DotOp x = y$\grqq{},
% -----File: 140.png---
sondern \glqq{}$(\exists x) \DotOp f(x,x)$\grqq{}; und nicht \glqq{}$(\exists x,y) \DotOp f(x,y) \DotOp
\Not{x = y}$\grqq{}, sondern \glqq{}$(\exists x,y) \DotOp f(x,y)$\grqq{}.

(Also statt des Russell'schen \glqq{}$(\exists x,y) \DotOp f(x,y)$\grqq{}:
\glqq{}$(\exists x,y) \DotOp f(x,y) \DotOp \lor \DotOp (\exists x) \DotOp f(x,x)$\grqq{}.)}


\PropositionG{5.5321}
{Statt \glqq{}$(x) : fx \Implies x = a$\grqq{} schreiben wir also \zumBeispiel\ \glqq{}$(\exists
x) \DotOp fx \DotOp \Implies \DotOp fa : \Not{(\exists x,y) \DotOp fx \DotOp fy}$\grqq{}.

Und der Satz \glqq{}\Emph{nur} Ein $x$ befriedigt $f()$\grqq{} lautet:
\glqq{}$(\exists x) \DotOp fx : \Not{(\exists x,y) \DotOp fx \DotOp fy}$\grqq{}.}


\PropositionG{5.533}
{Das Gleichheitszeichen ist also kein wesentlicher
Bestandteil der Begriffsschrift.}


\PropositionG{5.534}
{Und nun sehen wir, dass Scheins&auml;tze wie:
\glqq{}$a = a$\grqq{}, \glqq{}$a = b \DotOp b = c \DotOp \Implies a = c$\grqq{}, \glqq{}$(x) \DotOp x = x$\grqq{}, \glqq{}$(\exists x) \DotOp
x = a$\grqq{}, etc.\ sich in einer richtigen Begriffsschrift gar
nicht hinschreiben lassen.}


\PropositionG{5.535}
{Damit erledigen sich auch alle Probleme, die
an solche Scheins&auml;tze gekn&uuml;pft waren.

Alle Probleme, die Russells \glqq{}Axiom of Infinity\grqq{}
\enlargethispage{7pt} % enlarge to make the last line fit
mit sich bringt, sind schon hier zu l&ouml;sen.

Das, was das Axiom of infinity sagen soll, w&uuml;rde
sich in der Sprache dadurch ausdr&uuml;cken, dass es
unendlich viele Namen mit verschiedener Bedeutung
g&auml;be.}


\PropositionG{5.5351}
{Es gibt gewisse F&auml;lle, wo man in Versuchung
ger&auml;t, Ausdr&uuml;cke von der Form \glqq{}$a = a$\grqq{} oder \glqq{}$p \Implies p$\grqq{}
u. dgl. zu ben&uuml;tzen. Und zwar geschieht dies,
wenn man von dem Urbild: Satz, Ding, etc.\ reden
m&ouml;chte. So hat Russell in den \glqq{}Principles of
Mathematics\grqq{} den Unsinn \glqq{}$p$ ist ein Satz\grqq{} in Symbolen
durch \glqq{}$p \Implies p$\grqq{} wiedergegeben und als Hypothese
vor gewisse S&auml;tze gestellt, damit deren
Argumentstellen nur von S&auml;tzen besetzt werden
k&ouml;nnten.

(Es ist schon darum Unsinn, die Hypothese
$p \Implies p$ vor einen Satz zu stellen, um ihm Argumente
der richtigen Form zu sichern, weil die Hypothese
f&uuml;r einen Nicht-Satz als Argument nicht falsch,
sondern unsinnig wird, und weil der Satz selbst
durch die unrichtige Gattung von Argumenten
% -----File: 142.png---
unsinnig wird, also sich selbst ebenso gut, oder so
schlecht, vor den unrechten Argumenten bewahrt,
wie die zu diesem Zweck angeh&auml;ngte sinnlose
Hypothese.)}


\PropositionG{5.5352}
{Ebenso wollte man \glqq{}Es gibt keine \Emph{Dinge}\grqq{} ausdr&uuml;cken
durch \glqq{}$\Not{(\exists x) \DotOp x = x}$\grqq{}. Aber selbst wenn
dies ein Satz w&auml;re,---w&auml;re er nicht auch wahr, wenn
es zwar \glqq{}Dinge g&auml;be\grqq{}, aber diese nicht mit sich
selbst identisch w&auml;ren?}


\PropositionG{5.54}
{In der allgemeinen Satzform kommt der Satz im
Satze nur als Basis der Wahrheitsoperationen vor.}


\PropositionG{5.541}
{Auf den ersten Blick scheint es, als k&ouml;nne ein Satz
in einem anderen auch auf andere Weise vorkommen.

Besonders in gewissen Satzformen der Psychologie,
wie \glqq{}A glaubt, dass $p$ der Fall ist\grqq{}, oder
\glqq{}A denkt $p$\grqq{}, etc.

Hier scheint es n&auml;mlich oberfl&auml;chlich, als st&uuml;nde
der Satz $p$ zu einem Gegenstand A in einer Art
von Relation.

(Und in der modernen Erkenntnistheorie (Russell,
Moore, etc.) sind jene S&auml;tze auch so aufgefasst
worden.)}


\PropositionG{5.542}
{Es ist aber klar, dass \glqq{}A glaubt, dass $p$\grqq{}, \glqq{}A
denkt $p$\grqq{}, \glqq{}A sagt $p$\grqq{} von der Form \glqq{}\glq{}$p$\grq{} sagt $p$\grqq{} sind:
Und hier handelt es sich nicht um eine Zuordnung
von einer Tatsache und einem Gegenstand, sondern
um die Zuordnung von Tatsachen durch Zuordnung
ihrer Gegenst&auml;nde.}


\PropositionG{5.5421}
{Dies zeigt auch, dass die Seele---das Subjekt,
etc.---wie sie in der heutigen oberfl&auml;chlichen Psychologie
aufgefasst wird, ein Unding ist.

{\verystretchyspace
Eine zusammengesetzte Seele w&auml;re n&auml;mlich
keine Seele mehr.}}


\PropositionG{5.5422}
{Die richtige Erkl&auml;rung der Form des Satzes \glqq{}A
urteilt $p$\grqq{} muss zeigen, dass es unm&ouml;glich ist, einen
Unsinn zu urteilen. (Russells Theorie gen&uuml;gt
dieser Bedingung nicht.)}


\PropositionG{5.5423}
{Einen Komplex wahrnehmen, heisst, wahrnehmen,
% -----File: 144.png---
dass sich seine Bestandteile so und so zu einander
verhalten.

Dies erkl&auml;rt wohl auch, dass man die Figur
\Illustration{cube}
auf zweierlei Art als W&uuml;rfel sehen kann; und alle
&auml;hnlichen Erscheinungen. Denn wir sehen eben
wirklich zwei verschiedene Tatsachen.

(Sehe ich erst auf die Ecken $a$ und nur fl&uuml;chtig
auf $b$, so erscheint $a$ vorne; und umgekehrt.)}


\PropositionG{5.55}
{Wir m&uuml;ssen nun die Frage nach allen m&ouml;glichen
Formen der Elementars&auml;tze a priori beantworten.

Der Elementarsatz besteht aus Namen. Da wir
aber die Anzahl der Namen von verschiedener
Bedeutung nicht angeben k&ouml;nnen, so k&ouml;nnen wir
auch nicht die Zusammensetzung des Elementarsatzes
angeben.}


\PropositionG{5.551}
{Unser Grundsatz ist, dass jede Frage, die sich
&uuml;berhaupt durch die Logik entscheiden l&auml;sst, sich
ohne weiteres entscheiden lassen muss.

(Und wenn wir in die Lage kommen, ein solches
Problem durch Ansehen der Welt beantworten zu
m&uuml;ssen, so zeigt dies, dass wir auf grundfalscher
F&auml;hrte sind.)}


\PropositionG{5.552}
{Die \glqq{}Erfahrung\grqq{}, die wir zum Verstehen der
Logik brauchen, ist nicht die, dass sich etwas so
und so verh&auml;lt, sondern, dass etwas \Emph{ist}: aber das
ist eben \Emph{keine} Erfahrung.

Die Logik ist \Emph{vor} jeder Erfahrung---dass etwas
\Emph{so} ist.

Sie ist vor dem Wie, nicht vor dem Was.}
% -----File: 146.png---


\PropositionG{5.5521}
{Und wenn dies nicht so w&auml;re, wie k&ouml;nnten wir
die Logik anwenden? Man k&ouml;nnte sagen: Wenn
es eine Logik g&auml;be, auch wenn es keine Welt g&auml;be,
wie k&ouml;nnte es dann eine Logik geben, da es eine
Welt gibt.}


\PropositionG{5.553}
{Russell sagte, es g&auml;be einfache Relationen
zwischen verschiedenen Anzahlen von Dingen
(Individuals). Aber zwischen welchen Anzahlen?
Und wie soll sich das entscheiden?---Durch die
Erfahrung?

(Eine ausgezeichnete Zahl gibt es nicht.)}


\PropositionG{5.554}
{Die Angabe jeder speziellen Form w&auml;re vollkommen
willk&uuml;rlich.}


\PropositionG{5.5541}
{Es soll sich a priori angeben lassen, ob ich \zumBeispiel\ in
die Lage kommen kann, etwas mit dem
Zeichen einer 27-stelligen Relation bezeichnen zu
m&uuml;ssen.}


\PropositionG{5.5542}
{D&uuml;rfen wir denn aber &uuml;berhaupt so fragen?
K&ouml;nnen wir eine Zeichenform aufstellen und nicht
wissen, ob ihr etwas entsprechen k&ouml;nne?

Hat die Frage einen Sinn: Was muss \Emph{sein},
damit etwas der-Fall-sein kann?}


\PropositionG{5.555}
{Es ist klar, wir haben vom Elementarsatz einen
Begriff, abgesehen von seiner besonderen logischen
Form.

Wo man aber Symbole nach einem System
bilden kann, dort ist dieses System das logisch
wichtige und nicht die einzelnen Symbole.

Und wie w&auml;re es auch m&ouml;glich, dass ich es in
der Logik mit Formen zu tun h&auml;tte, die ich erfinden
kann; sondern mit dem muss ich es zu tun haben,
was es mir m&ouml;glich macht, sie zu erfinden.}


\PropositionG{5.556}
{Eine Hierarchie der Formen der Elementars&auml;tze
kann es nicht geben. Nur was wir selbst
konstruieren, k&ouml;nnen wir voraussehen.}


\PropositionG{5.5561}
{Die empirische Realit&auml;t ist begrenzt durch die
Gesamtheit der Gegenst&auml;nde. Die Grenze zeigt
sich wieder in der Gesamtheit der Elementars&auml;tze.
% -----File: 148.png---

Die Hierarchien sind, und m&uuml;ssen unabh&auml;ngig
von der Realit&auml;t sein.}


\PropositionG{5.5562}
{Wissen wir aus rein logischen Gr&uuml;nden, dass
es Elementars&auml;tze geben muss, dann muss es jeder
wissen, der die S&auml;tze in ihrer unanalysierten Form
versteht.}


\PropositionG{5.5563}
{Alle S&auml;tze unserer Umgangssprache sind tats&auml;chlich,
so wie sie sind, logisch vollkommen geordnet.---Jenes
Einfachste, was wir hier angeben sollen,
ist nicht ein Gleichnis der Wahrheit, sondern die
volle Wahrheit selbst.

(Unsere Probleme sind nicht abstrakt, sondern
vielleicht die konkretesten, die es gibt.)}


\PropositionG{5.557}
{Die \Emph{Anwendung} der Logik entscheidet
dar&uuml;ber, welche Elementars&auml;tze es gibt.

Was in der Anwendung liegt, kann die Logik
nicht vorausnehmen.

Das ist klar: Die Logik darf mit ihrer Anwendung
nicht kollidieren.

Aber die Logik muss sich mit ihrer Anwendung
ber&uuml;hren.

Also d&uuml;rfen die Logik und ihre Anwendung
einander nicht &uuml;bergreifen.}


\PropositionG{5.5571}
{Wenn ich die Elementars&auml;tze nicht a priori
angeben kann, dann muss es zu offenbarem Unsinn
f&uuml;hren, sie angeben zu wollen.}


\PropositionG{5.6}
{\Emph{Die Grenzen meiner Sprache} bedeuten
die Grenzen meiner Welt.}


\PropositionG{5.61}
{Die Logik erf&uuml;llt die Welt; die Grenzen der
Welt sind auch ihre Grenzen.

Wir k&ouml;nnen also in der Logik nicht sagen: Das
und das gibt es in der Welt, jenes nicht.

Das w&uuml;rde n&auml;mlich scheinbar voraussetzen, dass
\enlargethispage{9pt} % enlarge to make one more line fit
wir gewisse M&ouml;glichkeiten ausschliessen und dies
kann nicht der Fall sein, da sonst die Logik
&uuml;ber die Grenzen der Welt hinaus m&uuml;sste; wenn
sie n&auml;mlich diese Grenzen auch von der anderen
Seite betrachten k&ouml;nnte.
% -----File: 150.png---

Was wir nicht denken k&ouml;nnen, das k&ouml;nnen wir
nicht denken; wir k&ouml;nnen also auch nicht \Emph{sagen},
was wir nicht denken k&ouml;nnen.}


\PropositionG{5.62}
{Diese Bemerkung gibt den Schl&uuml;ssel zur
Entscheidung der Frage, inwieweit der Solipsismus
eine Wahrheit ist.

Was der Solipsismus n&auml;mlich \Emph{meint}, ist ganz
richtig, nur l&auml;sst es sich nicht \Emph{sagen}, sondern es
zeigt sich.

Dass die Welt \Emph{meine} Welt ist, das zeigt sich darin,
dass die Grenzen \Emph{der} Sprache (der Sprache, die allein
ich verstehe) die Grenzen \Emph{meiner} Welt bedeuten.}


\PropositionG{5.621}
{Die Welt und das Leben sind Eins.}


\PropositionG{5.63}
{Ich bin meine Welt. (Der Mikrokosmos.)}


\PropositionG{5.631}
{Das denkende, vorstellende, Subjekt gibt es nicht.

Wenn ich ein Buch schriebe \glqq{}Die Welt, wie ich
sie vorfand\grqq{}, so w&auml;re darin auch &uuml;ber meinen Leib
zu berichten und zu sagen, welche Glieder meinem
Willen unterstehen und welche nicht etc., dies ist
n&auml;mlich eine Methode, das Subjekt zu isolieren,
oder vielmehr zu zeigen, dass es in einem wichtigen
Sinne kein Subjekt gibt: Von ihm allein n&auml;mlich
k&ouml;nnte in diesem Buche \Emph{nicht} die Rede sein.---}


\PropositionG{5.632}
{Das Subjekt geh&ouml;rt nicht zur Welt, sondern es
ist eine Grenze der Welt.}


\PropositionG{5.633}
{Wo in der Welt ist ein \DPtypo{methaphysisches}{metaphysisches} Subjekt
zu merken?

Du sagst, es verh&auml;lt sich hier ganz, wie mit Auge
und Gesichtsfeld. Aber das Auge siehst du wirklich
\Emph{nicht}.

Und nichts \Emph{am Gesichtsfeld} l&auml;sst darauf
schliessen, dass es von einem Auge gesehen wird.}


\PropositionG{5.6331}
{Das Gesichtsfeld hat n&auml;mlich nicht etwa eine
solche Form:
\Illustration{sight-de}
}
% -----File: 152.png---


\PropositionG{5.634}
{Das h&auml;ngt damit zusammen, dass kein Teil
unserer Erfahrung auch a priori ist.

Alles, was wir sehen, k&ouml;nnte auch anders
sein.

Alles, was wir &uuml;berhaupt beschreiben k&ouml;nnen,
k&ouml;nnte auch anders sein.

Es gibt keine Ordnung der Dinge a priori.}


\PropositionG{5.64}
{Hier sieht man, dass der Solipsismus, streng
durchgef&uuml;hrt, mit dem reinen Realismus zusammenf&auml;llt.
Das Ich des Solipsismus schrumpft zum
ausdehnungslosen Punkt zusammen, und es bleibt
die ihm koordinierte Realit&auml;t.}


\PropositionG{5.641}
{Es gibt also wirklich einen Sinn, in welchem in
der Philosophie nicht-psy\-cho\-lo\-gisch vom Ich die
Rede sein kann.

Das Ich tritt in die Philosophie dadurch ein,
dass die \glqq{}Welt meine Welt ist\grqq{}.

{\verystretchyspace
Das philosophische Ich ist nicht der Mensch,
nicht der menschliche K&ouml;rper, oder die menschliche
Seele, von der die Psychologie handelt, sondern das
metaphysische Subjekt, die Grenze---nicht ein Teil
der Welt.}}


\PropositionG{6}
{Die allgemeine Form der Wahrheitsfunktion ist:
$[\overline{p}, \overline{\xi}, N(\overline{\xi})]$.

Dies ist die allgemeine Form des Satzes.}


\PropositionG{6.001}
{Dies sagt nichts anderes, als dass jeder Satz ein
Resultat der successiven Anwendung der Operation
$N'(\overline{\xi})$ auf die Elementars&auml;tze ist.}


\PropositionG{6.002}
{Ist die allgemeine Form gegeben, wie ein Satz
gebaut ist, so ist damit auch schon die allgemeine
Form davon gegeben, wie aus einem Satz durch
eine Operation ein anderer erzeugt werden
kann.}


\PropositionG{6.01}
{Die allgemeine Form der Operation $\Omega'(\overline{\eta})$ ist
also: $[\overline{\xi}, N(\overline{\xi})]'${}$(\overline{\eta})$ (=~[$\overline{\eta}$, $\overline{\xi}$, $N(\overline{\xi})$]).

Das ist die allgemeinste Form des &uuml;berganges
von einem Satz zum anderen.}
% -----File: 154.png---


\PropositionG{6.02}
{Und so kommen wir zu den Zahlen: Ich definiere <br/>
$x = \Omega^{0}{}' x \text{ Def. }$ <br/>
und <br/>
$\Omega'\Omega^{\nu}{}'x = \Omega^{\nu+1}{}'x \text{ Def.}$ <br/>
Nach diesen Zeichenregeln schreiben wir also
die Reihe <br/>
$x$, $\Omega'x$, $\Omega'\Omega'x$, $\Omega'\Omega'\Omega'x\fivedots$ <br/>
so: <br/>
$\Omega^{0}{}'x$, $\Omega^{0+1}{}'x$, $\Omega^{0+1+1}{}'x$, $\Omega^{0+1+1+1}{}'x$ \fivedots <br/>
Also schreibe ich statt ``$[x, \xi, \Omega'\xi]$'', <br/>
``$[\Omega^{0}{}'x, \Omega^{\nu}{}'x, \Omega^{\nu+1}{}'x]\text{''.}$ <br/>
Und definiere: <br/>
$0 + 1 = 1\text{ Def.}$ </br>
$0 + 1 + 1 = 2\text{ Def.}$ </br>
$0 + 1 + 1 + 1 = 3\text{ Def.}$ </br>
(usf.)
}


\PropositionG{6.021}
{Die Zahl ist der Exponent einer Operation.}


\PropositionG{6.022}
{Der Zahlbegriff ist nichts anderes, als das
Gemeinsame aller Zahlen, die allgemeine Form
der Zahl.

Der Zahlbegriff ist die variable Zahl.

Und der Begriff der Zahlengleichheit ist die
allgemeine Form aller speziellen Zahlengleichheiten.}


\PropositionG{6.03}
{Die allgemeine Form der ganzen Zahl ist:
$[0, \xi, \xi + 1]$.}


\PropositionG{6.031}
{Die Theorie der Klassen ist in der Mathematik
ganz &uuml;berfl&uuml;ssig.

Dies h&auml;ngt damit zusammen, dass die Allgemeinheit,
welche wir in der Mathematik brauchen,
nicht die \Emph{zuf&auml;llige} ist.}


\PropositionG{6.1}
{Die S&auml;tze der Logik sind Tautologien.}


\PropositionG{6.11}
{Die S&auml;tze der Logik sagen also Nichts. (Sie
sind die analytischen S&auml;tze.)}


\PropositionG{6.111}
{Theorien, die einen Satz der Logik gehaltvoll
erscheinen lassen, sind immer falsch. Man k&ouml;nnte
\zumBeispiel\ glauben, dass die Worte \glqq{}wahr\grqq{} und \glqq{}falsch\grqq{}
zwei Eigenschaften unter anderen Eigenschaften
bezeichnen, und da erschiene es als eine merkw&uuml;rdige
% -----File: 156.png---
Tatsache, dass jeder Satz eine dieser
Eigenschaften besitzt. Das scheint nun nichts
weniger als selbstverst&auml;ndlich zu sein, ebensowenig
selbstverst&auml;ndlich, wie etwa der Satz, \glqq{}alle Rosen
sind entweder gelb oder rot\grqq{} kl&auml;nge, auch wenn er
wahr w&auml;re. Ja, jener Satz bekommt nun ganz
den Charakter eines naturwissenschaftlichen Satzes
und dies ist das sichere Anzeichen daf&uuml;r, dass er
falsch aufgefasst wurde.}


\PropositionG{6.112}
{Die richtige Erkl&auml;rung der logischen S&auml;tze
muss ihnen eine einzigartige Stellung unter allen
S&auml;tzen geben.}


\PropositionG{6.113}
{Es ist das besondere Merkmal der logischen
S&auml;tze, dass man am Symbol allein erkennen kann,
dass sie wahr sind, und diese Tatsache schliesst
die ganze Philosophie der Logik in sich. Und
so ist es auch eine der wichtigsten Tatsachen, dass
sich die Wahrheit oder Falschheit der nicht-logischen
S&auml;tze \Emph{nicht} am Satz allein erkennen
l&auml;sst.}


\PropositionG{6.12}
{Dass die S&auml;tze der Logik Tautologien sind,
das \Emph{zeigt} die for\-ma\-len---lo\-gi\-schen---Ei\-gen\-schaf\-ten
der Sprache, der Welt.

Dass ihre Bestandteile \Emph{so} verkn&uuml;pft eine Tautologie
ergeben, das charakterisiert die Logik ihrer
Bestandteile.

Damit S&auml;tze, auf bestimmte Art und Weise
verkn&uuml;pft, eine Tautologie ergeben, dazu m&uuml;ssen
sie bestimmte Eigenschaften der Struktur haben.
Dass sie \Emph{so} verbunden eine Tautologie ergeben,
zeigt also, dass sie diese Eigenschaften der Struktur
besitzen.}


\PropositionG{6.1201}
{Dass \zumBeispiel\ die S&auml;tze \glqq{}$p$\grqq{} und \glqq{}$\Not{p}$\grqq{} in der
Verbindung \glqq{}$\Not{(p \DotOp \Not{p})}$\grqq{} eine Tautologie ergeben,
zeigt, dass sie einander widersprechen. Dass
die S&auml;tze \glqq{}$p \Implies q$\grqq{}, \glqq{}$p$\grqq{} und \glqq{}$q$\grqq{} in der Form
\glqq{}$(p \Implies q) \DotOp (p) : \Implies : (q)$\grqq{} miteinander verbunden eine
Tautologie ergeben, zeigt, dass $q$ aus $p$ und $p \Implies q$
% -----File: 158.png---
folgt. Dass \glqq{}$(x) \DotOp fx : \Implies : fa$\grqq{} eine Tautologie ist,
dass $fa$ aus $(x) \DotOp fx$ folgt.{} etc.\ etc.}


\PropositionG{6.1202}
{Es ist klar, dass man zu demselben Zweck statt
der Tautologien auch die Kontradiktionen verwenden
k&ouml;nnte.}


\PropositionG{6.1203}
{
Um eine Tautologie als solche zu erkennen,
kann man sich, in den F&auml;llen, in welchen in der
Tautologie keine Allgemeinheitsbezeichnung vorkommt,
folgender anschaulichen Methode bedienen:
Ich schreibe statt \glqq{}$p$\grqq{}, \glqq{}$q$\grqq{}, \glqq{}$r$\grqq{} etc.\ \glqq{}W$p$F\grqq{},
\glqq{}W$q$F\grqq{}, \glqq{}W$r$F\grqq{} etc. Die Wahrheitskombinationen
dr&uuml;cke ich durch Klammern aus.
\zumBeispiel:
\Illustration[0.35\textwidth]{brackets01-de}
und die Zuordnung der Wahr- oder Falschheit des
ganzen Satzes und der Wahrheitskombinationen
der Wahrheitsargumente durch Striche auf
folgende Weise:
\Illustration[0.4\textwidth]{brackets02-de}
Dies Zeichen w&uuml;rde also \zumBeispiel\ den Satz $p \Implies q$
darstellen. Nun will ich \zumBeispiel\ den Satz $\Not{(p \DotOp \Not{p})}$
(Gesetz des Widerspruchs) daraufhin untersuchen,
ob er eine Tautologie ist. Die Form \glqq{}$\Not{\xi}$\grqq{} wird
in unserer Notation
\Illustration[0.1\textwidth]{brackets03-de}
% -----File: 160.png---
geschrieben; die Form \glqq{}$\xi \DotOp \eta$\grqq{} so:
\Illustration[0.4\textwidth]{brackets04-de}
Daher lautet der Satz $\Not{(p \DotOp \Not{q})}$ so:
\Illustration{brackets05-de}
Setzen wir hier statt \glqq{}$q$\grqq{} \glqq{}$p$\grqq{} ein und untersuchen
die Verbindung der &auml;ussersten W und F mit den
innersten, so ergibt sich, dass die Wahrheit des
ganzen Satzes \Emph{allen} Wahrheitskombinationen
seines Argumentes, seine Falschheit keiner der
Wahrheitskombinationen zugeordnet ist.}


\PropositionG{6.121}
{Die S&auml;tze der Logik demonstrieren die logischen
Eigenschaften der S&auml;tze, indem sie sie zu nichtssagenden
S&auml;tzen verbinden.

Diese Methode k&ouml;nnte man auch eine Nullmethode
nennen. Im logischen Satz werden S&auml;tze
miteinander ins Gleichgewicht gebracht und der
Zustand des Gleichgewichts zeigt dann an, wie
diese S&auml;tze logisch beschaffen sein m&uuml;ssen.}


\PropositionG{6.122}
{Daraus ergibt sich, dass wir auch ohne die
logischen S&auml;tze auskommen k&ouml;nnen, da wir ja in
einer entsprechenden Notation die formalen Eigenschaften
der S&auml;tze durch das blosse Ansehen dieser
S&auml;tze erkennen k&ouml;nnen.}
% -----File: 162.png---


\PropositionG{6.1221}
{Ergeben \zumBeispiel\ zwei S&auml;tze \glqq{}$p$\grqq{} und \glqq{}$q$\grqq{} in der
Verbindung \glqq{}$p \Implies q$\grqq{} eine Tautologie, so ist \DPtypo{kar}{klar},
dass $q$ aus $p$ folgt.

Dass \zumBeispiel\ \glqq{}$q$\grqq{} aus \glqq{}$p \Implies q \DotOp p$\grqq{} folgt, ersehen wir
aus diesen beiden S&auml;tzen selbst, aber wir k&ouml;nnen
es auch \Emph{so} zeigen, indem wir sie zu \glqq{}$p \Implies q \DotOp p : \Implies : q$\grqq{}
verbinden und nun zeigen, dass dies eine Tautologie
ist.}


\PropositionG{6.1222}
{Dies wirft ein Licht auf die Frage, warum die
logischen S&auml;tze nicht durch die Erfahrung best&auml;tigt
werden k&ouml;nnen, ebenso wenig, wie sie durch die
Erfahrung widerlegt werden k&ouml;nnen. Nicht nur
muss ein Satz der Logik durch keine m&ouml;gliche Erfahrung
widerlegt werden k&ouml;nnen, sondern er darf auch
nicht durch eine solche best&auml;tigt werden k&ouml;nnen.}


\PropositionG{6.1223}
{Nun wird klar, warum man oft f&uuml;hlte, als w&auml;ren
die \glqq{}logischen Wahrheiten\grqq{} von uns zu \glqq{}\Emph{fordern}\grqq{}:
Wir k&ouml;nnen sie n&auml;mlich insofern fordern, als wir
eine gen&uuml;gende Notation fordern k&ouml;nnen.}


\PropositionG{6.1224}
{Es wird jetzt auch klar, warum die Logik die
Lehre von den Formen und vom Schliessen genannt
wurde.}


\PropositionG{6.123}
{Es ist klar: Die logischen Gesetze d&uuml;rfen nicht
selbst wieder logischen Gesetzen unterstehen.

(Es gibt nicht, wie Russell meinte, f&uuml;r jede
\glqq{}Type\grqq{} ein eigenes Gesetz des Widerspruches,
sondern Eines gen&uuml;gt, da es auf sich selbst nicht
angewendet wird.)}


\PropositionG{6.1231}
{Das Anzeichen des logischen Satzes ist \Emph{nicht}
die Allgemeing&uuml;ltigkeit.

Allgemein sein, heisst ja nur: Zuf&auml;lligerweise
f&uuml;r alle Dinge gelten. Ein unverallgemeinerter
Satz kann ja ebensowohl tautologisch sein, als ein
verallgemeinerter.}


\PropositionG{6.1232}
{Die logische Allgemeing&uuml;ltigkeit k&ouml;nnte man
wesentlich nennen, im Gegensatz zu jener zuf&auml;lligen,
etwa des Satzes \glqq{}alle Menschen sind sterblich\grqq{}.
S&auml;tze, wie Russells \glqq{}Axiom of reducibility\grqq{} sind
% -----File: 164.png---
nicht logische S&auml;tze, und dies erkl&auml;rt unser Gef&uuml;hl:
Dass sie, wenn wahr, so doch nur durch einen
g&uuml;nstigen Zufall wahr sein k&ouml;nnten.}


\PropositionG{6.1233}
{Es l&auml;sst sich eine Welt denken, in der das
Axiom of reducibility nicht gilt. Es ist aber klar,
dass die Logik nichts mit der Frage zu schaffen
hat, ob unsere Welt wirklich so ist oder nicht.}


\PropositionG{6.124}
{Die logischen S&auml;tze beschreiben das Ger&uuml;st der
Welt, oder vielmehr, sie stellen es dar. Sie
\glqq{}handeln\grqq{} von nichts. Sie setzen voraus, dass
Namen Bedeutung, und Elementars&auml;tze Sinn
haben: Und dies ist ihre Verbindung mit der
Welt. Es ist klar, dass es etwas &uuml;ber die Welt
anzeigen muss, dass gewisse Verbindungen von
Symbolen---welche wesentlich einen bestimmten
Charakter haben---Tautologien sind. Hierin liegt
das Entscheidende. Wir sagten, manches an
den Symbolen, die wir gebrauchen, w&auml;re willk&uuml;rlich,
manches nicht. In der Logik dr&uuml;ckt nur
dieses aus: Dass heisst aber, in der Logik dr&uuml;cken
nicht \Emph{wir} mit Hilfe der Zeichen aus, was wir
wollen, sondern in der Logik sagt die Natur der
naturnotwendigen Zeichen selbst aus: Wenn wir die
logische Syntax irgend einer Zeichensprache kennen,
dann sind bereits alle S&auml;tze der Logik gegeben.}


\PropositionG{6.125}
{Es ist m&ouml;glich, und zwar auch nach der alten
Auffassung der Logik, von vornherein eine Beschreibung
aller \glqq{}wahren\grqq{} logischen S&auml;tze zu geben.}


\PropositionG{6.1251}
{Darum kann es in der Logik auch \Emph{nie} &uuml;berraschungen
geben.}


\PropositionG{6.126}
{Ob ein Satz der Logik angeh&ouml;rt, kann man
berechnen, indem man die logischen Eigenschaften
des \Emph{Symbols} berechnet.

Und dies tun wir, wenn wir einen logischen
Satz \glqq{}beweisen\grqq{}. Denn, ohne uns um einen Sinn
und eine Bedeutung zu k&uuml;mmern, bilden wir den
logischen Satz aus anderen nach blossen \Emph{Zeichenregeln}.
% -----File: 166.png---

Der Beweis der logischen S&auml;tze besteht darin,
dass wir sie aus anderen logischen S&auml;tzen durch
successive Anwendung gewisser Operationen entstehen
lassen, die aus den ersten immer wieder
Tautologien erzeugen. (Und zwar \Emph{folgen} aus
einer Tautologie nur Tautologien.)

Nat&uuml;rlich ist diese Art zu zeigen, dass ihre
S&auml;tze Tautologien sind, der Logik durchaus unwesentlich.
Schon darum, weil die S&auml;tze, von
welchen der Beweis ausgeht, ja ohne Beweis zeigen
m&uuml;ssen, dass sie Tautologien sind.}


\PropositionG{6.1261}
{In der Logik sind Prozess und Resultat &auml;quivalent.
(Darum keine &uuml;berraschung.)}


\PropositionG{6.1262}
{Der Beweis in der Logik ist nur ein mechanisches
Hilfsmittel zum leichteren Erkennen der
Tautologie, wo sie kompliziert ist.}


\PropositionG{6.1263}
{Es w&auml;re ja auch zu merkw&uuml;rdig, wenn man
einen sinnvollen Satz \Emph{logisch} aus anderen beweisen
k&ouml;nnte, und einen logischen Satz \Emph{auch}.
Es ist von vornherein klar, dass der logische
Beweis eines sinnvollen Satzes und der Beweis \Emph{in}
der Logik zwei ganz verschiedene Dinge sein
m&uuml;ssen.}


\PropositionG{6.1264}
{Der sinnvolle Satz sagt etwas aus, und sein
Beweis zeigt, dass es so ist; in der Logik ist jeder
Satz die Form eines Beweises.

Jeder Satz der Logik ist ein in Zeichen dargestellter
modus ponens. (Und den modus ponens
kann man nicht durch einen Satz ausdr&uuml;cken.)}


\PropositionG{6.1265}
{Immer kann man die Logik so auffassen, dass
jeder Satz sein eigener Beweis ist.}


\PropositionG{6.127}
{Alle S&auml;tze der Logik sind gleichberechtigt, es
gibt unter ihnen nicht wesentlich Grundgesetze
und abgeleitete S&auml;tze.

Jede Tautologie zeigt selbst, dass sie eine
Tautologie ist.}


\PropositionG{6.1271}
{Es ist klar, dass die Anzahl der \glqq{}logischen
Grundgesetze\grqq{} willk&uuml;rlich ist, denn man k&ouml;nnte
% -----File: 168.png---
die Logik ja aus Einem Grundgesetz ableiten,
indem man einfach \zumBeispiel\ aus Freges Grundgesetzen
das logische Produkt bildet. (Frege w&uuml;rde
vielleicht sagen, dass dieses Grundgesetz nun
nicht mehr unmittelbar einleuchte. Aber es ist
merkw&uuml;rdig, dass ein so exakter Denker wie
Frege sich auf den Grad des Einleuchtens als
Kriterium des logischen Satzes berufen hat.)}


\PropositionG{6.13}
{Die Logik ist keine Lehre, sondern ein Spiegelbild
der Welt.

Die Logik ist transcendental.}


\PropositionG{6.2}
{Die Mathematik ist eine logische Methode.

Die S&auml;tze der Mathematik sind Gleichungen
also Scheins&auml;tze.}


\PropositionG{6.21}
{Der Satz der Mathematik dr&uuml;ckt keinen Gedanken
aus.}


\PropositionG{6.211}
{Im Leben ist es ja nie der mathematische Satz,
den wir brauchen, sondern wir ben&uuml;tzen den
mathematischen Satz \Emph{nur}, um aus S&auml;tzen, welche
nicht der Mathematik angeh&ouml;ren, auf andere zu
schliessen, welche gleichfalls nicht der Mathematik
angeh&ouml;ren.

(In der Philosophie f&uuml;hrt die Frage \glqq{}wozu
gebrauchen wir eigentlich jenes Wort, jenen Satz\grqq{}
immer wieder zu wertvollen Einsichten.)}


\PropositionG{6.22}
{Die Logik der Welt, die die S&auml;tze der Logik in
den Tautologien zeigen, zeigt die Mathematik in
den Gleichungen.}


\PropositionG{6.23}
{Wenn zwei Ausdr&uuml;cke durch das Gleichheitszeichen
verbunden werden, so heisst das, sie sind
durch einander ersetzbar. Ob dies aber der Fall ist
muss sich an den beiden Ausdr&uuml;cken selbst zeigen.

Es charakterisiert die logische Form zweier Ausdr&uuml;cke,
dass sie durch einander ersetzbar sind.}


\PropositionG{6.231}
{Es ist eine Eigenschaft der Bejahung, dass man
sie als doppelte Verneinung auffassen kann.

Es ist eine Eigenschaft von \glqq{}$1 + 1 + 1 + 1$\grqq{}, dass
man es als \glqq{}$(1 + 1) + (1 + 1)$\grqq{} auffassen kann.}
% -----File: 170.png---


\PropositionG{6.232}
{Frege sagt, die beiden Ausdr&uuml;cke haben dieselbe
Bedeutung, aber verschiedenen Sinn.

Das Wesentliche an der Gleichung ist aber, dass
sie nicht notwendig ist, um zu zeigen, dass die beiden
Ausdr&uuml;cke, die das Gleichheitszeichen verbindet,
dieselbe Bedeutung haben, da sich dies aus den
beiden Ausdr&uuml;cken selbst ersehen l&auml;sst.}


\PropositionG{6.2321}
{Und, dass die S&auml;tze der Mathematik bewiesen
werden k&ouml;nnen, heisst ja nichts anderes, als dass
ihre Richtigkeit einzusehen ist, ohne dass das, was
sie ausdr&uuml;cken, selbst mit den Tatsachen auf seine
Richtigkeit hin verglichen werden muss.}


\PropositionG{6.2322}
{Die Identit&auml;t der Bedeutung zweier Ausdr&uuml;cke
l&auml;sst sich nicht \Emph{behaupten}. Denn um etwas von
ihrer Bedeutung behaupten zu k&ouml;nnen, muss ich
ihre Bedeutung kennen: und indem ich ihre Bedeutung
kenne, weiss ich, ob sie dasselbe oder
verschiedenes bedeuten.}


\PropositionG{6.2323}
{Die Gleichung kennzeichnet nur den Standpunkt,
von welchem ich die beiden Ausdr&uuml;cke
betrachte, n&auml;mlich vom Standpunkte ihrer Bedeutungsgleichheit.}


\PropositionG{6.233}
{Die Frage, ob man zur L&ouml;sung der mathematischen
Probleme die Anschauung brauche, muss
dahin beantwortet werden, dass eben die Sprache
hier die n&ouml;tige Anschauung liefert.}


\PropositionG{6.2331}
{Der Vorgang des \Emph{Rechnens} vermittelt eben
diese Anschauung.

Die Rechnung ist kein Experiment.}


\PropositionG{6.234}
{Die Mathematik ist eine Methode der Logik.}


\PropositionG{6.2341}
{Das Wesentliche der mathematischen Methode
ist es, mit Gleichungen zu arbeiten. Auf dieser
Methode beruht es n&auml;mlich, dass jeder Satz der
Mathematik sich von selbst verstehen muss.}


\PropositionG{6.24}
{Die Methode der Mathematik, zu ihren Gleichungen
zu kommen, ist die Substitutionsmethode.

{\verystretchyspace
Denn die Gleichungen dr&uuml;cken die Ersetzbarkeit
zweier Ausdr&uuml;cke aus und wir schreiten von einer
% -----File: 172.png---
Anzahl von Gleichungen zu neuen Gleichungen
vor, indem wir, den Gleichungen entsprechend,
Ausdr&uuml;cke durch andere ersetzen.}}


\PropositionG{6.241}
{So lautet der Beweis des Satzes $2 \times 2 = 4$: <br/>
$(\Omega^{\nu})^{\mu}{}'x = \Omega^{\nu \times \mu}{}'x \text{ Def.} $<br/>
$\Omega^{2 \times 2}{}'x = (\Omega^{2})^{2}{}'x = (\Omega^{2})^{1 + 1}{}'x = \Omega^{2}{}'\Omega^{2}{}'x = \Omega^{1 + 1}{}'\Omega^{1 + 1}{}'x$ <br/>
$= (\Omega'\Omega)'(\Omega'\Omega)'x = \Omega'\Omega'\Omega'\Omega'x = \Omega^{1 + 1 + 1 + 1}{}'x = \Omega^{4}{}'x.$
}


\PropositionG{6.3}
{Die Erforschung der Logik bedeutet die Erforschung
\Emph{aller Gesetzm&auml;ssigkeit}. Und ausserhalb
der Logik ist alles Zufall.}


\PropositionG{6.31}
{Das sogenannte Gesetz der Induktion kann
jedenfalls kein logisches Gesetz sein, denn es ist
offenbar ein sinnvoller Satz.---Und darum kann es
auch kein Gesetz a priori sein.}


\PropositionG{6.32}
{Das Kausalit&auml;tsgesetz ist kein Gesetz, sondern
die Form eines Gesetzes.}


\PropositionG{6.321}
{\glqq{}Kausalit&auml;tsgesetz\grqq{}, das ist ein Gattungsname.
Und wie es in der Mechanik, sagen wir, Minimum-Gesetze
gibt,---etwa der kleinsten Wir\-kung---so
gibt es in der Physik Kausalit&auml;tsgesetze, Gesetze
von der Kausalit&auml;tsform.}


\PropositionG{6.3211}
{Man hat ja auch davon eine Ahnung gehabt, dass
es \Emph{ein} \glqq{}Gesetz der kleinsten Wirkung\grqq{} geben m&uuml;sse,
ehe man genau wuss\-te, wie es lautete. (Hier, wie
immer, stellt sich das a priori Gewisse als etwas
rein Logisches heraus.)}


\PropositionG{6.33}
{Wir \Emph{glauben} nicht a priori an ein Erhaltungsgesetz,
sondern wir \Emph{wissen} a priori die
M&ouml;glichkeit einer logischen Form.}


\PropositionG{6.34}
{Alle jene S&auml;tze, wie der Satz vom Grunde, von
der Kontinuit&auml;t in der Natur, vom kleinsten Aufwande
in der Natur etc.\ etc., alle diese sind Einsichten
a priori &uuml;ber die m&ouml;gliche Formgebung der
S&auml;tze der Wissenschaft.}


\PropositionG{6.341}
{Die Newtonsche Mechanik \zumBeispiel\ bringt die Weltbeschreibung
auf eine einheitliche Form. Denken
% -----File: 174.png---
wir uns eine weisse Fl&auml;che, auf der unregelm&auml;ssige
schwarze Flecken w&auml;ren. Wir sagen nun: Was f&uuml;r
ein Bild immer hierdurch entsteht, immer kann ich
seiner Beschreibung beliebig nahe kommen, indem
ich die Fl&auml;che mit einem entsprechend feinen quadratischen
Netzwerk bedecke und nun von jedem
Quadrat sage, dass es weiss oder schwarz ist. Ich
werde auf diese Weise die Beschreibung der Fl&auml;che
auf eine einheitliche Form gebracht haben. Diese
Form ist beliebig, denn ich h&auml;tte mit dem gleichen
Erfolge ein Netz aus dreieckigen oder sechseckigen
Maschen verwenden k&ouml;nnen. Es kann sein, dass
die Beschreibung mit Hilfe eines Dreiecks-Netzes
einfacher geworden w&auml;re; das heisst, dass wir die
Fl&auml;che mit einem gr&ouml;beren Dreiecks-Netz genauer
beschreiben k&ouml;nnten, als mit einem feineren quadratischen
(oder umgekehrt) usw. Den verschiedenen
Netzen entsprechen verschiedene Systeme der
Weltbeschreibung. Die Mechanik bestimmt eine
Form der Weltbeschreibung, indem sie sagt:
Alle S&auml;tze der Weltbeschreibung m&uuml;ssen aus einer
Anzahl gegebener S&auml;tze---den mechanischen Axiomen---auf
eine gegebene Art und Weise erhalten
werden. Hierdurch liefert sie die Bausteine zum
Bau des wissenschaftlichen Geb&auml;udes und sagt:
Welches Geb&auml;ude immer du auff&uuml;hren willst, jedes
musst du irgendwie mit diesen und nur diesen
Bausteinen zusammenbringen.

(Wie man mit dem Zahlensystem jede beliebige
Anzahl, so muss man mit dem System der
Mechanik jeden beliebigen Satz der Physik
hinschreiben k&ouml;nnen.)}


\PropositionG{6.342}
{Und nun sehen wir die gegenseitige Stellung
von Logik und Mechanik. (Man k&ouml;nnte das Netz
auch aus verschiedenartigen Figuren etwa aus
Dreiecken und Sechsecken bestehen lassen.) Dass
sich ein Bild, wie das vorhin erw&auml;hnte, durch ein
Netz von gegebener Form beschreiben l&auml;sst, sagt
% -----File: 176.png---
&uuml;ber das Bild \Emph{nichts} aus. (Denn dies gilt f&uuml;r
jedes Bild dieser Art.) \Emph{Das} aber charakterisiert
das Bild, dass es sich durch ein bestimmtes Netz
von \Emph{bestimmter} Feinheit \Emph{vollst&auml;ndig} beschreiben
l&auml;sst.

So auch sagt es nichts &uuml;ber die Welt aus, dass
sie sich durch die Newtonsche Mechanik beschreiben
l&auml;sst; wohl aber, dass sie sich \Emph{so} durch
jene beschreiben l&auml;sst, wie dies eben der Fall ist.
Auch das sagt etwas &uuml;ber die Welt, dass sie sich
durch die eine Mechanik einfacher beschreiben
l&auml;sst, als durch die andere.}


\PropositionG{6.343}
{Die Mechanik ist ein Versuch, alle \Emph{wahren}
S&auml;tze, die wir zur Weltbeschreibung brauchen,
nach Einem Plane zu konstruieren.}


\PropositionG{6.3431}
{Durch den ganzen logischen Apparat hindurch
sprechen die physikalischen Gesetze doch von den
Gegenst&auml;nden der Welt.}


\PropositionG{6.3432}
{Wir d&uuml;rfen nicht vergessen, dass die Weltbeschreibung
durch die Mechanik immer die ganz
allgemeine ist. Es ist in ihr \zumBeispiel\ nie von
\Emph{bestimmten} materiellen Punkten die Rede,
sondern immer nur von \Emph{irgend welchen}.}


\PropositionG{6.35}
{Obwohl die Flecke in unserem Bild geometrische
Figuren sind, so kann doch selbstverst&auml;ndlich
die Geometrie gar nichts &uuml;ber ihre
tats&auml;chliche Form und Lage sagen. Das Netz
aber ist \Emph{rein} geometrisch, alle seine Eigenschaften
k&ouml;nnen a priori angegeben werden.

Gesetze, wie der Satz vom Grunde, etc., handeln
vom Netz, nicht von dem, was das Netz beschreibt.}


\PropositionG{6.36}
{Wenn es ein Kausalit&auml;tsgesetz g&auml;be, so k&ouml;nnte
es lauten: \glqq{}Es gibt Naturgesetze\grqq{}.

Aber freilich kann man das nicht sagen: es
zeigt sich.}


\PropositionG{6.361}
{In der Ausdrucksweise Hertz's k&ouml;nnte man
sagen: Nur \Emph{gesetzm&auml;ssige} Zusammenh&auml;nge
sind \Emph{denkbar}.}
% -----File: 178.png---


\PropositionG{6.3611}
{Wir k&ouml;nnen keinen Vorgang mit dem \glqq{}Ablauf
der Zeit\grqq{} ver\-glei\-chen---diesen gibt es nicht---,
sondern nur mit einem anderen Vorgang (etwa
mit dem Gang des Chronometers).

Daher ist die Beschreibung des zeitlichen
Verlaufs nur so m&ouml;glich, dass wir uns auf einen
anderen Vorgang st&uuml;tzen.

Ganz Analoges gilt f&uuml;r den Raum. Wo man
\zumBeispiel\ sagt, es k&ouml;nne keines von zwei Ereignissen
(die sich gegenseitig aus\-schlies\-sen) eintreten, weil
\Emph{keine Ursache} vorhanden sei, warum das eine
eher als das andere eintreten solle, da handelt es
sich in Wirklichkeit darum, dass man gar nicht
\Emph{eines} der beiden Ereignisse beschreiben kann,
wenn nicht irgend eine Asymmetrie vorhanden ist.
Und \Emph{wenn} eine solche Asymmetrie vorhanden \Emph{ist},
so k&ouml;nnen wir diese als \Emph{Ursache} des Eintreffens
des einen und Nicht-Eintreffens des anderen
auffassen.}


\PropositionG{6.36111}
{Das Kant'sche Problem von der rechten und
linken Hand, die man nicht zur Deckung bringen
kann, besteht schon in der Ebene, ja im eindimensionalen
Raum, wo die beiden kongruenten
Figuren $a$ und $b$ auch nicht zur Deckung gebracht
werden k&ouml;nnen, ohne aus diesem Raum
herausbewegt zu werden. Rechte und linke Hand
sind tats&auml;chlich vollkommen kongruent. Und
dass man sie nicht zur Deckung bringen kann,
hat damit nichts zu tun.

\Illustration[0.45\textwidth]{space}

Den rechten Handschuh k&ouml;nnte man an die
linke Hand ziehen, wenn man ihn im vierdimensionalen
Raum umdrehen k&ouml;nnte.}


\PropositionG{6.362}
{Was sich beschreiben l&auml;sst, das kann auch
geschehen, und was das Kausalit&auml;tsgesetz ausschliessen
soll, das l&auml;sst sich auch nicht beschreiben.}


\PropositionG{6.363}
{Der Vorgang der Induktion besteht darin, dass
% -----File: 180.png---
wir das \Emph{einfachste} Gesetz annehmen, das mit
unseren Erfahrungen in Einklang zu bringen ist.}


\PropositionG{6.3631}
{Dieser Vorgang hat aber keine logische, sondern
nur eine psychologische Begr&uuml;ndung.

Es ist klar, dass kein Grund vorhanden ist, zu
glauben, es werde nun auch wirklich der einfachste
Fall eintreten.}


\PropositionG{6.36311}
{Dass die Sonne morgen aufgehen wird, ist eine
Hypothese; und das heisst: wir \Emph{wissen} nicht, ob
sie aufgehen wird.}


\PropositionG{6.37}
{Einen Zwang, nach dem Eines geschehen m&uuml;sste,
weil etwas anderes geschehen ist, gibt es nicht. Es
gibt nur eine \Emph{logische} Notwendigkeit.}


\PropositionG{6.371}
{Der ganzen modernen Weltanschauung liegt die
T&auml;uschung zugrunde, dass die sogenannten Naturgesetze
die Erkl&auml;rungen der Naturerscheinungen
seien.}


\PropositionG{6.372}
{So bleiben sie bei den Naturgesetzen als bei
etwas Unantastbarem stehen, wie die &auml;lteren bei
Gott und dem Schicksal.

Und sie haben ja beide Recht, und Unrecht. Die
Alten sind allerdings insofern klarer, als sie einen
klaren Abschluss anerkennen, w&auml;hrend es bei dem
neuen System scheinen soll, als sei \Emph{alles} erkl&auml;rt.}


\PropositionG{6.373}
{Die Welt ist unabh&auml;ngig von meinem Willen.}


\PropositionG{6.374}
{Auch wenn alles, was wir w&uuml;nschen, gesch&auml;he,
so w&auml;re dies doch nur, sozusagen, eine Gnade des
Schicksals, denn es ist kein \Emph{logischer} Zusammenhang
zwischen Willen und Welt, der dies
verb&uuml;rgte, und den angenommenen physikalischen
Zusammenhang k&ouml;nnten wir doch nicht selbst
wieder wollen.}


\PropositionG{6.375}
{Wie es nur eine \Emph{logische} Notwendigkeit gibt,
so gibt es auch nur eine \Emph{logische} Unm&ouml;glichkeit.}


\PropositionG{6.3751}
{Dass \zumBeispiel\ zwei Farben zugleich an einem Ort
des Gesichtsfeldes sind, ist unm&ouml;glich und zwar
logisch unm&ouml;glich, denn es ist durch die logische
Struktur der Farbe ausgeschlossen.
% -----File: 182.png---

Denken wir daran, wie sich dieser Widerspruch
in der Physik darstellt: Ungef&auml;hr so, dass ein
Teilchen nicht zu gleicher Zeit zwei Geschwindigkeiten
haben kann; das heisst, dass es nicht zu
gleicher Zeit an zwei Orten sein kann; das heisst,
dass Teilchen an verschiedenen Orten zu Einer Zeit
nicht identisch sein k&ouml;nnen.

(Es ist klar, dass das logische Produkt zweier
Elementars&auml;tze weder eine Tautologie noch eine
Kontradiktion sein kann. Die Aussage, dass ein
Punkt des Gesichtsfeldes zu gleicher Zeit zwei
verschiedene Farben hat, ist eine Kontradiktion.)}


\PropositionG{6.4}
{Alle S&auml;tze sind gleichwertig.}


\PropositionG{6.41}
{Der Sinn der Welt muss ausserhalb ihrer liegen.
In der Welt ist alles wie es ist und geschieht alles
wie es geschieht; es gibt \Emph{in} ihr keinen Wert---und
wenn es ihn g&auml;be, so h&auml;tte er keinen Wert.

Wenn es einen Wert gibt, der Wert hat, so muss
er ausserhalb alles Geschehens und So-Seins liegen.
Denn alles Geschehen und So-Sein ist zuf&auml;llig.

Was es nicht-zuf&auml;llig macht, kann nicht \Emph{in} der
Welt liegen, denn sonst w&auml;re dies wieder zuf&auml;llig.

Es muss ausserhalb der Welt liegen.}


\PropositionG{6.42}
{Darum kann es auch keine S&auml;tze der Ethik geben.

S&auml;tze k&ouml;nnen nichts H&ouml;heres ausdr&uuml;cken.}


\PropositionG{6.421}
{Es ist klar, dass sich die Ethik nicht aussprechen
l&auml;sst.

Die Ethik ist \DPtypo{transscendental}{transcendental}.

(Ethik und Aesthetik sind Eins.)}


\PropositionG{6.422}
{Der erste Gedanke bei der Aufstellung eines
ethischen Gesetzes von der Form \glqq{}du sollst $\fourdots$\grqq{}
ist: Und was dann, wenn ich es nicht tue? Es ist
aber klar, dass die Ethik nichts mit Strafe und
Lohn im gew&ouml;hnlichen Sinne zu tun hat. Also
muss diese Frage nach den \Emph{Folgen} einer Handlung
belanglos sein.---Zum Mindesten d&uuml;rfen diese
Folgen nicht Ereignisse sein. Denn etwas muss
doch an jener Fragestellung richtig sein. Es muss
% -----File: 184.png---
zwar eine Art von ethischem Lohn und ethischer
Strafe geben, aber diese m&uuml;ssen in der Handlung
selbst liegen.

(Und das ist auch klar, dass der Lohn etwas
Angenehmes, die Strafe etwas Unangenehmes sein
muss.)}


\PropositionG{6.423}
{Vom Willen als dem Tr&auml;ger des Ethischen kann
nicht gesprochen werden.

Und der Wille als Ph&auml;nomen interessiert nur
die Psychologie.}


\PropositionG{6.43}
{Wenn das gute oder b&ouml;se Wollen die Welt
&auml;ndert, so kann es nur die Grenzen der Welt &auml;ndern,
nicht die Tatsachen; nicht das, was durch die
Sprache ausgedr&uuml;ckt werden kann.

Kurz, die Welt muss dann dadurch &uuml;berhaupt
eine andere werden. Sie muss sozusagen als
Ganzes abnehmen oder zunehmen.

Die Welt des Gl&uuml;cklichen ist eine andere als die
des Ungl&uuml;cklichen.}


\PropositionG{6.431}
{Wie auch beim Tod die Welt sich nicht &auml;ndert,
sondern aufh&ouml;rt.}


\PropositionG{6.4311}
{Der Tod ist kein Ereignis des Lebens. Den
Tod erlebt man nicht.

Wenn man unter Ewigkeit nicht unendliche
Zeitdauer, sondern Unzeitlichkeit versteht, dann
lebt der ewig, der in der Gegenwart lebt.

Unser Leben ist ebenso endlos, wie unser
Gesichtsfeld grenzenlos ist.}


\PropositionG{6.4312}
{Die zeitliche Unsterblichkeit der Seele des
Menschen, das heisst also ihr ewiges Fortleben
auch nach dem Tode, ist nicht nur auf keine Weise
verb&uuml;rgt, sondern vor allem leistet diese Annahme
gar nicht das, was man immer mit ihr erreichen
wollte. Wird denn dadurch ein R&auml;tsel gel&ouml;st, dass
ich ewig fortlebe? Ist denn dieses ewige Leben
dann nicht ebenso r&auml;tselhaft wie das gegenw&auml;rtige?
Die L&ouml;sung des R&auml;tsels des Lebens in Raum und
Zeit liegt \Emph{ausserhalb} von Raum und Zeit.
% -----File: 186.png---

(Nicht Probleme der Naturwissenschaft sind ja
zu l&ouml;sen.)}


\PropositionG{6.432}
{\Emph{Wie} die Welt ist, ist f&uuml;r das H&ouml;here vollkommen
\enlargethispage{1pt} % enlarge to make the last line fit
gleichg&uuml;ltig. Gott offenbart sich nicht \Emph{in}
der Welt.}


\PropositionG{6.4321}
{Die Tatsachen geh&ouml;ren alle nur zur Aufgabe,
nicht zur L&ouml;sung.}


\PropositionG{6.44}
{Nicht \Emph{wie} die Welt ist, ist das Mystische,
sondern \Emph{dass} sie ist.}


\PropositionG{6.45}
{Die Anschauung der Welt sub specie aeterni
ist ihre Anschauung als---be\-grenz\-tes---Gan\-zes.

Das Gef&uuml;hl der Welt als begrenztes Ganzes ist
das mystische.}


\PropositionG{6.5}
{Zu einer Antwort, die man nicht aussprechen
kann, kann man auch die Frage nicht aussprechen.

\Emph{Das R&auml;tsel} gibt es nicht.

Wenn sich eine Frage &uuml;berhaupt stellen l&auml;sst,
so \Emph{kann} sie auch beantwortet werden.}


\PropositionG{6.51}
{Skeptizismus ist \Emph{nicht} unwiderleglich, sondern
offenbar unsinnig, wenn er bezweifeln will, wo
nicht gefragt werden kann.

Denn Zweifel kann nur bestehen, wo eine Frage
besteht; eine Frage nur, wo eine Antwort besteht,
und diese nur, wo etwas \Emph{gesagt} werden \Emph{kann}.}


\PropositionG{6.52}
{Wir f&uuml;hlen, dass selbst, wenn alle \Emph{m&ouml;glichen}
wissenschaftlichen Fragen beantwortet sind, unsere
Lebensprobleme noch gar nicht ber&uuml;hrt sind.
Freilich bleibt dann eben keine Frage mehr; und
eben dies ist die Antwort.}


\PropositionG{6.521}
{Die L&ouml;sung des Problems des Lebens merkt
man am Verschwinden dieses Problems.

(Ist nicht dies der Grund, warum Menschen,
denen der Sinn des Lebens nach langen Zweifeln
klar wurde, warum diese dann nicht sagen konnten,
worin dieser Sinn bestand.)}


\PropositionG{6.522}
{Es gibt allerdings Unaussprechliches. Dies
\Emph{zeigt} sich, es ist das Mystische.}


\PropositionG{6.53}
{{\verystretchyspace
Die richtige Methode der Philosophie w&auml;re
% -----File: 188.png---
eigentlich die: Nichts zu sagen, als was sich sagen
l&auml;sst, also S&auml;tze der Na\-tur\-wis\-sen\-schaft---also etwas,
was mit Philosophie nichts zu tun hat---, und dann
immer, wenn ein anderer etwas \DPtypo{Methaphysisches}{Metaphysisches}
sagen wollte, ihm nachzuweisen, dass er gewissen
Zeichen in seinen S&auml;tzen keine Bedeutung gegeben
hat. Diese Methode w&auml;re f&uuml;r den anderen un\-be\-frie\-di\-gend---er
h&auml;tte nicht das Gef&uuml;hl, dass wir
ihn Philosophie lehrten---aber \Emph{sie} w&auml;re die einzig
streng richtige.}}


\PropositionG{6.54}
{Meine S&auml;tze erl&auml;utern dadurch, dass sie der,
welcher mich versteht, am Ende als unsinnig
erkennt, wenn er durch sie---auf ihnen---&uuml;ber sie
hinausgestiegen ist. (Er muss sozusagen die Leiter
wegwerfen, nachdem er auf ihr hinaufgestiegen ist.)

Er muss diese S&auml;tze &uuml;berwinden, dann sieht er
die Welt richtig.}


\PropositionG{7}
{Wovon man nicht sprechen kann, dar&uuml;ber muss
man schweigen.}
