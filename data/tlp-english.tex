\PropositionE{1}
{The world is everything that is the case.\footnote{The decimal figures as numbers of the separate propositions indicate the logical
importance of the propositions, the emphasis laid upon them in my exposition.
The propositions \textit{n}.1, \textit{n}.2, \textit{n}.3, etc., are comments on proposition No.\textit{n}; the propositions
\textit{n}.\textit{m}1, \textit{n}.\textit{m}2, etc., are comments on the proposition No.\textit{n}.\textit{m}; and so on.}}


\PropositionE{1.1}
{The world is the totality of facts, not of
things.}


\PropositionE{1.11}
{The world is determined by the facts, and by
these being \emph{all} the facts.}


\PropositionE{1.12}
{For the totality of facts determines both what is
the case, and also all that is not the case.}


\PropositionE{1.13}
{The facts in logical space are the world.}


\PropositionE{1.2}
{The world divides into facts.}


\PropositionE{1.21}
{Any one can either be the case or not be the
case, and everything else remain the same.}


\PropositionE{2}
{What is the case, the fact, is the existence of
atomic facts.}


\PropositionE{2.01}
{An atomic fact is a combination of objects
(entities, things).}


\PropositionE{2.011}
{It is essential to a thing that it can be a constituent
part of an atomic fact.}


\PropositionE{2.012}
{In logic nothing is accidental: if a thing \emph{can}
occur in an atomic fact the possibility of that
atomic fact must already be prejudged in the
thing.}


\PropositionE{2.0121}
{It would, so to speak, appear as an accident, when
to a thing that could exist alone on its own account,
subsequently a state of affairs could be made to fit.

If things can occur in atomic facts, this possibility
must already lie in them.

(A logical entity cannot be merely possible.
Logic treats of every possibility, and all possibilities
are its facts.)
% -----File: 033.png---

Just as we cannot think of spatial objects at
all apart from space, or temporal objects apart
from time, so we cannot think of \emph{any} object apart
from the possibility of its connexion with other
things.

If I can think of an object in the context of an
atomic fact, I cannot think of it apart from the
\emph{possibility} of this context.}


\PropositionE{2.0122}
{The thing is independent, in so far as it can
occur in all \emph{possible} circumstances, but this form
of independence is a form of connexion with the
atomic fact, a form of dependence. (It is impossible
for words to occur in two different ways,
alone and in the proposition.)}


\PropositionE{2.0123}
{If I know an object, then I also know all the
possibilities of its occurrence in atomic facts.

(Every such possibility must lie in the nature
of the object.)

A new possibility cannot subsequently be
found.}


\PropositionE{2.01231}
{In order to know an object, I must know not
its external but all its internal qualities.}


\PropositionE{2.0124}
{If all objects are given, then thereby are all
\emph{possible} atomic facts also given.}


\PropositionE{2.013}
{Every thing is, as it were, in a space of possible
atomic facts. I can think of this space as empty,
but not of the thing without the space.}


\PropositionE{2.0131}
{A spatial object must lie in infinite space.
(A point in space is a place for an argument.)

A speck in a visual field need not be red,
but it must have a colour; it has, so to speak,
a colour space round it. A tone must have \emph{a}
% -----File: 035.png---
pitch, the object of the sense of touch \emph{a} hardness,
etc.}


\PropositionE{2.014}
{Objects contain the possibility of all states of
affairs.}


\PropositionE{2.0141}
{The possibility of its occurrence in atomic facts
is the form of the object.}


\PropositionE{2.02}
{The object is simple.}


\PropositionE{2.0201}
{Every statement about complexes can be analysed
into a statement about their constituent parts, and
into those propositions which completely describe
the complexes.}


\PropositionE{2.021}
{Objects form the substance of the world.
Therefore they cannot be compound.}


\PropositionE{2.0211}
{If the world had no substance, then whether
a proposition had sense would depend on whether
another proposition was true.}


\PropositionE{2.0212}
{It would then be impossible to form a picture
of the world (true or false).}


\PropositionE{2.022}
{It is clear that however different from the real
one an imagined world may be, it must have something---a
form---in common with the real world.}


\PropositionE{2.023}
{This fixed form consists of the objects.}


\PropositionE{2.0231}
{The substance of the world \emph{can} only determine
a form and not any material properties. For these
are first presented by the propositions---first formed
by the configuration of the objects.}


\PropositionE{2.0232}
{Roughly speaking: objects are colourless.}


\PropositionE{2.0233}
{Two objects of the same logical form are---apart
from their external prop\-er\-ties---only differentiated
from one another in that they are
different.}


\PropositionE{2.02331}
{Either a thing has properties which no other
has, and then one can distinguish it straight away
from the others by a description and refer to it;
or, on the other hand, there are several things
which have the totality of their properties in
% -----File: 037.png---
common, and then it is quite impossible to point
to any one of them.

For if a thing is not distinguished by anything,
I cannot distinguish it---for otherwise it would be
distinguished.}


\PropositionE{2.024}
{Substance is what exists independently of what
is the case.}


\PropositionE{2.025}
{It is form and content.}


\PropositionE{2.0251}
{Space, time and colour (colouredness) are forms
of objects.}


\PropositionE{2.026}
{Only if there are objects can there be a fixed
form of the world.}


\PropositionE{2.027}
{The fixed, the existent and the object are
one.}


\PropositionE{2.0271}
{The object is the fixed, the existent; the configuration
is the changing, the variable.}


\PropositionE{2.0272}
{The configuration of the objects forms the
atomic fact.}


\PropositionE{2.03}
{In the atomic fact objects hang one in another,
like the members of a chain.}


\PropositionE{2.031}
{In the atomic fact the objects are combined in
a definite way.}


\PropositionE{2.032}
{The way in which objects hang together in
the atomic fact is the structure of the atomic
fact.}


\PropositionE{2.033}
{The form is the possibility of the structure.}


\PropositionE{2.034}
{The structure of the fact consists of the structures
of the atomic facts.}


\PropositionE{2.04}
{The totality of existent atomic facts is the
world.}


\PropositionE{2.05}
{The totality of existent atomic facts also determines
which atom\-ic facts do not exist.}


\PropositionE{2.06}
{The existence and non-existence of atomic facts
is the reality.

(The existence of atomic facts we also call
a positive fact, their non-existence a negative
fact.)}


\PropositionE{2.061}
{Atomic facts are independent of one another.}
% -----File: 039.png---


\PropositionE{2.062}
{From the existence or non-existence of an
atomic fact we cannot infer the existence or non-existence
of another.}


\PropositionE{2.063}
{The total reality is the world.}


\PropositionE{2.1}
{We make to ourselves pictures of facts.}


\PropositionE{2.11}
{The picture presents the facts in logical space,
the existence and non-ex\-is\-tence of atomic
facts.}


\PropositionE{2.12}
{The picture is a model of reality.}


\PropositionE{2.13}
{To the objects correspond in the picture the
elements of the picture.}


\PropositionE{2.131}
{The elements of the picture stand, in the picture,
for the objects.}


\PropositionE{2.14}
{The picture consists in the fact that its elements
are combined with one another in a definite way.}


\PropositionE{2.141}
{The picture is a fact.}


\PropositionE{2.15}
{That the elements of the picture are combined
with one another in a definite way, represents that
\enlargethispage{9pt} % enlarge to make the last line fit
the things are so combined with one another.

This connexion of the elements of the picture is
called its structure, and the possibility of this structure
is called the form of representation of the picture.}


\PropositionE{2.151}
{The form of representation is the possibility that
the things are combined with one another as are
the elements of the picture.}


\PropositionE{2.1511}
{Thus the picture is linked with reality; it reaches
up to it.}


\PropositionE{2.1512}
{It is like a scale applied to reality.}


\PropositionE{2.15121}
{Only the outermost points of the dividing lines
\emph{touch} the object to be measured.}


\PropositionE{2.1513}
{According to this view the representing relation
which makes it a picture, also belongs to the
picture.}


\PropositionE{2.1514}
{The representing relation consists of the co-ordinations
of the elements of the picture and the
things.}


\PropositionE{2.1515}
{These co-ordinations are as it were the feelers of
% -----File: 041.png---
its elements with which the picture touches
reality.}


\PropositionE{2.16}
{In order to be a picture a fact must have something
in common with what it pictures.}


\PropositionE{2.161}
{In the picture and the pictured there must be
something identical in order that the one can be a
picture of the other at all.}


\PropositionE{2.17}
{What the picture must have in common with
reality in order to be able to represent it after its
manner---rightly or falsely---is its form of representation.}


\PropositionE{2.171}
{The picture can represent every reality whose
form it has.

The spatial picture, everything spatial, the
coloured, everything coloured, etc.}


\PropositionE{2.172}
{The picture, however, cannot represent its form
of representation; it shows it forth.}


\PropositionE{2.173}
{The picture represents its object from without
(its standpoint is its form of representation), therefore
the picture represents its object rightly or
falsely.}


\PropositionE{2.174}
{But the picture cannot place itself outside of its
form of representation.}


\PropositionE{2.18}
{What every picture, of whatever form, must
have in common with reality in order to be able to
represent it at all---rightly or falsely---is the logical
form, that is, the form of reality.}


\PropositionE{2.181}
{If the form of representation is the logical form,
then the picture is called a logical picture.}


\PropositionE{2.182}
{Every picture is \emph{also} a logical picture. (On the
other hand, for example, not every picture is spatial.)}


\PropositionE{2.19}
{The logical picture can depict the world.}


\PropositionE{2.2}
{The picture has the logical form of representation
in common with what it pictures.}


\PropositionE{2.201}
{The picture depicts reality by representing a
possibility of the existence and non-existence of
atomic facts.}
% -----File: 043.png---


\PropositionE{2.202}
{The picture represents a possible state of affairs
in logical space.}


\PropositionE{2.203}
{The picture contains the possibility of the state
of affairs which it represents.}


\PropositionE{2.21}
{The picture agrees with reality or not; it is
right or wrong, true or false.}


\PropositionE{2.22}
{The picture represents what it represents, independently
of its truth or falsehood, through the
form of representation.}


\PropositionE{2.221}
{What the picture represents is its sense.}


\PropositionE{2.222}
{In the agreement or disagreement of its sense
with reality, its truth or falsity consists.}


\PropositionE{2.223}
{In order to discover whether the picture is true
or false we must compare it with reality.}


\PropositionE{2.224}
{It cannot be discovered from the picture alone
whether it is true or false.}


\PropositionE{2.225}
{There is no picture which is a priori true.}


\PropositionE{3}
{The logical picture of the facts is the
thought.}


\PropositionE{3.001}
{``An atomic fact is thinkable''---means: we can
imagine it.}


\PropositionE{3.01}
{The totality of true thoughts is a picture of the
world.}


\PropositionE{3.02}
{The thought contains the possibility of the state
of affairs which it thinks.
\enlargethispage{3pt} % enlarge to make the last line fit

What is thinkable is also possible.}


\PropositionE{3.03}
{We cannot think anything unlogical, for otherwise
we should have to think unlogically.}


\PropositionE{3.031}
{It used to be said that God could create everything,
except what was contrary to the laws of logic.
The truth is, we could not \emph{say} of an ``unlogical''
world how it would look.}


\PropositionE{3.032}
{To present in language anything which
``contradicts logic'' is as impossible as in
geometry to present by its co-ordinates a figure
which contradicts the laws of space; or to give
% -----File: 045.png---
the co-ordinates of a point which does not
exist.}


\PropositionE{3.0321}
{We could present spatially an atomic fact which
contradicted the laws of physics, but not one which
contradicted the laws of geometry.}


\PropositionE{3.04}
{An a priori true thought would be one whose
possibility guaranteed its truth.}


\PropositionE{3.05}
{We could only know a priori that a thought
is true if its truth was to be recognized from
the thought itself (without an object of comparison).}


\PropositionE{3.1}
{In the proposition the thought is expressed
perceptibly through the senses.}


\PropositionE{3.11}
{We use the sensibly perceptible sign (sound or
written sign, etc.) of the proposition as a projection
of the possible state of affairs.

The method of projection is the thinking of
the sense of the proposition.}


\PropositionE{3.12}
{The sign through which we express the thought
I call the propositional sign. And the proposition
is the propositional sign in its projective
relation to the world.}


\PropositionE{3.13}
{To the proposition belongs everything which
belongs to the projection; but not what is projected.

Therefore the possibility of what is projected but
not this itself.

In the proposition, therefore, its sense is not yet
contained, but the possibility of expressing it.

(``The content of the proposition'' means the
content of the significant proposition.)

In the proposition the form of its sense is
\enlargethispage{15pt} % enlarge to make last line fit
contained, but not its content.}


\PropositionE{3.14}
{The propositional sign consists in the fact that
its elements, the words, are combined in it in a
definite way.

The propositional sign is a fact.}


\PropositionE{3.141}
{The proposition is not a mixture of words
% -----File: 047.png---
(just as the musical theme is not a mixture of
tones).

The proposition is articulate.}


\PropositionE{3.142}
{Only facts can express a sense, a class of names
cannot.}


\PropositionE{3.143}
{That the propositional sign is a fact is concealed
by the ordinary form of expression, written or
printed.

(For in the printed proposition, for example, the
sign of a proposition does not appear essentially
different from a word. Thus it was possible for
Frege to call the proposition a compounded
name.)}


\PropositionE{3.1431}
{The essential nature of the propositional sign
becomes very clear when we imagine it made up
of spatial objects (such as tables, chairs, books)
instead of written signs.

The mutual spatial position of these things then
expresses the sense of the proposition.}


\PropositionE{3.1432}
{We must not say, ``The complex sign `$aRb$'
says `$a$ stands in relation $R$ to $b$'{}''; but we must
say, ``\emph{That} `$a$' stands in a certain relation to `$b$'
says \emph{that $aRb$}''.}


\PropositionE{3.144}
{States of affairs can be described but not
\emph{named}.

(Names resemble points; propositions resemble
arrows, they have sense.)}


\PropositionE{3.2}
{In propositions thoughts can be so expressed
that to the objects of the thoughts correspond the
elements of the propositional sign.}


\PropositionE{3.201}
{These elements I call ``simple signs'' and the
proposition ``completely analysed''.}


\PropositionE{3.202}
{The simple signs employed in propositions are
called names.}


\PropositionE{3.203}
{The name means the object. The object is its
meaning. (``$A$'' is the same sign as ``$A$''.)}


\PropositionE{3.21}
{To the configuration of the simple signs in the
% -----File: 049.png---
propositional sign corresponds the configuration
\enlargethispage{12pt} % enlarge to make the last word fit
of the objects in the state of affairs.}


\PropositionE{3.22}
{In the proposition the name represents the object.}


\PropositionE{3.221}
{Objects I can only \emph{name}. Signs represent them.
I can only speak \emph{of} them. I cannot \emph{assert them}.
A proposition can only say \emph{how} a thing is, not
\emph{what} it is.}


\PropositionE{3.23}
{The postulate of the possibility of the simple
signs is the postulate of the determinateness of
the sense.}


\PropositionE{3.24}
{A proposition about a complex stands in
internal relation to the proposition about its
constituent part.

A complex can only be given by its description,
and this will either be right or wrong. The proposition
in which there is mention of a complex,
if this does not exist, becomes not nonsense but
simply false.

That a propositional element signifies a complex
can be seen from an indeterminateness in the propositions
in which it occurs. We \emph{know} that everything
is not yet determined by this proposition.
(The notation for generality \emph{contains} a prototype.)

The combination of the symbols of a complex
in a simple symbol can be expressed by a definition.}


\PropositionE{3.25}
{There is one and only one complete analysis of
the proposition.}


\PropositionE{3.251}
{The proposition expresses what it expresses in
a definite and clearly specifiable way: the proposition
is articulate.}


\PropositionE{3.26}
{The name cannot be analysed further by any
definition. It is a primitive sign.}


\PropositionE{3.261}
{Every defined sign signifies \emph{via} those signs
by which it is defined, and the definitions show
the way.

Two signs, one a primitive sign, and one
defined by primitive signs, cannot signify in the
% -----File: 051.png---
same way. Names \emph{cannot} be taken to pieces by
definition (nor any sign which alone and independently
has a meaning).}


\PropositionE{3.262}
{What does not get expressed in the sign is
shown by its application. What the signs conceal,
their application declares.}


\PropositionE{3.263}
{The meanings of primitive signs can be
explained by elucidations. Elucidations are propositions
which contain the primitive signs. They
can, therefore, only be understood when the
meanings of these signs are already known.}


\PropositionE{3.3}
{Only the proposition has sense; only in the
context of a proposition has a name meaning.}


\PropositionE{3.31}
{Every part of a proposition which characterizes
its sense I call an expression (a symbol).

(The proposition itself is an expression.)

Expressions are everything---essential for the
sense of the prop\-o\-si\-tion---that propositions can
have in common with one another.

An expression characterizes a form and a
content.}


\PropositionE{3.311}
{An expression presupposes the forms of all
propositions in which it can occur. It is the
common characteristic mark of a class of propositions.}


\PropositionE{3.312}
{It is therefore represented by the general form
of the propositions which it characterizes.

And in this form the expression is \emph{constant} and
everything else \emph{variable}.}


\PropositionE{3.313}
{An expression is thus presented by a variable,
whose values are the propositions which contain
the expression.

(In the limiting case the variables become
constants, the expression a proposition.)

I call such a variable a ``propositional variable''.}


\PropositionE{3.314}
{An expression has meaning only in a proposition.
Every variable can be conceived as a
propositional variable.
% -----File: 053.png---

(Including the variable name.)}


\PropositionE{3.315}
{If we change a constituent part of a proposition
into a variable, there is a class of propositions
which are all the values of the resulting variable
proposition. This class in general still depends
on what, by arbitrary agreement, we mean by
parts of that proposition. But if we change all
those signs, whose meaning was arbitrarily determined,
into variables, there always remains such
a class. But this is now no longer dependent on
any agreement; it depends only on the nature of
the proposition. It corresponds to a logical form,
to a logical prototype.}


\PropositionE{3.316}
{What values the propositional variable can
assume is determined.

The determination of the values \emph{is} the variable.}


\PropositionE{3.317}
{The determination of the values of the propositional
variable is done by \emph{indicating the propositions}
whose common mark the variable is.

The determination is a description of these
propositions.

The determination will therefore deal only with
symbols not with their meaning.

And \emph{only} this is essential to the determination,
\emph{that it is only a description of symbols and asserts
nothing about what is symbolized}.

The way in which we describe the propositions
is not essential.}


\PropositionE{3.318}
{I conceive the proposition---like Frege and
Russell---as a function of the expressions contained
in it.}


\PropositionE{3.32}
{The sign is the part of the symbol perceptible
by the senses.}


\PropositionE{3.321}
{Two different symbols can therefore have the
sign (the written sign or the sound sign) in
common---they then signify in different ways.}
% -----File: 055.png---


\PropositionE{3.322}
{It can never indicate the common characteristic
of two objects that we symbolize them with the
same signs but by different \emph{methods of symbolizing}.
For the sign is arbitrary. We could therefore
equally well choose two different signs and
where then would be what was common in the
symbolization.}


\PropositionE{3.323}
{In the language of everyday life it very often
happens that the same word signifies in two different
ways---and therefore belongs to two different
symbols---or that two words, which signify in
different ways, are apparently applied in the same
way in the proposition.

Thus the word ``is'' appears as the copula,
as the sign of equality, and as the expression of
existence; ``to exist'' as an intransitive verb like
``to go''; ``identical'' as an adjective; we speak
of \emph{something} but also of the fact of \emph{something}
happening.

(In the proposition ``Green is green''---where
the first word is a proper name and the last an
adjective---these words have not merely different
meanings but they are \emph{different symbols}.)}


\PropositionE{3.324}
{Thus there easily arise the most fundamental
confusions (of which the whole of philosophy is
full).}


\PropositionE{3.325}
{In order to avoid these errors, we must employ
a symbolism which excludes them, by not applying
the same sign in different symbols and by
not applying signs in the same way which signify
in different ways. A symbolism, that is to say,
which obeys the rules of \emph{logical} grammar---of logical
syntax.

(The logical symbolism of Frege and Russell
is such a language, which, however, does still not
exclude all errors.)}
% -----File: 057.png---


\PropositionE{3.326}
{In order to recognize the symbol in the sign
we must consider the significant use.}


\PropositionE{3.327}
{The sign determines a logical form only together
with its logical syntactic application.}


\PropositionE{3.328}
{If a sign is \emph{not necessary} then it is meaningless.
That is the meaning of Occam's razor.

(If everything in the symbolism works as
though a sign had meaning, then it has meaning.)}


\PropositionE{3.33}
{In logical syntax the meaning of a sign ought
never to play a r&#244;le; it must admit of being
established without mention being thereby made
of the \emph{meaning} of a sign; it ought to presuppose
\emph{only} the description of the expressions.}


\PropositionE{3.331}
{From this observation we get a further view---into
Russell's \BookTitle{Theory of Types}. Russell's error is
shown by the fact that in drawing up his symbolic
rules he has to speak of the meaning of
the signs.}


\PropositionE{3.332}
{No proposition can say anything about itself,
because the propositional sign cannot be contained
in itself (that is the ``whole theory of types'').}


\PropositionE{3.333}
{A function cannot be its own argument, because
the functional sign already contains the prototype
of its own argument and it cannot contain
itself.

{\verystretchyspace
If, for example, we suppose that the function
$F(fx)$ could be its own argument, then there would
be a proposition ``$F(F(fx))$'', and in this the outer
function $F$ and the inner function $F$ must have
different meanings; for the inner has the form
$\phi(fx)$, the outer the form $\psi(\phi(fx))$. Common to
both functions is only the letter ``$F$'', which by
itself signifies nothing.}

This is at once clear, if instead of ``$F(F(u))$'' we
write ``$(\exists\phi) : F(\phi u) \DotOp \phi u = Fu$''.

Herewith Russell's paradox vanishes.}
% -----File: 059.png---


\PropositionE{3.334}
{The rules of logical syntax must follow of themselves,
if we only know how every single sign
signifies.}


\PropositionE{3.34}
{A proposition possesses essential and accidental
features.

Accidental are the features which are due to a
particular way of producing the propositional sign.
Essential are those which alone enable the proposition
to express its sense.}


\PropositionE{3.341}
{The essential in a proposition is therefore that
which is common to all propositions which can
express the same sense.

And in the same way in general the essential in
a symbol is that which all symbols which can
fulfil the same purpose have in common.}


\PropositionE{3.3411}
{One could therefore say the real name is that
which all symbols, which signify an object, have
in common. It would then follow, step by step,
that no sort of composition was essential for a name.}


\PropositionE{3.342}
{In our notations there is indeed something
arbitrary, but \emph{this} is not arbitrary, namely that
\emph{if} we have determined anything arbitrarily, then
something else \emph{must} be the case. (This results
from the \emph{essence} of the notation.)}


\PropositionE{3.3421}
{A particular method of symbolizing may be
unimportant, but it is always important that this
is a \emph{possible} method of symbolizing. And this
happens as a rule in philosophy: The single
thing proves over and over again to be unimportant,
but the possibility of every single thing reveals
something about the nature of the world.}


\PropositionE{3.343}
{Definitions are rules for the translation of one
language into another. Every correct symbolism
must be translatable into every other according
to such rules. It is \emph{this} which all have in
common.}
\enlargethispage{-9pt} % force the next proposition to the next page


\PropositionE{3.344}
{What signifies in the symbol is what is
common to all those symbols by which it can
% -----File: 061.png---
be replaced according to the rules of logical
syntax.}


\PropositionE{3.3441}
{We can, for example, express what is common to
all notations for the truth-functions as follows: It
is common to them that they all, for example, \emph{can
be replaced} by the notations of ``$\Not{p}$'' (``not $p$'')
and ``$p \lor q$'' (``$p$ or $q$'').

(Herewith is indicated the way in which a special
possible notation can give us general information.)}


\PropositionE{3.3442}
{The sign of the complex is not arbitrarily
resolved in the analysis, in such a way that its
resolution would be different in every propositional
structure.}


\PropositionE{3.4}
{The proposition determines a place in logical
space: the existence of this logical place is guaranteed
by the existence of the constituent parts alone,
by the existence of the significant proposition.}


\PropositionE{3.41}
{The propositional sign and the logical co-ordinates:
that is the logical place.}


\PropositionE{3.411}
{The geometrical and the logical place agree in
that each is the possibility of an existence.}


\PropositionE{3.42}
{Although a proposition may only determine
one place in logical space, the whole logical space
must already be given by it.

(Otherwise denial, the logical sum, the logical
product, etc., would always introduce new elements---in
co-ordination.)

(The logical scaffolding round the picture determines
the logical space. The proposition reaches
through the whole logical space.)}


\PropositionE{3.5}
{The applied, thought, propositional sign is the
thought.}


\PropositionE{4}
{The thought is the significant proposition.}


\PropositionE{4.001}
{The totality of propositions is the language.}


\PropositionE{4.002}
{Man possesses the capacity of constructing
languages, in which every sense can be expressed,
% -----File: 063.png---
without having an idea how and what each word
means---just as one speaks without knowing how
the single sounds are produced.

Colloquial language is a part of the human
organism and is not less complicated than it.

From it it is humanly impossible to gather
immediately the logic of language.

Language disguises the thought; so that from
the external form of the clothes one cannot infer
the form of the thought they clothe, because the
external form of the clothes is constructed with
quite another object than to let the form of the
body be recognized.

The silent adjustments to understand colloquial
language are enormously complicated.}


\PropositionE{4.003}
{Most propositions and questions, that have been
written about philosophical matters, are not false, but
senseless. We cannot, therefore, answer questions
of this kind at all, but only state their senselessness.
Most questions and propositions of the philosophers
result from the fact that we do not understand the
logic of our language.

(They are of the same kind as the question
whether the Good is more or less identical than the
Beautiful.)

And so it is not to be wondered at that the
deepest problems are really \emph{no} problems.}


\PropositionE{4.0031}
{All philosophy is ``Critique of language'' (but
not at all in Mauthner's sense). Russell's merit is
to have shown that the apparent logical form of the
proposition need not be its real form.}


\PropositionE{4.01}
{The proposition is a picture of reality.

The proposition is a model of the reality as we
think it is.}


\PropositionE{4.011}
{At the first glance the proposition---say as it
stands printed on paper---does not seem to be a
% -----File: 065.png---
picture of the reality of which it treats. But nor
does the musical score appear at first sight to be a
picture of a musical piece; nor does our phonetic
spelling (letters) seem to be a picture of our spoken
language. And yet these symbolisms prove to be
pictures---even in the ordinary sense of the word---of
what they represent.}


\PropositionE{4.012}
{It is obvious that we perceive a proposition
of the form $aRb$ as a picture. Here the sign is
obviously a likeness of the signified.}


\PropositionE{4.013}
{And if we penetrate to the essence of this
pictorial nature we see that this is not disturbed
by \emph{apparent irregularities} (like the use of $\sharp$ and $\flat$ in
the score).

For these irregularities also picture what they
are to express; only in another way.}


\PropositionE{4.014}
{The gramophone record, the musical thought,
the score, the waves of sound, all stand to one
another in that pictorial internal relation, which
holds between language and the world. To all of
them the logical structure is common.

(Like the two youths, their two horses and their
lilies in the story. They are all in a certain sense
one.)}


\PropositionE{4.0141}
{In the fact that there is a general rule by which
the musician is able to read the symphony out of
the score, and that there is a rule by which one
could reconstruct the symphony from the line on
a gramophone record and from this again---by
means of the first rule---construct the score, herein
lies the internal similarity between these things
which at first sight seem to be entirely different.
And the rule is the law of projection which projects
the symphony into the language of the musical
score. It is the rule of translation of this language
into the language of the gramophone record.}


\PropositionE{4.015}
{The possibility of all similes, of all the
% -----File: 067.png---
imagery of our language, rests on the logic of
representation.}


\PropositionE{4.016}
{In order to understand the essence of the
proposition, consider hieroglyphic writing, which
pictures the facts it describes.

And from it came the alphabet without the
essence of the representation being lost.}


\PropositionE{4.02}
{This we see from the fact that we understand
the sense of the propositional sign, without having
had it explained to us.}


\PropositionE{4.021}
{The proposition is a picture of reality, for I know
the state of affairs presented by it, if I understand
the proposition. And I understand the proposition,
without its sense having been explained to me.}


\PropositionE{4.022}
{The proposition \emph{shows} its sense.

The proposition \emph{shows} how things stand, \emph{if} it is
true. And it \emph{says}, that they do so stand.}


\PropositionE{4.023}
{The proposition determines reality to this
extent, that one only needs to say ``Yes'' or
``No'' to it to make it agree with reality.

It must therefore be completely described by
the proposition.

A proposition is the description of a fact.

As the description of an object describes it by
its external properties so propositions describe
reality by its internal properties.

The proposition constructs a world with the
help of a logical scaffolding, and therefore one
can actually see in the proposition all the logical
features possessed by reality if it is true. One can
\emph{draw conclusions} from a false proposition.}


\PropositionE{4.024}
{To understand a proposition means to know
what is the case, if it is true.

(One can therefore understand it without
knowing whether it is true or not.)

One understands it if one understands its
constituent parts.}


\PropositionE{4.025}
{The translation of one language into another
% -----File: 069.png---
is not a process of translating each proposition
of the one into a proposition of the other, but
only the constituent parts of propositions are
translated.

(And the dictionary does not only translate
substantives but also adverbs and conjunctions,
etc., and it treats them all alike.)}


\PropositionE{4.026}
{The meanings of the simple signs (the words)
must be explained to us, if we are to understand
them.

By means of propositions we explain ourselves.}


\PropositionE{4.027}
{It is essential to propositions, that they can
communicate a \emph{new} sense to us.}


\PropositionE{4.03}
{A proposition must communicate a new sense
with old words.

The proposition communicates to us a state of
affairs, therefore it must be \emph{essentially} connected
with the state of affairs.

And the connexion is, in fact, that it is its
logical picture.

The proposition only asserts something, in so
far as it is a picture.}


\PropositionE{4.031}
{In the proposition a state of affairs is, as it
were, put together for the sake of experiment.

One can say, instead of, This proposition has
such and such a sense, This proposition represents
such and such a state of affairs.}


\PropositionE{4.0311}
{One name stands for one thing, and another
for another thing, and they are connected together.
And so the whole, like a living picture, presents
the atomic fact.}


\PropositionE{4.0312}
{The possibility of propositions is based upon the
principle of the representation of objects by signs.

My fundamental thought is that the ``logical
constants'' do not represent. That the \emph{logic} of the
facts cannot be represented.}


\PropositionE{4.032}
{The proposition is a picture of its state of
affairs, only in so far as it is logically articulated.
% -----File: 071.png---

(Even the proposition ``ambulo'' is composite,
for its stem gives a different sense with another
termination, or its termination with another
stem.)}


\PropositionE{4.04}
{In the proposition there must be exactly as
many things distinguishable as there are in the
state of affairs, which it represents.

They must both possess the same logical
(mathematical) multiplicity (cf. Hertz's Mechanics,
on Dynamic Models).}


\PropositionE{4.041}
{This mathematical multiplicity naturally cannot
in its turn be represented. One cannot get outside
it in the representation.}


\PropositionE{4.0411}
{If we tried, for example, to express what is
expressed by ``$(x) \DotOp fx$'' by putting an index before
$fx$, like: ``Gen. $fx$'', it would not do, we should
not know what was generalized. If we tried to
show it by an index $g$, like: ``$f(x_{g})$'' it would not
do---we should not know the scope of the generalization.

If we were to try it by introducing a mark
in the argument places, like ``$(G,G) \DotOp F(G,G)$'', it
would not do---we could not determine the identity
of the variables, etc.

All these ways of symbolizing are inadequate
because they have not the necessary mathematical
multiplicity.}


\PropositionE{4.0412}
{For the same reason the idealist explanation of
the seeing of spatial relations through ``spatial
spectacles'' does not do, because it cannot explain
the multiplicity of these relations.}


\PropositionE{4.05}
{Reality is compared with the proposition.}


\PropositionE{4.06}
{Propositions can be true or false only by being
pictures of the reality.}


\PropositionE{4.061}
{If one does not observe that propositions have
a sense independent of the facts, one can easily
believe that true and false are two relations
% -----File: 073.png---
between signs and things signified with equal
rights.

One could then, for example, say that ``$p$''
signifies in the true way what ``$\Not{p}$'' signifies in
the false way, etc.}


\PropositionE{4.062}
{Can we not make ourselves understood by
means of false propositions as hitherto with true
ones, so long as we know that they are meant to
be false? No! For a proposition is true, if
what we assert by means of it is the case; and if
by ``$p$'' we mean $\Not{p}$, and what we mean is the
case, then ``$p$'' in the new conception is true
and not false.}


\PropositionE{4.0621}
{That, however, the signs ``$p$'' and ``$\Not{p}$'' \emph{can}
say the same thing is important, for it shows
that the sign ``$\Not{}$'' corresponds to nothing in
reality.

That negation occurs in a proposition, is no
characteristic of its sense ($\Not{\Not{p = p}}$).

The propositions ``$p$'' and ``$\Not{p}$'' have opposite
senses, but to them corresponds one and
the same reality.}


\PropositionE{4.063}
{An illustration to explain the concept of truth.
A black spot on white paper; the form of the spot
can be described by saying of each point of the
plane whether it is white or black. To the fact
that a point is black corresponds a positive fact;
to the fact that a point is white (not black), a
negative fact. If I indicate a point of the plane
(a truth-value in Frege's terminology), this corresponds
to the assumption proposed for judgment,
etc.\ etc.

But to be able to say that a point is black or
white, I must first know under what conditions a
point is called white or black; in order to be able
to say ``$p$'' is true (or false) I must have determined
under what conditions I call ``$p$'' true,
% -----File: 075.png---
and thereby I determine the sense of the proposition.

The point at which the simile breaks down is
this: we can indicate a point on the paper, without
knowing what white and black are; but to a proposition
without a sense corresponds nothing at
all, for it signifies no thing (truth-value) whose
properties are called ``false'' or ``true''; the verb
of the proposition is not ``is true'' or ``is false''---as
Frege thought---but that which ``is true'' must
already contain the verb.}


\PropositionE{4.064}
{Every proposition must \emph{already} have a sense;
assertion cannot give it a sense, for what it asserts
is the sense itself. And the same holds of
denial, etc.}


\PropositionE{4.0641}
{One could say, the denial is already related to
the logical place determined by the proposition
that is denied.

The denying proposition determines a logical
place \emph{other} than does the proposition denied.

The denying proposition determines a logical
place, with the help of the logical place of the
proposition denied, by saying that it lies outside
the latter place.

That one can deny again the denied proposition,
shows that what is denied is already a proposition
and not merely the preliminary to a
proposition.}


\PropositionE{4.1}
{A proposition presents the existence and non-existence
of atomic facts.}


\PropositionE{4.11}
{The totality of true propositions is the total
natural science (or the totality of the natural
sciences).}


\PropositionE{4.111}
{Philosophy is not one of the natural
sciences.

(The word ``philosophy'' must mean something
% -----File: 077.png---
which stands above or below, but not beside the
natural sciences.)}


\PropositionE{4.112}
{The object of philosophy is the logical clarification
of thoughts.

Philosophy is not a theory but an activity.

A philosophical work consists essentially of
elucidations.

The result of philosophy is not a number of
``philosophical propositions'', but to make propositions
clear.

{\verystretchyspace
Philosophy should make clear and delimit
sharply the thoughts which otherwise are, as it
were, opaque and blurred.}}


\PropositionE{4.1121}
{Psychology is no nearer related to philosophy,
than is any other natural science.

The theory of knowledge is the philosophy of
psychology.

Does not my study of sign-language correspond
to the study of thought processes which philosophers
held to be so essential to the philosophy of logic?
Only they got entangled for the most part in unessential
psychological investigations, and there
is an analogous danger for my method.}


\PropositionE{4.1122}
{The Darwinian theory has no more to do with
philosophy than has any other hypothesis of natural
science.}


\PropositionE{4.113}
{Philosophy limits the disputable sphere of natural
science.}


\PropositionE{4.114}
{It should limit the thinkable and thereby the
unthinkable.

{\stretchyspace
It should limit the unthinkable from within
through the thinkable.}}


\PropositionE{4.115}
{It will mean the unspeakable by clearly displaying
the speakable.}


\PropositionE{4.116}
{Everything that can be thought at all can be
% -----File: 079.png---
thought clearly. Everything that can be said can
be said clearly.}


\PropositionE{4.12}
{Propositions can represent the whole reality,
but they cannot represent what they must have in
common with reality in order to be able to represent
it---the logical form.

To be able to represent the logical form, we
should have to be able to put ourselves with the
propositions outside logic, that is outside the
world.}


\PropositionE{4.121}
{Propositions cannot represent the logical form:
this mirrors itself in the propositions.

That which mirrors itself in language, language
cannot represent.

That which expresses \emph{itself} in language, \emph{we}
cannot express by language.

The propositions \emph{show} the logical form of reality.

They exhibit it.}


\PropositionE{4.1211}
{Thus a proposition ``$fa$'' shows that in its sense
the object $a$ occurs, two propositions ``$fa$'' and
``$ga$'' that they are both about the same object.

If two propositions contradict one another, this
is shown by their structure; similarly if one follows
from another, etc.}


\PropositionE{4.1212}
{What \emph{can} be shown \emph{cannot} be said.}


\PropositionE{4.1213}
{Now we understand our feeling that we are in
possession of the right logical conception, if only
all is right in our symbolism.}


\PropositionE{4.122}
{We can speak in a certain sense of formal
properties of objects and atomic facts, or of properties
of the structure of facts, and in the same
sense of formal relations and relations of
structures.

(Instead of property of the structure I also say
% -----File: 081.png---
``internal property''; instead of relation of structures
``internal relation''.

I introduce these expressions in order to show
the reason for the confusion, very widespread
among philosophers, between internal relations
and proper (external) relations.)

The holding of such internal properties and relations
cannot, however, be asserted by propositions,
but it shows itself in the propositions, which
present the atomic facts and treat of the objects in
question.}


\PropositionE{4.1221}
{An internal property of a fact we also call a
feature of this fact. (In the sense in which we
speak of facial features.)}


\PropositionE{4.123}
{A property is internal if it is unthinkable that
its object does not possess it.

(This blue colour and that stand in the internal
relation of brighter and darker eo ipso. It is
unthinkable that \emph{these} two objects should not stand
in this relation.)

(Here to the shifting use of the words ``property''
and ``relation'' there corresponds the shifting use
of the word ``object''.)}


\PropositionE{4.124}
{The existence of an internal property of a possible
state of affairs is not expressed by a proposition,
but it expresses itself in the proposition which
presents that state of affairs, by an internal property
of this proposition.

It would be as senseless to ascribe a formal
property to a proposition as to deny it the formal
property.}


\PropositionE{4.1241}
{One cannot distinguish forms from one another
by saying that one has this property but the other
that: for this assumes that there is a sense in asserting
either property of either form.}


\PropositionE{4.125}
{The existence of an internal relation between
% -----File: 083.png---
possible states of affairs expresses itself in language
by an internal relation between the propositions
presenting them.}


\PropositionE{4.1251}
{Here the disputed question ``whether all relations
are internal or external'' disappears.}


\PropositionE{4.1252}
{Series which are ordered by \emph{internal} relations I
call formal series.

The series of numbers is ordered not by an
external, but by an internal relation.

Similarly the series of propositions ``$aRb$'',
\[
\begin{array}{l}
``(\exists x) : aRx \DotOp xRb\text{'',}\\
``(\exists x,y) : aRx \DotOp aRy \DotOp yRb\text{'', etc.}
\end{array}
\]

(If $b$ stands in one of these relations to $a$, I call
$b$ a successor of $a$.)}


\PropositionE{4.126}
{In the sense in which we speak of formal
properties we can now speak also of formal
concepts.

(I introduce this expression in order to make
clear the confusion of formal concepts with proper
concepts which runs through the whole of the old
logic.)

That anything falls under a formal concept as
an object belonging to it, cannot be expressed by
a proposition. But it shows itself in the sign of
this object itself. (The name shows that it signifies
an object, the numerical sign that it signifies a
number, etc.)

Formal concepts cannot, like proper concepts,
be presented by a function.

For their characteristics, the formal properties,
are not expressed by the functions.

The expression of a formal property is a feature
of certain symbols.

The sign that signifies the characteristics of a
formal concept is, therefore, a characteristic feature
of all symbols, whose meanings fall under the
concept.
% -----File: 085.png---

The expression of the formal concept is therefore
a propositional variable in which only this
characteristic feature is constant.}


\PropositionE{4.127}
{The propositional variable signifies the formal
concept, and its values signify the objects which
fall under this concept.}


\PropositionE{4.1271}
{Every variable is the sign of a formal
concept.

For every variable presents a constant form,
which all its values possess, and which can
be conceived as a formal property of these
values.}


\PropositionE{4.1272}
{So the variable name ``$x$'' is the proper sign of
the pseudo-concept \emph{object}.

Wherever the word ``object'' (``thing'', ``entity'',
etc.) is rightly used, it is expressed in logical
symbolism by the variable name.

For example in the proposition ``there are two
objects which \ldots'', by ``$(\exists x,y)$ \ldots''.

Wherever it is used otherwise, \idEst\ as a proper
concept word, there arise senseless pseudo-propositions.

So one cannot, \exempliGratia\ say ``There are objects''
as one says ``There are books''. Nor ``There
are 100 objects'' or ``There are $\aleph_0$ objects''. And
it is senseless to speak of the \emph{number of all
objects}.

The same holds of the words ``Complex'',
``Fact'', ``Function'', ``Number'', etc.

They all signify formal concepts and are
presented in logical symbolism by variables, not
by functions or classes (as Frege and Russell
thought).

Expressions like ``1 is a number'', ``there is
only one number nought'', and all like them are
senseless.

(It is as senseless to say, ``there is only one 1''
% -----File: 087.png---
as it would be to say: 2 + 2 is at 3 o'clock equal
to 4.)}


\PropositionE{4.12721}
{The formal concept is already given with an
object, which falls under it. One cannot, therefore,
introduce both, the objects which fall under
a formal concept \emph{and} the formal concept itself,
as primitive ideas. One cannot, therefore, \exempliGratia\ introduce
(as Russell does) the concept of function
and also special functions as primitive ideas; or
the concept of number and definite numbers.}


\PropositionE{4.1273}
{If we want to express in logical symbolism
the general proposition ``$b$ is a successor of $a$''
we need for this an expression for the general
term of the formal series: $aRb$, $(\exists x) : aRx \DotOp xRb$,
$(\exists x,y) : aRx \DotOp xRy \DotOp yRb$,\;\ldots The general term of
a formal series can only be expressed by a
variable, for the concept symbolized by ``term of
this formal series'' is a \emph{formal} concept. (This
Frege and Russell overlooked; the way in
which they express general propositions like the
above is, therefore, false; it contains a vicious
circle.)

We can determine the general term of the
formal series by giving its first term and the
general form of the operation, which generates
the following term out of the preceding proposition.}


\PropositionE{4.1274}
{The question about the existence of a formal
concept is senseless. For no proposition can
answer such a question.

(For example, one cannot ask: ``Are there
unanalysable sub\-ject-pre\-di\-cate propositions?'')}


\PropositionE{4.128}
{The logical forms are \emph{anumerical}.

Therefore there are in logic no pre-eminent
numbers, and therefore there is no philosophical
monism or dualism, etc.}


\PropositionE{4.2}
{The sense of a proposition is its agreement
and disagreement with the possibilities of the
% -----File: 089.png---
existence and non-existence of the atomic
facts.}


\PropositionE{4.21}
{The simplest proposition, the elementary proposition,
asserts the existence of an atomic fact.}


\PropositionE{4.211}
{It is a sign of an elementary proposition,
that no elementary proposition can contradict
it.}


\PropositionE{4.22}
{The elementary proposition consists of names.
It is a connexion, a concatenation, of names.}


\PropositionE{4.221}
{It is obvious that in the analysis of propositions
we must come to elementary propositions, which
consist of names in immediate combination.

The question arises here, how the propositional
connexion comes to be.}


\PropositionE{4.2211}
{Even if the world is infinitely complex, so
that every fact consists of an infinite number
of \DPtypo{atomatic}{atomic} facts and every atomic fact is
composed of an infinite number of objects,
even then there must be objects and atomic
facts.}


\PropositionE{4.23}
{The name occurs in the proposition only in
the context of the elementary proposition.}


\PropositionE{4.24}
{The names are the simple symbols, I indicate
them by single letters ($x$, $y$, $z$).

The elementary proposition I write as function
of the names, in the form ``$fx$'', ``$\phi(x,y)$'', etc.

Or I indicate it by the letters $p$, $q$, $r$.}


\PropositionE{4.241}
{If I use two signs with one and the same
meaning, I express this by putting between them
the sign ``=''.

``$a = b$'' means then, that the sign ``$a$'' is
replaceable by the sign ``$b$''.

(If I introduce by an equation a new sign ``$b$'',
by determining that it shall replace a previously
known sign ``$a$'', I write the equation---definition---(like
% -----File: 091.png---
Russell) in the form ``$a = b$ Def.''. A
definition is a symbolic rule.)}


\PropositionE{4.242}
{Expressions of the form ``$a = b$'' are therefore only
expedients in presentation: They assert nothing
about the meaning of the signs ``$a$'' and ``$b$''.}


\PropositionE{4.243}
{Can we understand two names without knowing
whether they signify the same thing or two
different things? Can we understand a proposition
in which two names occur, without knowing if they
mean the same or different things?

If I know the meaning of an English and a
synonymous German word, it is impossible for
me not to know that they are synonymous, it is
impossible for me not to be able to translate them
into one another.

Expressions like ``$a = a$'', or expressions
deduced from these are neither elementary propositions
nor otherwise significant signs. (This
will be shown later.)}


\PropositionE{4.25}
{If the elementary proposition is true, the atomic
fact exists; if it is false the atomic fact does not
exist.}


\PropositionE{4.26}
{The specification of all true elementary propositions
describes the world completely. The
world is completely described by the specification
of all elementary propositions plus the specification,
which of them are true and which false.}


\PropositionE{4.27}
{With regard to the existence of $n$ atomic facts
there are $K_{n} = \sum\limits_{\nu = 0}^n\binom{n}{\nu}$ possibilities.

It is possible for all combinations of atomic
facts to exist, and the others not to exist.}


\PropositionE{4.28}
{To these combinations correspond the same
number of possibilities of the truth---and falsehood---of
$n$ elementary propositions.}


\PropositionE{4.3}
{The truth-possibilities of the elementary propositions
mean the possibilities of the existence
and non-existence of the atomic facts.}
% -----File: 093.png---


\PropositionE{4.31}
{The truth-possibilities can be presented by
schemata of the following kind (``T'' means
``true'', ``F'' ``false''. The rows of T's and F's
under the row of the elementary propositions mean
their truth-possibilities in an easily intelligible
symbolism).

<div class="truthtable__container">
    <table class="truthtable">
        <tr>
            <th>p</th>
            <th>q</th>
            <th>r</th>
        </tr>
        <tr>
            <td class="truthtable__separator"></td>
            <td class="truthtable__separator"></td>
            <td class="truthtable__separator"></td>
        </tr>
        <tr>
            <td>T</td>
            <td>T</td>
            <td>T</td>
        </tr>
        <tr>
            <td>F</td>
            <td>T</td>
            <td>T</td>
        </tr>
        <tr>
            <td>T</td>
            <td>F</td>
            <td>T</td>
        </tr>
        <tr>
            <td>T</td>
            <td>T</td>
            <td>F</td>
        </tr>
        <tr>
            <td>F</td>
            <td>F</td>
            <td>T</td>
        </tr>
        <tr>
            <td>F</td>
            <td>T</td>
            <td>F</td>
        </tr>
        <tr>
            <td>T</td>
            <td>F</td>
            <td>F</td>
        </tr>
        <tr>
            <td>F</td>
            <td>F</td>
            <td>F</td>
        </tr>
    </table>

    <table class="truthtable">
        <tr>
            <th>p</th>
            <th>q</th>
        </tr>
        <tr>
            <td class="truthtable__separator"></td>
            <td class="truthtable__separator"></td>
        </tr>
        <tr>
            <td>T</td>
            <td>T</td>
        </tr>
        <tr>
            <td>F</td>
            <td>T</td>
        </tr>
        <tr>
            <td>T</td>
            <td>F</td>
        </tr>
        <tr>
            <td>F</td>
            <td>F</td>
        </tr>
    </table>

    <table class="truthtable">
        <tr>
            <th>p</th>
        </tr>
        <tr>
            <td class="truthtable__separator"></td>
        </tr>
        <tr>
            <td>T</td>
        </tr>
        <tr>
            <td>F</td>
        </tr>
    </table>
</div>

}


\PropositionE{4.4}
{A proposition is the expression of agreement
and disagreement with the truth-pos\-si\-bil\-i\-ties of
the elementary propositions.}


\PropositionE{4.41}
{The truth-possibilities of the elementary propositions
are the conditions of the truth and
falsehood of the propositions.}


\PropositionE{4.411}
{It seems probable even at first sight that the
introduction of the elementary propositions is
fundamental for the comprehension of the other
kinds of propositions. Indeed the comprehension
of the general propositions depends \emph{palpably} on
that of the elementary propositions.}


\PropositionE{4.42}
{With regard to the agreement and disagreement
of a proposition with the truth-possibilities
of $n$ elementary propositions there
are $\sum\limits_{\kappa = 0}^{K_n}\binom{K_n}{\kappa} = L_{n}$ possibilities.}


\PropositionE{4.43}
{Agreement with the truth-possibilities can be
% -----File: 095.png---
expressed by co-or\-di\-na\-ting with them in the
schema the mark ``T'' (true).

Absence of this mark means disagreement.}


\PropositionE{4.431}
{The expression of the agreement and disagreement
with the truth-pos\-si\-bil\-i\-ties of the elementary
propositions expresses the truth-conditions of the
proposition.

The proposition is the expression of its truth-conditions.

(Frege has therefore quite rightly put them at
the beginning, as explaining the signs of his
logical symbolism. Only Frege's explanation
of the truth-concept is false: if ``the true'' and
``the false'' were real objects and the arguments
in $\Not{p}$, etc., then the sense of $\Not{p}$ would by no
means be determined by Frege's determination.)}


\PropositionE{4.44}
{The sign which arises from the co-ordination of
that mark ``T'' with the truth-pos\-si\-bil\-i\-ties is a
propositional sign.}


\PropositionE{4.441}
{It is clear that to the complex of the signs ``F''
and ``T'' no object (or complex of objects) corresponds;
any more than to horizontal and vertical
lines or to brackets. There are no ``logical
objects''.

Something analogous holds of course for all
signs, which express the same as the schemata of
``T'' and ``F''.}


\PropositionE{4.442}
{Thus \exempliGratia

<div class="truthtable__container">
    <span>``</span>
    <table class="truthtable">
        <tr>
            <th>p</th>
            <th>q</th>
            <th class="truthtable__separator--left"></th>
            <th></th>
        </tr>
        <tr>
            <td class="truthtable__separator"></td>
            <td class="truthtable__separator"></td>
            <td class="truthtable__separator--both"></td>
            <td class="truthtable__separator"></td>
        </tr>
        <tr>
            <td>T</td>
            <td>T</td>
            <td class="truthtable__separator--left"></td>
            <td>T</td>
        </tr>
        <tr>
            <td>F</td>
            <td>T</td>
            <td class="truthtable__separator--left"></td>
            <td>T</td>
        </tr>
        <tr>
            <td>T</td>
            <td>F</td>
            <td class="truthtable__separator--left"></td>
            <td></td>
        </tr>
        <tr>
            <td>F</td>
            <td>F</td>
            <td class="truthtable__separator--left"></td>
            <td>T</td>
        </tr>
    </table>
    <span>''</span>
</div>

is a propositional sign.

(Frege's assertion sign ``$\vdash$'' is logically altogether
% -----File: 097.png---
meaningless; in Frege (and Russell) it only shows
that these authors hold as true the propositions
marked in this way.

``$\vdash$'' belongs therefore to the propositions no
more than does the number of the proposition. A
proposition cannot possibly assert of itself that it
is true.)

If the sequence of the truth-possibilities in the
schema is once for all determined by a rule of
combination, then the last column is by itself an
expression of the truth-conditions. If we write
this column as a row the propositional sign becomes:
``(TT-T)($p$, $q$)'', or more plainly: ``(TTFT)($p$, $q$)''.

(The number of places in the left-hand bracket
is determined by the number of terms in the right-hand
bracket.)}


\PropositionE{4.45}
{For $n$ elementary propositions there are $L_{n}$
possible groups of truth-con\-di\-tions.

The groups of truth-conditions which belong to
the truth-pos\-si\-bil\-i\-ties of a number of elementary
propositions can be ordered in a series.}


\PropositionE{4.46}
{Among the possible groups of truth-conditions
there are two extreme cases.

In the one case the proposition is true for all the
truth-pos\-si\-bil\-i\-ties of the elementary propositions.
We say that the truth-conditions are \emph{tautological}.

In the second case the proposition is false for all
the truth-pos\-si\-bil\-i\-ties. The truth-conditions are
\emph{self-contradictory}.

In the first case we call the proposition a
tautology, in the second case a contradiction.}


\PropositionE{4.461}
{The proposition shows what it says, the
tautology and the contradiction that they say
nothing.

The tautology has no truth-conditions, for it is
% -----File: 099.png---
unconditionally true; and the contradiction is on
no condition true.

Tautology and contradiction are without sense.

(Like the point from which two arrows go out in
opposite directions.)

(I know, \exempliGratia\ nothing about the weather, when
I know that it rains or does not rain.)}


\PropositionE{4.4611}
{Tautology and contradiction are, however, not
senseless; they are part of the symbolism, in the
same way that ``0'' is part of the symbolism of
Arithmetic.}


\PropositionE{4.462}
{Tautology and contradiction are not pictures of
the reality. They present no possible state of
affairs. For the one allows \emph{every} possible state
of affairs, the other \emph{none}.

In the tautology the conditions of agreement
with the world\AllowBreak---the presenting re\-la\-tions---cancel
one another, so that it stands in no presenting
relation to reality.}


\PropositionE{4.463}
{The truth-conditions determine the range, which
is left to the facts by the proposition.

(The proposition, the picture, the model, are in
a negative sense like a solid body, which restricts
the free movement of another: in a positive sense,
like the space limited by solid substance, in which
a body may be placed.)

Tautology leaves to reality the whole infinite
logical space; contradiction fills the whole logical
space and leaves no point to reality. Neither of
them, therefore, can in any way determine
reality.}


\PropositionE{4.464}
{The truth of tautology is certain, of propositions
possible, of contradiction impossible. (Certain,
possible, impossible: here we have an indication
of that gradation which we need in the theory of
probability.)}


\PropositionE{4.465}
{The logical product of a tautology and a proposition
% -----File: 101.png---
says the same as the proposition. Therefore
that product is identical with the proposition.
For the essence of the symbol cannot be altered
without altering its sense.}


\PropositionE{4.466}
{To a definite logical combination of signs
corresponds a definite logical combination of their
meanings; \emph{every arbitrary} combination only corresponds
to the unconnected signs.

That is, propositions which are true for every
state of affairs cannot be combinations of signs at
all, for otherwise there could only correspond to
them definite combinations of objects.

(And to no logical combination corresponds \emph{no}
combination of the objects.)

Tautology and contradiction are the limiting
cases of the combinations of symbols, namely their
dissolution.}


\PropositionE{4.4661}
{Of course the signs are also combined with one
another in the tautology and contradiction, \idEst\ they
stand in relations to one another, but these
relations are meaningless, unessential to the
\emph{symbol}.}


\PropositionE{4.5}
{Now it appears to be possible to give the
most general form of proposition; \idEst\ to give a
description of the propositions of some one sign
language, so that every possible sense can be
expressed by a symbol, which falls under the
description, and so that every symbol which falls
under the description can express a sense, if
the meanings of the names are chosen accordingly.

It is clear that in the description of the most
general form of proposition \emph{only} what is essential
to it may be described---otherwise it would not be
the most general form.

That there is a general form is proved by the
fact that there cannot be a proposition whose
form could not have been foreseen (\idEst\ constructed).
% -----File: 103.png---
The general form of proposition is: Such and
such is the case.}


\PropositionE{4.51}
{Suppose \emph{all} elementary propositions were given
me: then we can simply ask: what propositions I
can build out of them. And these are \emph{all} propositions
and \emph{so} are they limited.}


\PropositionE{4.52}
{The propositions are everything which follows
from the totality of all elementary propositions (of
course also from the fact that it is the \emph{totality of
them all}). (So, in some sense, one could say, that
\emph{all} propositions are generalizations of the elementary
propositions.)}


\PropositionE{4.53}
{The general propositional form is a variable.}


\PropositionE{5}
{Propositions are truth-functions of elementary
propositions.

(An elementary proposition is a truth-function
of itself.)}


\PropositionE{5.01}
{The elementary propositions are the truth-arguments
of propositions.}


\PropositionE{5.02}
{It is natural to confuse the arguments of
functions with the indices of names. For I
recognize the meaning of the sign containing it
from the argument just as much as from the
index.

In Russell's ``$\DPtypo{+}{+_{c}}$'', for example, ``$c$'' is an
index which indicates that the whole sign is the
addition sign for cardinal numbers. But this way
of symbolizing depends on arbitrary agreement,
and one could choose a simple sign instead of
``$+_{c}$'': but in ``$\Not{p}$'' ``$p$'' is not an index but
an argument; the sense of ``$\Not{p}$'' \emph{cannot} be understood,
unless the sense of ``$p$'' has previously been
understood. (In the name Julius C&aelig;sar, Julius is
an index. The index is always part of a description
of the object to whose name we attach it, \exempliGratia\ \emph{The}
C&aelig;sar of the Julian gens.)

The confusion of argument and index is, if I
am not mistaken, at the root of Frege's theory
% -----File: 105.png---
of the meaning of propositions and functions. For
Frege the propositions of logic were names and
their arguments the indices of these names.}


\PropositionE{5.1}
{The truth-functions can be ordered in
series.

That is the foundation of the theory of probability.}
\enlargethispage{-9pt} % force the next proposition to the next page

\PropositionE{5.101}
{The truth-functions of every number of elementary
propositions can be written in a schema of
the following kind:

<table class="table5101">
    <tr>
        <td>(TTTT)($p, q$)</td> 
        <td>Tautology</td> 
        <td>(if $p$ then $p$, and if $q$ then $q$) <br/> ($p \Implies p \DotOp q \Implies q$)</td>
    </tr>
    <tr>
        <td>(FTTT)($p, q$)</td>
        <td>in words:</td> 
        <td>Not both $p$ and $q$. ($\Not{(p \DotOp q)}$)</td>
    </tr>
    <tr>
        <td>(TFTT)($p, q$)</td> 
        <td>" &nbsp &nbsp "</td>
        <td>If $q$ then $p$. ($q \Implies p$)</td>
    </tr>
    <tr>
        <td>(TTFT)($p, q$)</td> 
        <td>" &nbsp &nbsp "</td>
        <td>If $p$ then $q$. ($p \Implies q$)</td>
    </tr>
    <tr>
        <td>(TTTF)($p, q$)</td>
        <td>" &nbsp &nbsp "</td>
        <td>$p$ or $q$. ($p \lor q$)</td>
    </tr>
    <tr>
        <td>(FFTT)($p, q$)</td>
        <td>" &nbsp &nbsp "</td>
        <td>Not $q$. ($\Not{q}$)</td>
    </tr>
    <tr>
        <td>(FTFT)($p, q$)</td>
        <td>" &nbsp &nbsp "</td>
        <td>Not $p$. ($\Not{p}$)</td>
    </tr>
    <tr>
        <td>(FTTF)($p, q$)</td>
        <td>" &nbsp &nbsp "</td>
        <td>$p$ or $q$, but not both. <br/> ($p \DotOp \Not{q} : \lor : q \DotOp \Not{p}$)</td>
    </tr>
    <tr>
        <td>(TFFT)($p, q$)</td>
        <td>" &nbsp &nbsp "</td>
        <td>If $p$, then $q$; and if $q$, then $p$. ($p \equiv q$)</td>
    </tr>
    <tr>
        <td>(TFTF)($p, q$)</td>
        <td>" &nbsp &nbsp "</td>
        <td>$p$</td>
    </tr>
    <tr>
        <td>(TTFF)($p, q$)</td>
        <td>" &nbsp &nbsp "</td>
        <td>$q$</td>
    </tr>
    <tr>
        <td>(FFFT)($p, q$)</td>
        <td>" &nbsp &nbsp "</td>
        <td>Neither $p$ nor $q$. ($\Not{p} \DotOp \Not{q}$) or ($p \BarOp q$)</td>
    </tr>
    <tr>
        <td>(FFTF)($p, q$)</td>
        <td>" &nbsp &nbsp "</td>
        <td>$p$ and not $q$. ($p \DotOp \Not{q}$)</td>
    </tr>
    <tr>
        <td>(FTFF)($p, q$)</td>
        <td>" &nbsp &nbsp "</td>
        <td>$q$ and not $p$. ($q \DotOp \Not{p}$)</td>
    </tr>
    <tr>
        <td>(TFFF)($p, q$)</td>
        <td>" &nbsp &nbsp "</td>
        <td>$p$ and $q$. ($p \DotOp q$)</td>
    </tr>
    <tr>
        <td>(FFFF)($p, q$)</td>
        <td>Contradiction</td>
        <td>($p$ and not $p$; and $q$ and not $q$.) <br/> ($p \DotOp \Not{p} \DotOp q \DotOp \Not{q}$)</td>
    </tr>
</table>
Those truth-possibilities of its truth-arguments,
which verify the proposition, I shall call its \emph{truth-grounds}.}


\PropositionE{5.11}
{If the truth-grounds which are common to a
number of propositions are all also truth-grounds
of some one proposition, we say that the truth of
this proposition follows from the truth of those
propositions.}


\PropositionE{5.12}
{In particular the truth of a proposition $p$ follows
from that of a proposition $q$, if all the truth-grounds
of the second are truth-grounds of the
first.}
% -----File: 107.png---


\PropositionE{5.121}
{The truth-grounds of $q$ are contained in those
of $p$; $p$ follows from $q$.}


\PropositionE{5.122}
{If $p$ follows from $q$, the sense of ``$p$'' is contained
in that of ``$q$''.}


\PropositionE{5.123}
{If a god creates a world in which certain propositions
are true, he creates thereby also a world
in which all propositions consequent on them are
true. And similarly he could not create a world
in which the proposition ``$p$'' is true without
creating all its objects.}


\PropositionE{5.124}
{A proposition asserts every proposition which
follows from it.}


\PropositionE{5.1241}
{``$p \DotOp q$'' is one of the propositions which assert
``$p$'' and at the same time one of the propositions
which assert ``$q$''.

Two propositions are opposed to one another
if there is no significant proposition which asserts
them both.

Every proposition which contradicts another,
denies it.}


\PropositionE{5.13}
{That the truth of one proposition follows from
the truth of other propositions, we perceive from
the structure of the propositions.}


\PropositionE{5.131}
{If the truth of one proposition follows from the
truth of others, this expresses itself in relations in
which the forms of these propositions stand to one
another, and we do not need to put them in these
relations first by connecting them with one another
in a proposition; for these relations are internal,
and exist as soon as, and by the very fact that,
the propositions exist.}


\PropositionE{5.1311}
{When we conclude from $p \lor q$ and $\Not{p}$ to $q$ the
relation between the forms of the propositions
``$p \lor q$'' and ``$\Not{p}$'' is here concealed by the method
of symbolizing. But if we write, \exempliGratia\ instead of
``$p \lor q$'' ``$p \BarOp q \DotOp \BarOp \DotOp p \BarOp q$'' and instead of ``$\Not{p}$''
``$p \BarOp p$'' ($p \BarOp q$ = neither $p$ nor $q$), then the inner
connexion becomes obvious.
% -----File: 109.png---

(The fact that we can infer $fa$ from $(x) \DotOp fx$ shows
that generality is present also in the symbol
``$(x) \DotOp fx$''.}


\PropositionE{5.132}
{If $p$ follows from $q$, I can conclude from $q$ to $p$;
infer $p$ from $q$.

The method of inference is to be understood
from the two propositions alone.

Only they themselves can justify the inference.

Laws of inference, which---as in Frege and
Russell---are to justify the conclusions, are senseless
and would be superfluous.}


\PropositionE{5.133}
{All inference takes place a priori.}


\PropositionE{5.134}
{From an elementary proposition no other can
be inferred.}


\PropositionE{5.135}
{In no way can an inference be made from the
existence of one state of affairs to the existence of
another entirely different from it.}


\PropositionE{5.136}
{There is no causal nexus which justifies such
an inference.}


\PropositionE{5.1361}
{The events of the future \emph{cannot} be inferred from
those of the present.

Superstition is the belief in the causal
nexus.}


\PropositionE{5.1362}
{The freedom of the will consists in the fact that
future actions cannot be known now. We could
only know them if causality were an \emph{inner} necessity,
like that of logical deduction.---The connexion
of knowledge and what is known is that of logical
necessity.

(``A knows that $p$ is the case'' is senseless if $p$
is a tautology.)}


\PropositionE{5.1363}
{If from the fact that a proposition is obvious
to us it does not \emph{follow} that it is true, then obviousness
is no justification for our belief in its truth.}


\PropositionE{5.14}
{If a proposition follows from another, then the
latter says more than the former, the former less
than the latter.}
% -----File: 111.png---


\PropositionE{5.141}
{If $p$ follows from $q$ and $q$ from $p$ then they are
one and the same proposition.}


\PropositionE{5.142}
{A tautology follows from all propositions: it
says nothing.}


\PropositionE{5.143}
{Contradiction is something shared by propositions,
which \emph{no} proposition has in common with
another. Tautology is that which is shared by
all propositions, which have nothing in common
with one another.

Contradiction vanishes so to speak outside,
tautology inside all propositions.

Contradiction is the external limit of the propositions,
tautology their substanceless centre.}


\PropositionE{5.15}
{If $T_{r}$ is the number of the truth-grounds of the
proposition ``$r$'', $T_{rs}$ the number of those truth-grounds
of the proposition ``$s$'' which are at the
same time truth-grounds of ``$r$'', then we call the
ratio $T_{rs} : T_{r}$ the measure of the probability which
the proposition ``$r$'' gives to the proposition ``$s$''.}


\PropositionE{5.151}
{Suppose in a schema like that above in No.~\PropERef{5.101}
$T_{r}$ is the number of the ``T'''s in the proposition
$r$, $T_{rs}$ the number of those ``T'''s in
the proposition $s$, which stand in the same columns
as ``T'''s of the proposition $r$; then the proposition
$r$ gives to the proposition $s$ the probability
$T_{rs} : T_{r}$.}


\PropositionE{5.1511}
{There is no special object peculiar to probability
propositions.}


\PropositionE{5.152}
{Propositions which have no truth-arguments
in common with one another we call independent.
\enlargethispage{-3pt} % force next paragraph to the next page

Independent propositions (\exempliGratia\ any two elementary
propositions) give to one another the probability $\frac{1}{2}$.

If $p$ follows from $q$, the proposition $q$ gives
to the proposition $p$ the probability 1. The
certainty of logical conclusion is a limiting case
of probability.

(Application to tautology and contradiction.)}


\PropositionE{5.153}
{A proposition is in itself neither probable nor
% -----File: 113.png---
improbable. An event occurs or does not occur,
there is no middle course.}


\PropositionE{5.154}
{In an urn there are equal numbers of white
and black balls (and no others). I draw one
ball after another and put them back in the
urn. Then I can determine by the experiment
that the numbers of the black and white balls
which are drawn approximate as the drawing
continues.

So \emph{this} is not a mathematical fact.

If then, I say, It is equally probable that
I should draw a white and a black ball, this
means, All the circumstances known to me (including
the natural laws hypothetically assumed)
give to the occurrence of the one event no more
probability than to the occurrence of the other.
That is they give---as can easily be understood
from the above explanations---to each the
probability $\frac{1}{2}$.

What I can verify by the experiment is that
the occurrence of the two events is independent
of the circumstances with which I have no closer
acquaintance.}


\PropositionE{5.155}
{The unit of the probability proposition is: The
circumstances---with which I am not further acquainted---give
to the occurrence of a definite event
such and such a degree of probability.}


\PropositionE{5.156}
{Probability is a generalization.

It involves a general description of a propositional
form. Only in default of certainty do we
need probability.

If we are not completely acquainted with a fact,
but know \emph{something} about its form.

(A proposition can, indeed, be an incomplete
picture of a certain state of affairs, but it is always
\emph{a} complete picture.)
% -----File: 115.png---

The probability proposition is, as it were, an
extract from other propositions.}


\PropositionE{5.2}
{The structures of propositions stand to one
another in internal relations.}


\PropositionE{5.21}
{We can bring out these internal relations in
our manner of expression, by presenting a proposition
as the result of an operation which produces
it from other propositions (the bases of the
operation).}


\PropositionE{5.22}
{The operation is the expression of a relation
between the structures of its result and its
bases.}


\PropositionE{5.23}
{The operation is that which must happen to a
proposition in order to make another out of it.}


\PropositionE{5.231}
{And that will naturally depend on their formal
properties, on the internal similarity of their
forms.}


\PropositionE{5.232}
{The internal relation which orders a series is
equivalent to the operation by which one term
arises from another.}


\PropositionE{5.233}
{The first place in which an operation can occur
is where a proposition arises from another in a
logically significant way; \idEst\ where the logical
construction of the proposition begins.}


\PropositionE{5.234}
{The truth-functions of elementary \DPtypo{proposition}{propositions},
are results of operations which have the elementary
propositions as bases. (I call these
operations, truth-operations.)}


\PropositionE{5.2341}
{The sense of a truth-function of $p$ is a function
of the sense of $p$.

Denial, logical addition, logical multiplication,
etc.\ etc., are operations.

(Denial reverses the sense of a proposition.)}


\PropositionE{5.24}
{An operation shows itself in a variable; it shows
how we can proceed from one form of proposition
to another.

It gives expression to the difference between
the forms.
% -----File: 117.png---

(And that which is common to the bases, and
the result of an operation, is the bases themselves.)}


\PropositionE{5.241}
{The operation does not characterize a form but
only the difference between forms.}


\PropositionE{5.242}
{The same operation which makes ``$q$'' from
``$p$'', makes ``$r$'' from ``$q$'', and so on. This
can only be expressed by the fact that ``$p$'', ``$q$'',
``$r$'', etc., are variables which give general expression
\enlargethispage{10pt} % enlarge to make last line fit
to certain formal relations.}


\PropositionE{5.25}
{The occurrence of an operation does not characterize
the sense of a proposition.

For an operation does not assert anything; only
its result does, and this depends on the bases of
the operation.

(Operation and function must not be confused
with one another.)}


\PropositionE{5.251}
{A function cannot be its own argument, but
the result of an operation can be its own
basis.}


\PropositionE{5.252}
{Only in this way is the progress from term
to term in a formal series possible (from type
to type in the hierarchy of Russell and Whitehead).
(Russell and Whitehead have not admitted
the possibility of this progress but have made use
of it all the same.)}


\PropositionE{5.2521}
{The repeated application of an operation to
its own result I call its successive application
(``$O' O' O' a$'' is the result of the threefold successive
application of ``$O' \xi$'' to ``$a$'').

In a similar sense I speak of the successive
application of \emph{several} operations to a number of
propositions.}


\PropositionE{5.2522}
{The general term of the formal series $a, O' a,
O' O' a$,\;$\fourdots$\ I write thus: ``[$a$, $x$, $O' x$]''. This
expression in brackets is a variable. The first
term of the expression is the beginning of the
% -----File: 119.png---
formal series, the second the form of an arbitrary
term $x$ of the series, and the third the form
of that term of the series which immediately
follows $x$.}


\PropositionE{5.2523}
{The concept of the successive application of
an operation is equivalent to the concept ``and
so on''.}


\PropositionE{5.253}
{One operation can reverse the effect of another.
Operations can cancel one another.}


\PropositionE{5.254}
{Operations can vanish (\exempliGratia\ denial in ``$\Not{\Not{p}}$''
$\Not{\Not{p}} = p$).}


\PropositionE{5.3}
{All propositions are results of truth-operations
on the elementary propositions.

The truth-operation is the way in which a
truth-function arises from elementary propositions.

According to the nature of truth-operations,
in the same way as out of elementary propositions
arise their truth-functions, from truth-func\-tions
arises a new one. Every truth-operation
creates from truth-functions of elementary propositions
another truth-func\-tion of elementary
propositions, \idEst\ a proposition. The result of
every truth-operation on the results of truth-op\-er\-a\-tions
on elementary propositions is also
the result of \emph{one} truth-operation on elementary
propositions.

Every proposition is the result of truth-operations
on elementary propositions.}


\PropositionE{5.31}
{The Schemata No.~\PropERef{4.31} are also significant, if
``$p$'', ``$q$'', ``$r$'', etc.\ are not elementary propositions.

And it is easy to see that the propositional
sign in No.~\DPtypo{\PropERef{4.42}}{\PropERef{4.442}} expresses one truth-function of
elementary propositions even when ``$p$'' and
``$q$'' are truth-functions of elementary propositions.}


\PropositionE{5.32}
{All truth-functions are results of the successive
% -----File: 121.png---
application of a finite number of truth-operations
to elementary propositions.}


\PropositionE{5.4}
{Here it becomes clear that there are no such
things as ``logical objects'' or ``logical constants''
(in the sense of Frege and Russell).}


\PropositionE{5.41}
{For all those results of truth-operations on truth-functions
are identical, which are one and the same
truth-function of elementary propositions.}


\PropositionE{5.42}
{That $\lor$, $\Implies$, etc., are not relations in the sense of
right and left, etc., is obvious.

The possibility of crosswise definition of the
logical ``primitive signs'' of Frege and Russell
shows by itself that these are not primitive signs
and that they signify no relations.

And it is obvious that the ``$\Implies$'' which we define
by means of ``$\Not{}$'' and ``$\lor$'' is identical with that
by which we define ``$\lor$'' with the help of ``$\Not{}$'', and
that this ``$\lor$'' is the same as the first, and
so on.}


\PropositionE{5.43}
{That from a fact $p$ an infinite number of \emph{others}
should follow, namely $\Not{\Not{p}}$, $\Not{\Not{\Not{\Not{p}}}}$, etc., is
indeed hardly to be believed, and it is no less
wonderful that the infinite number of propositions
of logic (of mathematics) should follow from half
a dozen ``primitive propositions''.

But all propositions of logic say the same thing.
That is, nothing.}


\PropositionE{5.44}
{Truth-functions are not material functions.

If \exempliGratia\ an affirmation can be produced by
repeated denial, is the denial---in any sense---contained
in the affirmation?

Does ``$\Not{\Not{p}}$'' deny $\Not{p}$, or does it affirm $p$;
or both?

The proposition ``$\Not{\Not{p}}$'' does not treat of
denial as an object, but the possibility of denial is
already prejudged in affirmation.

And if there was an object called ``$\Not{}$'', then
% -----File: 123.png---
``$\Not{\Not{p}}$'' would have to say something other than
``$p$''. For the one proposition would then treat
of $\Not{}$, the other would not.}


\PropositionE{5.441}
{This disappearance of the apparent logical
constants also occurs if ``$\Not{(\exists x) \DotOp \Not{fx}}$'' says the
same as ``$(x) \DotOp fx$'', or ``$(\exists x) \DotOp fx \DotOp x = a$'' the same
as ``$fa$''.}


\PropositionE{5.442}
{If a proposition is given to us then the results
of all truth-operations which have it as their basis
are given \emph{with} it.}


\PropositionE{5.45}
{If there are logical primitive signs a correct logic
must make clear their position relative to one
another and justify their existence. The construction
of logic \emph{out of} its primitive signs must become
clear.}


\PropositionE{5.451}
{If logic has primitive ideas these must be
independent of one another. If a primitive idea
is introduced it must be introduced in all contexts
in which it occurs at all. One cannot therefore
introduce it for \emph{one} context and then again for
another. For example, if denial is introduced,
we must understand it in propositions of the form
``$\Not{p}$'', just as in propositions like ``$\Not{(p \lor q)}$'',
``$(\exists x) \DotOp \Not{fx}$'' and others. We may not first
introduce it for one class of cases and then for
another, for it would then remain doubtful whether
its meaning in the two cases was the same, and
there would be no reason to use the same way of
symbolizing in the two cases.

(In short, what Frege (``Grundgesetze der
Arithmetik'') has said about the introduction of
signs by definitions holds, mutatis mutandis, for
the introduction of primitive signs also.)}


\PropositionE{5.452}
{The introduction of a new expedient in the
symbolism of logic must always be an event full
of consequences. No new symbol may be introduced
in logic in brackets or in the margin---with,
so to speak, an entirely innocent face.
% -----File: 125.png---

(Thus in the ``Principia Mathematica'' of
Russell and Whitehead there occur definitions
and primitive propositions in words. Why suddenly
words here? This would need a justification.
There was none, and can be none for the
process is actually not allowed.)

But if the introduction of a new expedient has
proved necessary in one place, we must immediately
ask: Where is this expedient \emph{always} to be
used? Its position in logic must be made
clear.}


\PropositionE{5.453}
{All numbers in logic must be capable of
justification.

Or rather it must become plain that there are
no numbers in logic.

There are no pre-eminent numbers.}


\PropositionE{5.454}
{In logic there is no side by side, there can be
no classification.

In logic there cannot be a more general and a
more special.}


\PropositionE{5.4541}
{The solution of logical problems must be simple
for they set the standard of simplicity.

Men have always thought that there must be a
sphere of questions whose answers---a priori---are
symmetrical and united into a closed regular
structure.

A sphere in which the proposition, simplex
sigillum veri, is valid.}


\PropositionE{5.46}
{When we have rightly introduced the logical
signs, the sense of all their combinations has been
already introduced with them: therefore not only
``$p \lor q$'' but also ``$\Not{(p \lor \Not{q})}$'', etc.\ etc. We should
then already have introduced the effect of all
possible combinations of brackets; and it would
then have become clear that the proper general
primitive signs are not ``$p \lor q$'', ``$(\exists x) \DotOp fx$'', etc.,
% -----File: 127.png---
but the most general form of their combinations.}


\PropositionE{5.461}
{The apparently unimportant fact that the apparent
relations like \DPtypo{$v$}{$\lor$} and $\Implies$ need brackets---unlike
real relations is of great importance.

The use of brackets with these apparent primitive
signs shows that these are not the real
primitive signs; and nobody of course would
believe that the brackets have meaning by themselves.}


\PropositionE{5.4611}
{Logical operation signs are punctuations.}


\PropositionE{5.47}
{It is clear that everything which can be said
\emph{beforehand} about the form of \emph{all} propositions at
all can be said \emph{on one occasion}.

For all logical operations are already contained
in the elementary proposition. For ``$fa$'' says
the same as ``$(\exists x) \DotOp fx \DotOp x = a$''.

Where there is composition, there is argument
and function, and where these are, all logical
constants already are.

One could say: the one logical constant is that
which \emph{all} propositions, according to their nature,
have in common with one another.

That however is the general form of proposition.}


\PropositionE{5.471}
{The general form of proposition is the essence
of proposition.}


\PropositionE{5.4711}
{To give the essence of proposition means to
give the essence of all description, therefore the
essence of the world.}


\PropositionE{5.472}
{The description of the most general propositional
form is the description of the one and only
general primitive sign in logic.}


\PropositionE{5.473}
{Logic must take care of itself.

A \emph{possible} sign must also be able to signify.
Everything which is possible in logic is also
permitted. (``Socrates is identical'' means nothing
% -----File: 129.png---
because there is no property which is called
``identical''. The proposition is senseless because
we have not made some arbitrary determination,
not because the symbol is in itself unpermissible.)

In a certain sense we cannot make mistakes in
logic.}


\PropositionE{5.4731}
{Self-evidence, of which Russell has said so
much, can only be discarded in logic by language
itself preventing every logical mistake. That
logic is a priori consists in the fact that we \emph{cannot}
think illogically.}


\PropositionE{5.4732}
{We cannot give a sign the wrong sense.}


\PropositionE{5.47321}
{Occam's razor is, of course, not an arbitrary rule
nor one justified by its practical success. It simply
says that \emph{unnecessary} elements in a symbolism
mean nothing.

Signs which serve \emph{one} purpose are logically
equivalent, signs which serve \emph{no} purpose are
logically meaningless.}


\PropositionE{5.4733}
{Frege says: Every legitimately constructed
proposition must have a sense; and I say: Every
possible proposition is legitimately constructed,
and if it has no sense this can only be because
we have given no \emph{meaning} to some of its constituent
parts.

(Even if we believe that we have done
so.)

Thus ``Socrates is identical'' says nothing,
because we have given \emph{no} meaning to the word
``identical'' as \emph{adjective}. For when it occurs as
the sign of equality it symbolizes in an entirely
different way---the symbolizing relation is another---therefore
the symbol is in the two cases entirely
different; the two symbols have the sign in
common with one another only by accident.}
% -----File: 131.png---


\PropositionE{5.474}
{The number of necessary fundamental operations
depends \emph{only} on our notation.}


\PropositionE{5.475}
{It is only a question of constructing a system
of signs of a definite number of di\-men\-sions---of
a definite mathematical multiplicity.}


\PropositionE{5.476}
{It is clear that we are not concerned here with
a \emph{number of primitive ideas} which must be signified
but with the expression of a rule.}


\PropositionE{5.5}
{Every truth-function is a result of the successive
application of the operation \mbox{(- - - - -T)}\AllowBreak($\xi, \fourdots$) to
elementary propositions.

This operation denies all the propositions in
the right-hand bracket and I call it the negation
of these propositions.}


\PropositionE{5.501}
{An expression in brackets whose terms are
propositions I in\-di\-cate---if the order of the terms
in the bracket is indifferent---by a sign of the form
``($\overline{\xi}$)''. ``$\xi$'' is a variable whose values are the
terms of the expression in brackets, and the line
over the variable indicates that it stands for all
its values in the bracket.

(Thus if $\xi$ has the 3 values P, Q, R, then
($\overline{\xi}$) = (P, Q, R).)

The values of the variables must be determined.

{\stretchyspace
The determination is the description of the propositions
which the variable stands for.}

How the description of the terms of the expression
in brackets takes place is unessential.

We may distinguish 3 kinds of description:
1. Direct enumeration. In this case we can place
simply its constant values instead of the variable.
2. Giving a function $fx$, whose values for all
values of $x$ are the propositions to be described.
3. Giving a formal law, according to which those
propositions are constructed. In this case the
% -----File: 133.png---
terms of the expression in brackets are all the
terms of a formal series.}


\PropositionE{5.502}
{Therefore I write instead of \mbox{``(- - - - -T)}\AllowBreak($\xi, \fourdots$)'',
``$N(\overline{\xi})$''.

$N(\overline{\xi})$ is the negation of all the values of the
propositional variable $\xi$.}


\PropositionE{5.503}
{As it is obviously easy to express how propositions
can be constructed by means of this operation
and how propositions are not to be constructed by
means of it, this must be capable of exact expression.}


\PropositionE{5.51}
{If $\xi$ has only one value, then $N(\overline{\xi}) = \Not{p}$ (not $p$),
if it has two values then $N(\overline{\xi}) = \Not{p} \DotOp \Not{q}$ (neither
$p$ nor $q$).}


\PropositionE{5.511}
{How can the all-embracing logic which mirrors
the world use such special catches and manipulations?
Only because all these are connected into
an infinitely fine network, to the great mirror.}


\PropositionE{5.512}
{``$\Not{p}$'' is true if ``$p$'' is false. Therefore in the
true proposition ``$\Not{p}$'' ``$p$'' is a false proposition.
How then can the stroke ``$\Not{}$'' bring it into
agreement with reality?

That which denies in ``$\Not{p}$'' is however not
``$\Not{}$'', but that which all signs of this notation,
which deny $p$, have in common.

Hence the common rule according to which
``$\Not{p}$'', ``$\Not{\Not{\Not{p}}}$'', ``${\Not{p}} \lor {\Not{p}}$'', ``$\Not{p} \DotOp \Not{p}$'',
etc.\ etc.\ (to infinity) are constructed. And this
which is common to them all mirrors denial.}


\PropositionE{5.513}
{We could say: What is common to all symbols,
which assert both $p$ and $q$, is the proposition
``$p \DotOp q$''. What is common to all symbols, which
assert either $p$ or $q$, is the proposition ``$p \lor q$''.

And similarly we can say: Two propositions
are opposed to one another when they have
nothing in common with one another; and every
proposition has only one negative, because there
is only one proposition which lies altogether
outside it.
% -----File: 135.png---

Thus even in Russell's notation it is evident
that ``${q : p} \lor {\Not{p}}$'' says the same as ``$q$''; that
``$p \lor {\Not{p}}$'' says nothing.}


\PropositionE{5.514}
{If a notation is fixed, there is in it a rule according
to which all the propositions denying $p$ are
constructed, a rule according to which all the
propositions asserting $p$ are constructed, a rule
according to which all the propositions asserting
$p$ or $q$ are constructed, and so on. These rules
are equivalent to the symbols and in them their
sense is mirrored.}


\PropositionE{5.515}
{It must be recognized in our symbols that what
is connected by ``$\lor$'', ``$\DotOp$'', etc., must be propositions.

And this is the case, for the symbols ``$p$'' and
``$q$'' presuppose ``$\lor$'', ``$\Not{}$'', etc. If the sign
``$p$'' in ``$p \lor q$'' does not stand for a complex sign,
then by itself it cannot have sense; but then also
the signs ``$p \lor p$'', ``$p \DotOp p$'', etc.\ which have the
same sense as ``$p$'' have no sense. If, however,
``$p \lor p$'' has no sense, then also ``$p \lor q$'' can have
no sense.}


\PropositionE{5.5151}
{Must the sign of the negative proposition be
constructed by means of the sign of the positive?
Why should one not be able to express the
negative proposition by means of a negative fact?
(Like: if ``$a$'' does not stand in a certain relation
to ``$b$'', it could express that $aRb$ is not the case.)

{\stretchyspace
But here also the negative proposition is indirectly
constructed with the positive.}

The positive \emph{proposition} must presuppose the
existence of the negative \emph{proposition} and conversely.}


\PropositionE{5.52}
{If the values of $\xi$ are the total values of a function
$fx$ for all values of $x$, then $N(\overline{\xi}) = \Not{(\exists x) \DotOp fx}$.}


\PropositionE{5.521}
{I separate the concept \emph{all} from the truth-function.

Frege and Russell have introduced generality
in connexion with the logical product or the logical
% -----File: 137.png---
sum. Then it would be difficult to understand
the propositions ``$(\exists x) \DotOp fx$'' and ``$(x) \DotOp fx$'' in which
both ideas lie concealed.}


\PropositionE{5.522}
{That which is peculiar to the ``symbolism of
generality'' is firstly, that it refers to a logical
prototype, and secondly, that it makes constants
prominent.}


\PropositionE{5.523}
{The generality symbol occurs as an argument.}


\PropositionE{5.524}
{If the objects are given, therewith are \emph{all} objects
also given.

If the elementary propositions are given, then
therewith \emph{all} elementary propositions are also
given.}


\PropositionE{5.525}
{It is not correct to render the proposition
\enlargethispage{9pt} % enlarge to make the last line fit
``$(\exists x) \DotOp fx$''---as Russell does---in words ``$fx$ is
\emph{possible}''.

Certainty, possibility or impossibility of a state
of affairs are not expressed by a proposition but
by the fact that an expression is a tautology, a
significant proposition or a contradiction.

That precedent to which one would always
appeal, must be present in the symbol itself.}


\PropositionE{5.526}
{One can describe the world completely by
completely generalized propositions, \idEst\ without
from the outset co-ordinating any name with a
definite object.

In order then to arrive at the customary way
of expression we need simply say after an expression
``there is one and only one $x$, which $\fourdots$'':
and this $x$ is $a$.}


\PropositionE{5.5261}
{A completely generalized proposition is like
every other proposition composite. (This is shown
by the fact that in ``$(\exists x, \phi) \DotOp \phi x$'' we must mention
``$\phi$'' and ``$x$'' separately. Both stand independently
in signifying relations to the world
as in the ungeneralized proposition.)
% -----File: 139.png---

A characteristic of a composite symbol: it has
something in common with \emph{other} symbols.}


\PropositionE{5.5262}
{The truth or falsehood of \emph{every} proposition alters
something in the general structure of the world.
And the range which is allowed to its structure by
the totality of elementary propositions is exactly
that which the completely general propositions
delimit.

(If an elementary proposition is true, then, at
any rate, there is one \emph{more} elementary proposition
true.)}


\PropositionE{5.53}
{Identity of the object I express by identity of
the sign and not by means of a sign of identity.
Difference of the objects by difference of the
signs.}


\PropositionE{5.5301}
{That identity is not a relation between objects is
obvious. This becomes very clear if, for example,
one considers the proposition ``$(x) : fx \DotOp \Implies \DotOp x = a$''.
What this proposition says is simply that \emph{only}
$a$ satisfies the function $f$, and not that only such
things satisfy the function $f$ which have a certain
relation to $a$.

One could of course say that in fact \emph{only}
$a$ has this relation to $a$, but in order to express
this we should need the sign of identity itself.}


\PropositionE{5.5302}
{Russell's definition of ``='' won't do; because
according to it one cannot say that two objects
have all their properties in common. (Even if
this proposition is never true, it is nevertheless
\emph{significant}.)}


\PropositionE{5.5303}
{Roughly speaking: to say of \emph{two} things that
they are identical is nonsense, and to say of \emph{one}
thing that it is identical with itself is to say
nothing.}


\PropositionE{5.531}
{I write therefore not ``$f(a,b) \DotOp a = b$'', but ``$f(a,a)$''
(or ``$f(b,b)$''). And not ``$f(a,b) \DotOp \Not{a} = b$'', but
``$f(a,b)$''.}


\PropositionE{5.532}
{And analogously: not ``$(\exists x,y) \DotOp f(x,y) \DotOp x = y$'',
% -----File: 141.png---
but ``$(\exists x) \DotOp f(x,x)$''; and not ``$(\exists x,y) \DotOp f(x,y) \DotOp
\Not{x} = y$'', but ``$(\exists x,y) \DotOp f(x,y)$''.

(Therefore instead of Russell's ``$(\exists x,y) \DotOp f(x,y)$'':
``$(\exists x,y) \DotOp f(x,y) \DotOp \lor \DotOp (\exists x) \DotOp f(x,x)$''.)}


\PropositionE{5.5321}
{Instead of ``$(x) : fx \Implies x = a$'' we therefore write
\exempliGratia\ ``$(\exists x) \DotOp fx \DotOp \Implies \DotOp fa : \Not{(\exists x,y) \DotOp fx \DotOp fy}$''.

And the proposition ``\emph{only} one $x$ satisfies $f()$''
reads: ``$(\exists x) \DotOp fx : \Not{(\exists x,y) \DotOp fx \DotOp fy}$''.}


\PropositionE{5.533}
{The identity sign is therefore not an essential
constituent of logical notation.}


\PropositionE{5.534}
{And we see that apparent propositions like:
``$a = a$'', ``$a = b \DotOp b = c \DotOp \Implies a = c$'', ``$(x) \DotOp x = x$'', ``$(\exists x) \DotOp
x = a$'', etc.\ cannot be written in a correct logical
notation at all.}


\PropositionE{5.535}
{So all problems disappear which are connected
with such pseu\-do-prop\-o\-si\-tions.

This is the place to solve all the problems which
arise through Russell's ``Axiom of Infinity''.

What the axiom of infinity is meant to say
would be expressed in language by the fact that
there is an infinite number of names with different
meanings.}


\PropositionE{5.5351}
{There are certain cases in which one is tempted
to use expressions of the form ``$a = a$'' or ``$p \Implies p$''
and of that kind. And indeed this takes place
when one would like to speak of the archetype
Proposition, Thing, etc. So Russell in the \BookTitle{Principles
of Mathematics} has rendered the nonsense ``$p$
is a proposition'' in symbols by ``$p \Implies p$'' and has
put it as hypothesis before certain propositions to
show that their places for arguments could only
be occupied by propositions.

(It is nonsense to place the hypothesis $p \Implies p$
before a proposition in order to ensure that its
arguments have the right form, because the
hypothesis for a non-proposition as argument
becomes not false but meaningless, and because
the proposition itself becomes senseless for arguments
% -----File: 143.png---
of the wrong kind, and therefore it survives
the wrong arguments no better and no worse
than the senseless hypothesis attached for this
purpose.)}


\PropositionE{5.5352}
{Similarly it was proposed to express ``There are
no things'' by ``$\Not{(\exists x) \DotOp x = x}$''. But even if this
were a proposition---would it not be true if indeed
``There were things'', but these were not identical
with themselves?}


\PropositionE{5.54}
{In the general propositional form, propositions
occur in a proposition only as bases of the truth-operations.}


\PropositionE{5.541}
{At first sight it appears as if there were also a
different way in which one proposition could occur
in another.

Especially in certain propositional forms of
psychology, like ``A thinks, that $p$ is the case'',
or ``A thinks $p$'', etc.

Here it appears superficially as if the proposition
$p$ stood to the object A in a kind of relation.

(And in modern \DPtypo{epistomology}{epistemology} (Russell, Moore,
etc.) those propositions have been conceived in
this way.)}


\PropositionE{5.542}
{But it is clear that ``A believes that $p$'', ``A
thinks $p$'', ``A says $p$'', are of the form ```$p$' says
$p$'': and here we have no co-ordination of a fact
and an object, but a co-ordination of facts by
means of a co-ordination of their objects.}


\PropositionE{5.5421}
{This shows that there is no such thing as the
soul---the subject, etc.---as it is conceived in contemporary
superficial psychology.

A composite soul would not be a soul any
longer.}


\PropositionE{5.5422}
{The correct explanation of the form of the
proposition ``A judges $p$'' must show that it is
impossible to judge a nonsense. (Russell's theory
does not satisfy this condition.)}


\PropositionE{5.5423}
{To perceive a complex means to perceive that
% -----File: 145.png---
its constituents are combined in such and such a
way.

This perhaps explains that the figure
\Illustration{cube}
can be seen in two ways as a cube; and all similar
phenomena. For we really see two different facts.

(If I fix my eyes first on the corners $a$ and only
glance at $b$, $a$ appears in front and $b$ behind, and
vice versa.)}


\PropositionE{5.55}
{We must now answer a priori the question
as to all possible forms of the elementary propositions.

The elementary proposition consists of names.
Since we cannot give the number of names with
different meanings, we cannot give the composition
of the elementary proposition.}


\PropositionE{5.551}
{Our fundamental principle is that every question
which can be decided at all by logic can be decided
without further trouble.

(And if we get into a situation where we need
to answer such a problem by looking at the world,
this shows that we are on a fundamentally wrong
track.)}


\PropositionE{5.552}
{The ``experience'' which we need to understand
logic is not that such and such is the
case, but that something \emph{is}; but that is \emph{no} experience.

Logic \emph{precedes} every experience---that something
is \emph{so}.

It is before the How, not before the What.}
% -----File: 147.png---


\PropositionE{5.5521}
{And if this were not the case, how could
we apply logic? We could say: if there
were a logic, even if there were no world,
how then could there be a logic, since there is a
world?}


\PropositionE{5.553}
{Russell said that there were simple relations
between different numbers of things (individuals).
But between what numbers? And how should
this be decided---by experience?

(There is no pre-eminent number.)}


\PropositionE{5.554}
{The enumeration of any special forms would
be entirely arbitrary.}


\PropositionE{5.5541}
{It should be possible to decide a priori whether,
for example, I can get into a situation in which
I need to symbolize with a sign of a 27-termed
relation.}


\PropositionE{5.5542}
{May we then ask this at all? Can we set out
a sign form and not know whether anything can
correspond to it?

Has the question sense: what must \emph{be} in order
that something can be the case?}


\PropositionE{5.555}
{It is clear that we have a concept of the
elementary proposition apart from its special
logical form.

Where, however, we can build symbols
according to a system, there this system is the
logically important thing and not the single
symbols.

And how would it be possible that I should
have to deal with forms in logic which I can
invent: but I must have to deal with that which
makes it possible for me to invent them.}


\PropositionE{5.556}
{There cannot be a hierarchy of the forms of the
elementary propositions. Only that which we
ourselves construct can we foresee.}


\PropositionE{5.5561}
{Empirical reality is limited by the totality of
objects. The boundary appears again in the
totality of elementary propositions.
% -----File: 149.png---

The hierarchies are and must be independent
of reality.}


\PropositionE{5.5562}
{If we know on purely logical grounds, that
there must be elementary propositions, then this
must be known by everyone who understands the
propositions in their unanalysed form.}


\PropositionE{5.5563}
{All propositions of our colloquial language are
actually, just as they are, logically completely in
order. That most simple thing which we ought to
give here is not a simile of truth but the complete
truth itself.

(Our problems are not abstract but perhaps the
most concrete that there are.)}


\PropositionE{5.557}
{The \emph{application} of logic decides what elementary
propositions there are.

What lies in the application logic cannot
anticipate.

It is clear that logic may not collide with its
application.

But logic must have contact with its application.

Therefore logic and its application may not
overlap one another.}


\PropositionE{5.5571}
{If I cannot give elementary propositions a
priori then it must lead to obvious nonsense to
try to give them.}


\PropositionE{5.6}
{\emph{The limits of my language} mean the limits of my
world.}


\PropositionE{5.61}
{Logic fills the world: the limits of the world
are also its limits.

We cannot therefore say in logic: This and
this there is in the world, that there is not.

For that would apparently presuppose that we
exclude certain possibilities, and this cannot be
the case since otherwise logic must get outside
the limits of the world: that is, if it could
consider these limits from the other side
also.
% -----File: 151.png---

What we cannot think, that we cannot think:
we cannot therefore \emph{say} what we cannot
think.}


\PropositionE{5.62}
{This remark provides a key to the question, to
what extent solipsism is a truth.

In fact what solipsism \emph{means}, is quite correct,
only it cannot be \emph{said}, but it shows itself.

That the world is \emph{my} world, shows itself in the
fact that the limits of the language (the language
which only I understand) mean the limits of \emph{my}
world.}


\PropositionE{5.621}
{The world and life are one.}


\PropositionE{5.63}
{I am my world. (The microcosm.)}


\PropositionE{5.631}
{The thinking, presenting subject; there is no
such thing.

If I wrote a book ``The world as I found it'',
I should also have therein to report on my body
and say which members obey my will and which
do not, etc. This then would be a method of
isolating the subject or rather of showing that in
an important sense there is no subject: that is to
say, of it alone in this book mention could \emph{not} be
made.}


\PropositionE{5.632}
{The subject does not belong to the world but
it is a limit of the world.}


\PropositionE{5.633}
{\emph{Where in} the world is a metaphysical subject to
be noted?

You say that this case is altogether like that of
the eye and the field of sight. But you do \emph{not}
really see the eye.

And from nothing \emph{in the field of sight} can it be
concluded that it is seen from an eye.}


\PropositionE{5.6331}
{For the field of sight has not a form like this:
\Illustration{sight-en}
}
% -----File: 153.png---


\PropositionE{5.634}
{This is connected with the fact that no part of
our experience is also a priori.

Everything we see could also be otherwise.

Everything we can describe at all could also be
otherwise.

There is no order of things a priori.}


\PropositionE{5.64}
{Here we see that solipsism strictly carried out
coincides with pure realism. The I in solipsism
shrinks to an extensionless point and there remains
the reality co-ordinated with it.}


\PropositionE{5.641}
{There is therefore really a sense in which in
philosophy we can talk of a non-psy\-cho\-log\-i\-cal I.

The I occurs in philosophy through the fact
that the ``world is my world''.

The philosophical I is not the man, not the
human body or the human soul of which psychology
treats, but the metaphysical subject, the
limit---not a part of the world.}


\PropositionE{6}
{The general form of truth-function is:
$[\overline{p}, \overline{\xi}, N(\overline{\xi})]$.

This is the general form of proposition.}


\PropositionE{6.001}
{This says nothing else than that every proposition
is the result of successive applications
of the operation $N'(\overline{\xi})$ to the elementary propositions.}


\PropositionE{6.002}
{If we are given the general form of the way in
which a proposition is constructed, then thereby
we are also given the general form of the way in
which by an operation out of one proposition
another can be created.}


\PropositionE{6.01}
{The general form of the operation $\Omega'(\overline{\eta})$ is
therefore: $[\overline{\xi}, N(\overline{\xi})]'${}$(\overline{\eta})$ (= [$\overline{\eta}$, $\overline{\xi}$, $N(\overline{\xi})$]).

This is the most general form of transition from
one proposition to another.}
% -----File: 155.png---


\PropositionE{6.02}
{And thus we come to numbers: I define <br/>
$x = \Omega^{0}{}' x \text{ Def. }$ <br/>
and <br/>
$\Omega'\Omega^{\nu}{}'x = \Omega^{\nu+1}{}'x \text{ Def.}$ <br/>
According, then, to these symbolic rules we
write the series <br/>
$x$, $\Omega'x$, $\Omega'\Omega'x$, $\Omega'\Omega'\Omega'x\fivedots$ <br/>
as: <br/>
$\Omega^{0}{}'x$, $\Omega^{0+1}{}'x$, $\Omega^{0+1+1}{}'x$, $\Omega^{0+1+1+1}{}'x$ \fivedots <br/>
Therefore I write in place of ``$[x, \xi, \Omega'\xi]$'', <br/>
``$[\Omega^{0}{}'x, \Omega^{\nu}{}'x, \Omega^{\nu+1}{}'x]\text{''.}$ <br/>
And I define: <br/>
$0 + 1 = 1\text{ Def.}$ </br>
$0 + 1 + 1 = 2\text{ Def.}$ </br>
$0 + 1 + 1 + 1 = 3\text{ Def.}$ </br>
and so on.}

\PropositionE{6.021}
{A number is the exponent of an operation.}


\PropositionE{6.022}
{The concept number is nothing else than that
which is common to all numbers, the general form
of number.

The concept number is the variable number.

And the concept of equality of numbers is the
general form of all special equalities of numbers.}


\PropositionE{6.03}
{The general form of the cardinal number is:
$[0, \xi, \xi + 1]$.}


\PropositionE{6.031}
{The theory of classes is altogether superfluous
in mathematics.

This is connected with the fact that the generality
which we need in mathematics is not the
\emph{accidental} one.}


\PropositionE{6.1}
{The propositions of logic are tautologies.}


\PropositionE{6.11}
{The propositions of logic therefore say nothing.
(They are the analytical propositions.)}


\PropositionE{6.111}
{Theories which make a proposition of logic
appear substantial are always false. One could
\exempliGratia\ believe that the words ``true'' and ``false''
signify two properties among other properties,
and then it would appear as a remarkable fact
% -----File: 157.png---
that every proposition possesses one of these
properties. This now by no means appears self-evident,
no more so than the proposition ``All
roses are either yellow or red'' would sound even
if it were true. Indeed our proposition now gets
quite the character of a proposition of natural
science and this is a certain symptom of its being
falsely understood.}


\PropositionE{6.112}
{The correct explanation of logical propositions
must give them a peculiar position among all
propositions.}


\PropositionE{6.113}
{It is the characteristic mark of logical propositions
that one can perceive in the symbol alone
that they are true; and this fact contains in itself
the whole philosophy of logic. And so also it is
one of the most important facts that the truth or
falsehood of non-logical propositions can \emph{not} be
recognized from the propositions alone.}


\PropositionE{6.12}
{The fact that the propositions of logic are
tautologies \emph{shows} the for\-mal---log\-i\-cal---prop\-er\-ties
of language, of the world.

That its constituent parts connected together \emph{in
this way} give a tautology characterizes the logic of
its constituent parts.

In order that propositions connected together
in a definite way may give a tautology they
must have definite properties of structure. That
they give a tautology when \emph{so} connected shows
therefore that they possess these properties of
structure.}


\PropositionE{6.1201}
{That \exempliGratia\ the propositions ``$p$'' and ``$\Not{p}$'' in
the connexion ``$\Not{(p \DotOp \Not{p})}$'' give a tautology
shows that they contradict one another. That the
propositions ``$p \Implies q$'', ``$p$'' and ``$q$'' connected
together in the form ``$(p \Implies q) \DotOp (p) : \Implies : (q)$'' give a
tautology shows that $q$ follows from $p$ and $p \Implies q$.
% -----File: 159.png---
That ``$(x) \DotOp fx : \Implies : fa$'' is a tautology shows that
$fa$ follows from $(x) \DotOp fx$, etc.\ etc.}


\PropositionE{6.1202}
{It is clear that we could have used for this
purpose contradictions instead of tautologies.}


\PropositionE{6.1203}
{In order to recognize a tautology as such, we
can, in cases in which no sign of generality occurs
in the tautology, make use of the following intuitive
method: I write instead of ``$p$'', ``$q$'', ``$r$'', etc.,
``$TpF$'', ``$TqF$'', ``$TrF$'', etc. The truth-combinations
I express by brackets, \exempliGratia:
\Illustration[0.35\textwidth]{brackets01-en}
and the co-ordination of the truth or falsity of the
whole proposition with the truth-combinations of
the truth-arguments by lines in the following way:
\Illustration[0.4\textwidth]{brackets02-en}

This sign, for example, would therefore present
the proposition $p \Implies q$. Now I will proceed
to inquire whether such a proposition as $\Not{(p \DotOp \Not{p})}$
(The Law of Contradiction) is a tautology. The
form ``$\Not{\xi}$'' is written in our notation
\Illustration[0.1\textwidth]{brackets03-en}
% -----File: 161.png---
the form ``$\xi \DotOp \eta$'' thus:---
\enlargethispage{-29pt} % force the next sentence to the next page
\Illustration[0.4\textwidth]{brackets04-en}

Hence the proposition $\Not{(p \DotOp \Not{q})}$ runs thus:---
\Illustration[0.3\textwidth]{brackets05-en}

If here we put ``$p$'' instead of ``$q$'' and examine
the combination of the outermost T and F with the
innermost, it is seen that the truth of the whole
proposition is co-ordinated with \emph{all} the truth-combinations
of its argument, its falsity with none of
the truth-combinations.}


\PropositionE{6.121}
{The propositions of logic demonstrate the logical
properties of propositions, by combining them into
propositions which say nothing.

This method could be called a zero-method. In
a logical proposition propositions are brought into
equilibrium with one another, and the state of
equilibrium then shows how these propositions
must be logically constructed.}


\PropositionE{6.122}
{Whence it follows that we can get on without
logical propositions, for we can recognize in an
adequate notation the formal properties of the propositions
by mere inspection.}
% -----File: 163.png---


\PropositionE{6.1221}
{If for example two propositions ``$p$'' and ``$q$''
give a tautology in the connexion ``$p \Implies q$'', then
it is clear that $q$ follows from $p$.

\ExempliGratia\ that ``$q$'' follows from ``$p \Implies q \DotOp p$'' we see from
these two propositions themselves, but we can also
show it by combining them to ``$p \Implies q \DotOp p : \Implies : q$'' and
then showing that this is a tautology.}


\PropositionE{6.1222}
{This throws light on the question why logical
propositions can no more be empirically established
than they can be empirically refuted. Not only
must a proposition of logic be incapable of being
contradicted by any possible experience, but it
must also be incapable of being established by any
such.}


\PropositionE{6.1223}
{It now becomes clear why we often feel as though
``logical truths'' must be ``\emph{postulated}'' by us. We
can in fact postulate them in so far as we can
postulate an adequate notation.}


\PropositionE{6.1224}
{It also becomes clear why logic has been called
the theory of forms and of inference.}


\PropositionE{6.123}
{It is clear that the laws of logic cannot themselves
obey further logical laws.

(There is not, as Russell supposed, for every
``type'' a special law of contradiction; but one is
sufficient, since it is not applied to itself.)}


\PropositionE{6.1231}
{The mark of logical propositions is not their
general validity.

To be general is only to be accidentally valid
for all things. An ungeneralized proposition can
be tautologous just as well as a generalized
one.}


\PropositionE{6.1232}
{Logical general validity, we could call essential
as opposed to accidental general validity, \exempliGratia\ of the
proposition ``all men are mortal''. Propositions
like Russell's ``axiom of reducibility'' are not
% -----File: 165.png---
logical propositions, and this explains our feeling
that, if true, they can only be true by a happy
chance.}


\PropositionE{6.1233}
{We can imagine a world in which the axiom of
reducibility is not valid. But it is clear that logic
has nothing to do with the question whether our
world is really of this kind or not.}


\PropositionE{6.124}
{The logical propositions describe the scaffolding
of the world, or rather they present it. They
``treat'' of nothing. They presuppose that names
have meaning, and that elementary propositions
have sense. And this is their connexion with the
world. It is clear that it must show something
about the world that certain combinations of symbols---which
essentially have a definite character---are
tautologies. Herein lies the decisive point. We
said that in the symbols which we use much is
arbitrary, much not. In logic only this expresses:
but this means that in logic it is not \emph{we} who express,
by means of signs, what we want, but in logic the
nature of the essentially necessary signs itself
asserts. That is to say, if we know the logical
syntax of any sign language, then all the propositions
of logic are already given.}


\PropositionE{6.125}
{It is possible, even in the old logic, to give
at the outset a description of all ``true'' logical
propositions.}


\PropositionE{6.1251}
{Hence there can \emph{never} be surprises in logic.}


\PropositionE{6.126}
{Whether a proposition belongs to logic can be
determined by determining the logical properties
of the \emph{symbol}.

And this we do when we prove a logical proposition.
For without troubling ourselves about
a sense and a meaning, we form the logical
propositions out of others by mere \emph{symbolic
rules}.
% -----File: 167.png---

We prove a logical proposition by creating it
out of other logical propositions by applying in
succession certain operations, which again generate
tautologies out of the first. (And from a tautology
only tautologies \emph{follow}.)

Naturally this way of showing that its propositions
are tautologies is quite unessential to
logic. Because the propositions, from which the
proof starts, must show without proof that they
are tautologies.}


\PropositionE{6.1261}
{In logic process and result are equivalent.
(Therefore no surprises.)}


\PropositionE{6.1262}
{Proof in logic is only a mechanical expedient
to facilitate the recognition of tautology, where
it is complicated.}


\PropositionE{6.1263}
{It would be too remarkable, if one could prove
a significant proposition \emph{logically} from another, and
a logical proposition \emph{also}. It is clear from the
beginning that the logical proof of a significant
proposition and the proof \emph{in} logic must be two
quite different things.}


\PropositionE{6.1264}
{{\stretchyspace
The significant proposition asserts something,
and its proof shows that it is so; in logic every
proposition is the form of a proof.}

Every proposition of logic is a modus ponens
presented in signs. (And the modus ponens can
not be expressed by a proposition.)}


\PropositionE{6.1265}
{Logic can always be conceived to be such that
every proposition is its own proof.}


\PropositionE{6.127}
{All propositions of logic are of equal rank;
there are not some which are essentially primitive
and others deduced from these.

Every tautology itself shows that it is a
tautology.}


\PropositionE{6.1271}
{It is clear that the number of ``primitive propositions
of logic'' is arbitrary, for we could deduce
% -----File: 169.png---
logic from one primitive proposition by simply
forming, for example, the logical product of Frege's
primitive propositions. (Frege would perhaps say
that this would no longer be immediately self-evident.
But it is remarkable that so exact a
thinker as Frege should have appealed to the
degree of self-evidence as the criterion of a
logical proposition.)}


\PropositionE{6.13}
{Logic is not a theory but a reflexion of the
world.

Logic is transcendental.}


\PropositionE{6.2}
{Mathematics is a logical method.

The propositions of mathematics are equations,
and therefore pseudo-prop\-o\-si\-tions.}


\PropositionE{6.21}
{Mathematical propositions express no thoughts.}


\PropositionE{6.211}
{In life it is never a mathematical proposition
which we need, but we use mathematical propositions
\emph{only} in order to infer from propositions
which do not belong to mathematics to others
which equally do not belong to mathematics.

(In philosophy the question ``Why do we really
use that word, that proposition?'' constantly leads
to valuable results.)}


\PropositionE{6.22}
{The logic of the world which the propositions
of logic show in tautologies, mathematics shows
in equations.}


\PropositionE{6.23}
{{\stretchyspace
If two expressions are connected by the sign of
equality, this means that they can be substituted
for one another. But whether this is the case
must show itself in the two expressions themselves.}

It characterizes the logical form of two expressions,
that they can be substituted for one
another.}


\PropositionE{6.231}
{It is a property of affirmation that it can be
conceived as double denial.

It is a property of ``$1 + 1 + 1 + 1$'' that it can be
conceived as ``$(1 + 1) + (1 + 1)$''.}
% -----File: 171.png---


\PropositionE{6.232}
{Frege says that these expressions have the same
meaning but different senses.

But what is essential about equation is that it
is not necessary in order to show that both expressions,
which are connected by the sign of
equality, have the same meaning: for this can be
perceived from the two expressions themselves.}


\PropositionE{6.2321}
{And, that the propositions of mathematics can
be proved means nothing else than that their
correctness can be seen without our having to
compare what they express with the facts as regards
correctness.}


\PropositionE{6.2322}
{The identity of the meaning of two expressions
cannot be \emph{asserted}. For in order to be able to
assert anything about their meaning, I must know
their meaning, and if I know their meaning, I
know whether they mean the same or something
different.}


\PropositionE{6.2323}
{The equation characterizes only the standpoint
from which I consider the two expressions, that
is to say the standpoint of their equality of
meaning.}


\PropositionE{6.233}
{To the question whether we need intuition for
the solution of mathematical problems it must be
answered that language itself here supplies the
necessary intuition.}


\PropositionE{6.2331}
{The process of calculation brings about just
this intuition.

Calculation is not an experiment.}


\PropositionE{6.234}
{Mathematics is a method of logic.}


\PropositionE{6.2341}
{The essential of mathematical method is working
with equations. On this method depends the
fact that every proposition of mathematics must
be self-intelligible.}


\PropositionE{6.24}
{The method by which mathematics arrives at
its equations is the method of substitution.

For equations express the substitutability of
two expressions, and we proceed from a number
% -----File: 173.png---
of equations to new equations, replacing expressions
by others in accordance with the
equations.}


\PropositionE{6.241}
{Thus the proof of the proposition $2 \times 2 = 4$ runs: <br/>
$(\Omega^{\nu})^{\mu}{}'x = \Omega^{\nu \times \mu}{}'x \text{ Def.} $<br/>
$\Omega^{2 \times 2}{}'x = (\Omega^{2})^{2}{}'x = (\Omega^{2})^{1 + 1}{}'x = \Omega^{2}{}'\Omega^{2}{}'x = \Omega^{1 + 1}{}'\Omega^{1 + 1}{}'x$ <br/>
$= (\Omega'\Omega)'(\Omega'\Omega)'x = \Omega'\Omega'\Omega'\Omega'x = \Omega^{1 + 1 + 1 + 1}{}'x = \Omega^{4}{}'x.$
}


\PropositionE{6.3}
{Logical research means the investigation of \emph{all
regularity}. And outside logic all is accident.}


\PropositionE{6.31}
{The so-called law of induction cannot in any
case be a logical law, for it is obviously a significant
proposition.---And therefore it cannot be
a law a priori either.}


\PropositionE{6.32}
{The law of causality is not a law but the form
of a law.\footnote{\IdEst\ not the form of one particular law, but of any law of a certain
sort (B.R.).}}


\PropositionE{6.321}
{``Law of Causality'' is a class name. And as
in mechanics there are, for instance, minimum-laws,
such as that of least action, so in physics
there are causal laws, laws of the causality
form.}


\PropositionE{6.3211}
{Men had indeed an idea that there must be
\emph{a} ``law of least action'', before they knew
exactly how it ran. (Here, as always, the a
priori certain proves to be something purely
logical.)}


\PropositionE{6.33}
{We do not \emph{believe} a priori in a law of conservation,
but we \emph{know} a priori the possibility of
a logical form.}


\PropositionE{6.34}
{All propositions, such as the law of causation,
the law of continuity in nature, the law of least
expenditure in nature, etc.\ etc., all these are
a priori intuitions of possible forms of the propositions
of science.}


\PropositionE{6.341}
{Newtonian mechanics, for example, brings the
description of the universe to a unified form.
% -----File: 175.png---
Let us imagine a white surface with irregular
black spots. We now say: Whatever kind of
picture these make I can always get as near as
I like to its description, if I cover the surface
with a sufficiently fine square network and now
say of every square that it is white or black.
In this way I shall have brought the description
of the surface to a unified form. This form is
arbitrary, because I could have applied with equal
success a net with a triangular or hexagonal
mesh. It can happen that the description would
have been simpler with the aid of a triangular
mesh; that is to say we might have described
the surface more accurately with a triangular,
and coarser, than with the finer square mesh, or
vice versa, and so on. To the different networks
correspond different systems of describing the
world. Mechanics determine a form of description
by saying: All propositions in the description
of the world must be obtained in a given
way from a number of given propositions---the
mechanical axioms. It thus provides the bricks
for building the edifice of science, and says:
Whatever building thou wouldst erect, thou shalt
construct it in some manner with these bricks
and these alone.

(As with the system of numbers one must be
able to write down any arbitrary number, so
with the system of mechanics one must be able
to write down any arbitrary physical proposition.)}


\PropositionE{6.342}
{And now we see the relative position of logic
and mechanics. (We could construct the network
out of figures of different kinds, as out of
triangles and hexagons together.) That a picture
like that instanced above can be described by a
network of a given form asserts \emph{nothing} about
% -----File: 177.png---
the picture. (For this holds of every picture of
this kind.) But \emph{this} does characterize the picture,
the fact, namely, that it can be \emph{completely} described
by a definite net of \emph{definite} fineness.

So too the fact that it can be described by
Newtonian mechanics asserts nothing about the
world; but \emph{this} asserts something, namely, that
it can be described in that particular way in which
it is described, as is indeed the case. The fact,
too, that it can be described more simply by one
system of mechanics than by another says something
about the world.}


\PropositionE{6.343}
{Mechanics is an attempt to construct according
to a single plan all \emph{true} propositions which we
need for the description of the world.}


\PropositionE{6.3431}
{Through the whole apparatus of logic the
physical laws still speak of the objects of the
world.}


\PropositionE{6.3432}
{We must not forget that the description of the
world by mechanics is always quite general.
There is, for example, never any mention of
\emph{particular} material points in it, but always only
of \emph{some points or other}.}


\PropositionE{6.35}
{Although the spots in our picture are geometrical
figures, geometry can obviously say nothing
about their actual form and position. But the
network is \emph{purely} geometrical, and all its properties
can be given a priori.

Laws, like the law of causation, etc., treat
of the network and not of what the network
described.}


\PropositionE{6.36}
{If there were a law of causality, it might run:
``There are natural laws''.

But that can clearly not be said: it shows
itself.}


\PropositionE{6.361}
{In the terminology of Hertz we might say:
Only \emph{uniform} connexions are \emph{thinkable}.}
% -----File: 179.png---


\PropositionE{6.3611}
{We cannot compare any process with the
``passage of time''---there is no such thing---but
only with another process (say, with the movement
of the chronometer).

Hence the description of the temporal sequence
of events is only possible if we support ourselves
on another process.

It is exactly analogous for space. When, for
example, we say that neither of two events (which
mutually exclude one another) can occur, because
there is \emph{no cause} why the one should occur rather
than the other, it is really a matter of our being
unable to describe \emph{one} of the two events unless
there is some sort of asymmetry. And if there \emph{is}
such an asymmetry, we can regard this as the
\emph{cause} of the occurrence of the one and of the non-occurrence
of the other.}


\PropositionE{6.36111}
{The Kantian problem of the right and left hand
which cannot be made to cover one another already
exists in the plane, and even in one-di\-men\-sio\-nal
space; where the two congruent figures $a$ and $b$
cannot be made to cover one another without
moving them out of this space. The right and
left hand are in fact completely congruent. And
the fact that they cannot be made to cover one
another has nothing to do with it.
\Illustration[0.45\textwidth]{space}

A right-hand glove could be put on a left hand
if it could be turned round in four-dimensional
space.}


\PropositionE{6.362}
{What can be described can happen too, and
what is excluded by the law of causality cannot be
described.}


\PropositionE{6.363}
{The process of induction is the process of
% -----File: 181.png---
assuming the \emph{simplest} law that can be made to
harmonize with our experience.}


\PropositionE{6.3631}
{This process, however, has no logical foundation
but only a psychological one.

It is clear that there are no grounds for believing
that the simplest course of events will really
happen.}


\PropositionE{6.36311}
{That the sun will rise to-morrow, is an hypothesis;
and that means that we do not \emph{know} whether
it will rise.}


\PropositionE{6.37}
{A necessity for one thing to happen because
another has happened does not exist. There is
only \emph{logical} necessity.}


\PropositionE{6.371}
{At the basis of the whole modern view of
the world lies the illusion that the so-called
laws of nature are the explanations of natural
phenomena.}


\PropositionE{6.372}
{So people stop short at natural laws as at something
unassailable, as did the ancients at God
and Fate.

And they both are right and wrong. But the
ancients were clearer, in so far as they recognized
one clear conclusion, whereas in the modern
system it should appear as though \emph{everything} were
explained.}


\PropositionE{6.373}
{The world is independent of my will.}


\PropositionE{6.374}
{Even if everything we wished were to happen,
this would only be, so to speak, a favour of
fate, for there is no \emph{logical} connexion between will
and world, which would guarantee this, and the
assumed physical connexion itself we could not
again will.}


\PropositionE{6.375}
{As there is only a \emph{logical} necessity, so there is
only a \emph{logical} impossibility.}


\PropositionE{6.3751}
{For two colours, \exempliGratia\ to be at one place in the
visual field, is impossible, logically impossible,
for it is excluded by the logical structure of
colour.
% -----File: 183.png---

Let us consider how this contradiction presents
itself in physics. Somewhat as follows: That a
particle cannot at the same time have two velocities,
\idEst\ that at the same time it cannot be in
two places, \idEst\ that particles in different places
at the same time cannot be identical.

(It is clear that the logical product of two
elementary propositions can neither be a tautology
nor a contradiction. The assertion that a point
in the visual field has two different colours at the
same time, is a contradiction.)}


\PropositionE{6.4}
{All propositions are of equal value.}


\PropositionE{6.41}
{The sense of the world must lie outside the
world. In the world everything is as it is and
happens as it does happen. \emph{In} it there is no value---and
\enlargethispage{11pt} % enlarge to make the last line fit
if there were, it would be of no value.

If there is a value which is of value, it must
lie outside all happening and being-so. For all
happening and being-so is accidental.

What makes it non-accidental cannot lie \emph{in}
the world, for otherwise this would again be accidental.

It must lie outside the world.}


\PropositionE{6.42}
{Hence also there can be no ethical propositions.

Propositions cannot express anything higher.}


\PropositionE{6.421}
{It is clear that ethics cannot be expressed.

Ethics are transcendental.

(Ethics and &aelig;sthetics are one.)}


\PropositionE{6.422}
{The first thought in setting up an ethical law
of the form ``thou shalt\ldots'' is: And what
if I do not do it. But it is clear that ethics has
nothing to do with punishment and reward in the
ordinary sense. This question as to the \emph{consequences}
of an action must therefore be irrelevant.
At least these consequences will not be events.
For there must be something right in that formulation
of the question. There must be some sort
% -----File: 185.png---
of ethical reward and ethical punishment, but this
must lie in the action itself.

(And this is clear also that the reward must be
something acceptable, and the punishment something
unacceptable.)}


\PropositionE{6.423}
{Of the will as the bearer of the ethical we cannot
speak.

And the will as a phenomenon is only of interest
to psychology.}


\PropositionE{6.43}
{If good or bad willing changes the world, it
can only change the limits of the world, not the
facts; not the things that can be expressed in
language.

In brief, the world must thereby become quite
another. It must so to speak wax or wane as a
whole.

The world of the happy is quite another than
that of the unhappy.}


\PropositionE{6.431}
{As in death, too, the world does not change,
but ceases.}


\PropositionE{6.4311}
{Death is not an event of life. Death is not lived
through.

If by eternity is understood not endless temporal
duration but timelessness, then he lives eternally
who lives in the present.

Our life is endless in the way that our visual
field is without limit.}


\PropositionE{6.4312}
{The temporal immortality of the soul of man,
that is to say, its eternal survival also after
death, is not only in no way guaranteed, but
this assumption in the first place will not do
for us what we always tried to make it do.
Is a riddle solved by the fact that I survive for
ever? Is this eternal life not as enigmatic as
our present one? The solution of the riddle of
life in space and time lies \emph{outside} space and
time.
% -----File: 187.png---

{\stretchyspace
(It is not problems of natural science which have
to be solved.)}}


\PropositionE{6.432}
{\emph{How} the world is, is completely indifferent for
what is higher. God does not reveal himself \emph{in} the
world.}


\PropositionE{6.4321}
{The facts all belong only to the task and not to
its performance.}


\PropositionE{6.44}
{Not \emph{how} the world is, is the mystical, but \emph{that}
it is.}


\PropositionE{6.45}
{The contemplation of the world sub specie aeterni
is its contemplation as a limited whole.

The feeling of the world as a limited whole is
the mystical feeling.}


\PropositionE{6.5}
{For an answer which cannot be expressed the
question too cannot be expressed.

\emph{The riddle} does not exist.

If a question can be put at all, then it \emph{can} also
be answered.}


\PropositionE{6.51}
{Scepticism is \emph{not} irrefutable, but palpably senseless,
if it would doubt where a question cannot be
asked.

For doubt can only exist where there is a
question; a question only where there is an answer,
and this only where something \emph{can} be \emph{said}.}


\PropositionE{6.52}
{We feel that even if \emph{all possible} scientific
questions be answered, the problems of life have
still not been touched at all. Of course there is
then no question left, and just this is the
answer.}


\PropositionE{6.521}
{The solution of the problem of life is seen in the
vanishing of this problem.

(Is not this the reason why men to whom
after long doubting the sense of life became
clear, could not then say wherein this sense
consisted?)}


\PropositionE{6.522}
{There is indeed the inexpressible. This \emph{shows}
itself; it is the mystical.}


\PropositionE{6.53}
{The right method of philosophy would be this.
% -----File: 189.png---
To say nothing except what can be said, \idEst\ the
propositions of natural science, \idEst\ something that
has nothing to do with philosophy: and then
always, when someone else wished to say something
metaphysical, to demonstrate to him that he
had given no meaning to certain signs in his
propositions. This method would be unsatisfying
to the other---he would not have the feeling that
we were teaching him philosophy---but it would be
the only strictly correct method.}


\PropositionE{6.54}
{My propositions are elucidatory in this way:
he who understands me finally recognizes them as
senseless, when he has climbed out through them,
on them, over them. (He must so to speak throw
away the ladder, after he has climbed up on it.)

He must surmount these propositions; then he
sees the world rightly.}


\PropositionE{7}
{Whereof one cannot speak, thereof one must be
silent.}
